\documentclass[10pt]{article}%a4paper
\usepackage[dutch,american]{babel}
\usepackage{palatino,csquotes}

\usepackage[a5paper,           %paper size
                hmargin=1.0cm, %horizontal margin = 1 cm on each side
                vmargin=1.0cm, %vertical margin = 1 cm top and bottom
                tmargin=1.2cm, %top margin = 1.2 cm
                bmargin=0.8cm, %bottom margin = 0.8 cm
                ]{geometry}

\usepackage[a4,center,noinfo]{crop}

\newcommand{\mystelling}[1]{%
\item #1
}
%
\newcommand*\crosshairsulc{% % crosshairs
\begin{picture}(0,0)
\linethickness{0.2pt}
\put(-35,0){\line(1,0){30}}
\put(0,35){\line(0,-1){30}}
\end{picture}%
}
%
\newcommand*\crosshairsurc{% % crosshairs
\linethickness{0.2pt}
\begin{picture}(0,0)
\put(35,0){\line(-1,0){30}}
\put(0,35){\line(0,-1){30}}
\end{picture}%
}
%
\newcommand*\crosshairsllc{% % crosshairs
\linethickness{0.2pt}
\begin{picture}(0,0)
\put(-35,0){\line(1,0){30}}
\put(0,-35){\line(0,1){30}}
\end{picture}%
}
%
\newcommand*\crosshairslrc{% % crosshairs
\linethickness{0.2pt}
\begin{picture}(0,0)
\put(35,0){\line(-1,0){30}}
\put(0,-35){\line(0,1){30}}
\end{picture}%
}
%
\cropdef\crosshairsulc\crosshairsurc\crosshairsllc\crosshairslrc{crosshairs}
%
%
\crop[crosshairs] %% uncomment for crosshairs
%\crop[frame] %% uncomment for frame
%
\pagestyle{empty}
\begin{document}
\hyphenation{met-ro-nome}
\selectlanguage{dutch}
\vspace*{1.5cm}
\begin{center}
{\LARGE \bfseries Stellingen}\\[5mm]
{\Large behorende bij het proefschrift}\\[2cm]
{\LARGE \bfseries Robustness Analysis and Controller Synthesis for Bilateral Teleoperation Systems via IQCs}\\[4mm]
\vspace{2cm}
{\Large \.Ilhan POLAT \\[1cm]
4 februari 2014}
\end{center}
%
\newpage
%
%
\selectlanguage{american}
\begin{enumerate}
% 1
\mystelling{Our sense and perception systems are quite imperfect and we are aware of it. Optical illusions, 
3D movies, surround audio systems all take advantage of this fact as our brain gracefully fails. The key to success 
in touch related technologies lies in the same realm (This thesis).} 
% 2
\mystelling{Decades after its official formulation, we are still unable to at least enumerate quantitative 
properties of a bilateral teleoperation system performance index. (This thesis).}
% 3
\mystelling{It is possible to obtain a satisfactory level of realism via a model-based control design (This thesis).}
% 4
\mystelling{The brain can control the human arm to behave as a passive physical system but only when it 
wishes to. Passivity is not an essential property of the human arm (This thesis).}
% 5
\mystelling{When Nietzsche wrote \enquote{\emph{Unexplained, obscure matters are regarded 
as more important than explained, clear ones}}, he should have been admitted to the 
nearest control-theory department.}
% 6
\mystelling{In \emph{Apprenticeship of a Mathematician}, Andr\'{e} Weil states that 
\enquote{\emph{It is all too widely believed that it is better to misspend a sum of 
money than not to have use of it all}}. Clearly, he never applied to a project grant
since otherwise he would have immediately noticed the unfortunate zero-sum structure.
}
% 7
\mystelling{When a scientific field needs a different perspective to look at the questions 
at hand, there's nothing more annoying than a bunch of disciples.}
% 8
\mystelling{Using expensive and often poorly performing proprietary software for research 
is both a terrible mistake and a waste of scarce resources. It is not a valid excuse to 
argue that it would cost time to build something from scratch. This practice forces the
society to buy software to use the tools provided by the academia which is funded already by 
the very same society.}
% 9
\mystelling{Fifty years from now people will still see themselves as modern.}
% 10
\mystelling{If you can't hear the click anymore, you are either perfectly on the beat or your metronome is broken.}
%
\end{enumerate}

\vfill%

\noindent These propositions are considered opposable and defendable and as such have been approved by the supervisors 
prof.dr.ir. Maarten Steinbuch en prof.dr. Siep Weiland.
\newpage
%----------------------------------------------------------------------------------------------------------------------
%
\selectlanguage{dutch}
%
\begin{enumerate}
\mystelling{Onze zintuigen en perceptiesystemen zijn niet bepaald perfect, en daar zijn we ons bewust van. Optische illusies, 3D films, 
surround audio systemen gebruiken het feit dat het ons brein niet altijd lukt. Dit gegeven is de sleutel tot succes in technologieën 
omtrent aanraking (Dit proefschrift).}
%
%
%% 2
\mystelling{Decennia nadat de index geformuleerd werd, zijn we nog altijd niet in staat om enkele kwantitatieve eigenschappen van een 
\enquote{performance index} van een bilateraal teleoperatie-systeem op te noemen (Dit proefschrift).}
%
%% 3
\mystelling{Het is mogelijk om een acceptabel realisme-niveau te bereiken met een control design gebaseerd op een model (Dit proefschrift).}
%
%% 4
\mystelling{Het brein is in staat om de menselijke arm te dwingen zich als een passief fysisch systeem te gedragen, maar alleen als het 
dat wil. Passiviteit is niet een inherente eigenschap van de menselijke arm (Dit proefschrift)}
%
%% 5
\mystelling{Toen Nietzsche schreeft dat \enquote{onverklaarde, obscure zaken worden gezien als belangrijker dan verklaarde, duidelijke}, 
had hij toegelaten moeten worden tot de dichtstbijzijnde regeltechniek faculteit.}
%
%% 6
\mystelling{In \enquote{The apprenticeship of a mathematician} beweert André Weil dat \enquote{het een weidverbreid geloof is dat je beter 
een som geld fout kunt besteden dan het niet uitgeven}. Het is duidelijk dat hij nooit een aanvraag heeft ingediend voor een subsidie. 
Anders had hij direct de zero-sum structuur gezien.}
%
%% 7
\mystelling{Als een bepaalde wetenschap een nieuw perspectief nodig heeft waarmee het naar belangrijke vraagstukken kan kijken, is er 
niets vervelender dan een stelletje discipels.}
%
%% 8
\mystelling{Het gebruik van dure en vaak slecht werkende gepatenteerde of beschermde software is zowel een grote fout als een verspilling 
van schaarse middelen. Dat het tijd kost om vanaf nul zelf iets te bouwen is geen valide excuus. Deze gewoonte dwingt de samenleving om 
software te kopen om wetenschappelijke tools te gebruiken die reeds door diezelfde samenleving gefinancierd zijn.}
%
%% 9
\mystelling{Over vijftig jaar zullen mensen zich nog steeds als modern zien.}
%
%% 10
\mystelling{Als je de tik niet meer hoort, sla je perfect op de beat. Dat, of je metronoom is kapot.}
\end{enumerate}
\vfill
%
\noindent Deze stellingen worden opponeerbaar en verdedigbaar geacht en zijn als zodanig goedgekeurd door de promotoren 
prof.dr.ir. Maarten Steinbuch en prof.dr. Siep Weiland.
%%----------------------------------------------------------------------------------------------------------------------------------------
\end{document}


\chapter{Synthesis}
\label{chap:synth}

As we have shown in the previous chapters, the bilateral teleoperation problem can be carried over to a more general
framework of IQCs. By doing so not only the existing results are preserved, but also substantial generalization
options are obtained. Moreover, the shortcomings of the classical results and the underlying relations are clearly 
seen. 

Hence, we can proceed with the controller design problem and obtain controllers using the results for IQC synthesis
literature.

\section{Robust Controller Design using IQCs}
LMI-based control synthesis methods are in general nonconvex and even the existence of a possible convexification step 
is not known. The $\mu$-synthesis and Lyapunov-based synthesis methods are the two main classical ways of designing
a robust controller. 

Clearly, this problem is not much different for IQC synthesis which is closely related to $\mu$
tools. As we have shown in \cref{chap:analysis}, the dynamic multipliers are of the most importance when it comes to 
reducing the conservatism involved in the frequency-domain stability analysis techniques. However including dynamics 
to the multipliers makes the problem harder for the robust control synthesis problem. Since there is no known convex 
robust control design method known, we resort back to a nonconvex multiplier/controller iteration as is the case with 
the classical $\mu$-tools with the so-called $DK$-iteration. However, it has been recently shown that, a large class
of robust control design problems are convex depending on the uncertainty contribution on certain entries of the plant 
data, \cite{scherer2009}. 
\subsection{Notation and Definitions}
We start with state-space descriptions of systems in the following form 
\begin{equation}
\pmatr{\dot{x}\\ z\\ y} = 
\underbrace{\left(
\begin{array}{ccc}
	A    &B_w    &B_u\\
	C_z  &D_{wz} &D_{uz}\\
	C_y  &D_{wy} &D_{uy}
\end{array}
\right)}_{G}
\pmatr{x\\ w\\ u}
\label{eq:gennomplant}
\end{equation}
Here, $w$ denotes the generalized disturbance signals (perturbations, reference signals etc.), $z$ denotes the controlled
output signals that is selected as the objectives of the control design (minimization of error norms, control actions etc.)
$y$ signals are the measurements and $u$ are the controller inputs. Typically, the second row is called the \emph{performance 
channel} and the last row is, similarly, the \emph{control channel}. The integers $n,n_u,n_w,m_z,m_y$ denote the number of 
states, control inputs, disturbance inputs, control outputs, and measurements respectively. For the performance characterization, 
we also use a quadratic form 
\begin{equation}
\int_0^\infty{\pmatr{w(t) \\ z(t)}^TP_p \pmatr{w(t) \\ z(t)} dt} \leq -\epsilon\norm{w}^2.
\label{eq:quadperf}
\end{equation}
As a typical example, the robust $\mathcal{L}_2$ gain from $w$ to $z$ can be rewritten as such by using 
\[
P_p = \pmatr{-\gamma^2I_{n_w} &0\\ 0&I_{m_z}}
\]
We will also be referring to the state-space representation of a controller $K$ via
\begin{equation}
\pmatr{\dot{x}_K\\ u} = 
\pmatr{
	A_K  &B_K\\
	C_K  &D_K
}
\pmatr{x_K\\ y}
\label{eq:contmats}
\end{equation}
with $n_K$ being the number of states of the controller. The common assumption of $D_{uy}=0$ is also adopted here. This causes no 
loss of generality as long as the interconnection of $G$-$K$ interconnection is well-posed i.e. $(I-D_{uy}D_K)$ is nonsingular. Then, 
the closed loop plant $(G\star K)(s)$ state-space matrices are obtained as
\begin{align}
\left[\begin{array}{c|c}
\mathcal{A}	&\mathcal{B} \\\hline
\mathcal{C}	&\mathcal{D}
\end{array}
\right] &= \left[\begin{array}{cc|c}
A + B_uD_KC_y        & B_uC_K    & B_w + B_uD_KD_{wy} \\
B_KC_y               & A_K       & B_KD_{wy}\\ \hline
C_z +  D_{uz}D_KC_y  & D_{uz}C_K & D_{wz} + D_{uz}D_KD_{wy}
\end{array}\right] \\
&= \left[
\begin{array}{cc|c}
A   & 0  & B_w \\
0   & 0  & 0\\ \hline
C_z & 0  & D_{wz}
\end{array}
\right] + 
\left[
\begin{array}{cc}
0  & B_u \\
I  & 0\\ \hline
0  & D_{uz}
\end{array}
\right]
\left[
\begin{array}{cc}
A_K  & B_K \\
C_K  & D_K
\end{array}
\right]
\left[
\begin{array}{cc|c}
0   & I & 0\\
C_y & 0 & D_{wy}
\end{array}
\right].
\label{eq:nominalclinc}
\end{align}
Here two-letter subscripts, $\cdot_{ab}$ denote the term that represents the contribution from input $a$ to output $b$ which hopefully gives some
relief to the reader since the manupulations quickly get ugly. 

\begin{define}[Generalized Plant] The plant $G$ is said to be a generalized plant if there exists at least one controller 
$K$ such that the interconnection of $G$ and $K$ is stable. Equivalently, $G$ is a generalized plant if $(A,B_u)$ is stabilizable
and $(A,C_y)$ is detectable. 
\end{define}


\subsection{Solving Analysis LMIs for Controller Matrices}
Once we have the generalized plant description we can design a controller that achieves closed loop stability and a performance level 
characterized by $P_p$. Before we state the nominal controller synthesis conditions, we provide a \enquote{trick}, the so-called 
Linearization Lemma which can be found in \cite{lmibook99}. It is also important to follow the rationale behind the arguments given
in \cite{scherermulti} for general quadratic performance synthesis. 

\begin{lem} Assume we are given with a quadratic form in which $A,S$ are constant matrices and $B(v),Q(v)$ and $R(v)\succeq 0$ are matrix functions 
of some decision variables denoted by $v$ such that
\begin{equation}
\pmatr{A\\B(v)}^T\pmatr{Q(v) &S\\S^T &R(v)}\pmatr{A\\B(v)} \prec 0
\label{eq:quadraticconst}
\end{equation}
should be verified. Also assume that there exists a decomposition such that $R(v) = T\inv{(U(v))}T^T$ where $U(v)\succ$ is affine in $v$ then 
\eqref{eq:quadraticconst} can be converted to the LMI problem
\begin{equation}
\pmatr{A^TQ(v)A + A^TSB(v) + B^T(v)S^TA &B^T(v)T\\T^TB(v) &-U(v)}\prec 0
\label{eq:linearizedconst}
\end{equation}
\end{lem}

\begin{proof} Applying Schur Complement formula with respect to the lower right block of the LMI and simply carrying out the block
multiplication in the quadratic inequality shows the equivalence. 
\end{proof}
Note that, even if $R(v)=R$ i.e. some constant matrix, this step is needed for resolving the quadratic dependence of $B(v)$. 

Following our analysis results from the previous chapter, we have seen that for a stable closed loop system $G\star K$ to have 
a performance level of $P_p$ the condition
\[
\pmatr{I&0\\\mathcal{A}&\mathcal{B}\\\hline 0&I\\\mathcal{C}&\mathcal{D}}^T
\left(
\begin{array}{cc|c}
	0&\mathcal{X}&0\\\mathcal{X}&0&0\\\hline 0&0&P_p
\end{array}
\right)
\pmatr{I&0\\\mathcal{A}&\mathcal{B}\\\hline 0&I\\\mathcal{C}&\mathcal{D}}\prec 0
\]
should hold for some symmetric matrix $\mathcal{X}$. However, in the absence of the knowledge of a stabilizing controller we encounter
an immediate problem. If we assume the controller matrices in calligraphic closed loop matrices as unknowns then they multiply the unknown
matrix $\mathcal{X}$ variables and hence destroying the affine dependence on the unknowns, rendering the constraint as a Bilinear Matrix 
Inequality (BMI). Also in the analysis case the stability was assumed at the outset however here the positivity constraint,
$\mathcal{X}\succ 0$ should be included to guarantee closed loop stability. Thus, the synthesis problem is more involved. The resolution of 
this problem appeared mainly in \cite{scherermulti,izumi} (also in a strict $\mathcal{H}_\infty$ context, \cite{gahapk,gahinet96}).

The essential trick is to manifacture a new set of derived variables from the original variables such that the problem is not altered but 
conditions become LMIs again. For this purpose, suppose we partition the matrices 
\[
\mathcal{X} = \pmatr{X &U\\U^T &\bullet}, \inv{\mathcal{X}} = \pmatr{Y &V\\V^T &\bullet}, 
\]
where $\bullet$ denotes the entries that we are not interested in. Also using the relations $\mathcal{X}\inv{\mathcal{X}}=I$, we have
\[
\mathcal{Y} = \pmatr{Y&I\\V^T&0}, \mathcal{Z} = \pmatr{I&0\\X&U}
\]
and $\mathcal{Y}^T\mathcal{X}= \mathcal{Z}$. Then by a congruence transformation the analysis LMI
\[
\pmatr{Y&0\\0&I}^T\pmatr{I&0\\\mathcal{A}&\mathcal{B}\\\hline 0&I\\\mathcal{C}&\mathcal{D}}^T
\left(
\begin{array}{cc|c}
	0&\mathcal{X}&0\\\mathcal{X}&0&0\\\hline 0&0&P_p
\end{array}
\right)
\pmatr{I&0\\\mathcal{A}&\mathcal{B}\\\hline 0&I\\\mathcal{C}&\mathcal{D}}\pmatr{Y&0\\0&I}\prec 0
\]
becomes
\begin{equation}
\pmatr{I&0\\\mathbf{A}&\mathbf{B}\\\hline 0&I\\\mathbf{C}&\mathbf{D}}^T
\left(
\begin{array}{cc|c}
	0&I&0\\I&0&0\\\hline 0&0&P_p
\end{array}
\right)
\pmatr{I&0\\\mathbf{A}&\mathbf{B}\\\hline 0&I\\\mathbf{C}&\mathbf{D}} \prec 0
\end{equation}
where boldface variables are defined as 
\[
\left(\begin{array}{c|c}
	\mathbf{A} &\mathbf{B}\\\hline\mathbf{C} &\mathbf{D}
\end{array}\right) \coloneqq \left(
\begin{array}{cc|c}
	AY+B_uM          &A+B_uNC_y     &B_w+B_uND_{wy}\\
	K                &AX+LC_y       &XB_w+LD_{wy}\\\hline
	C_zY+D_{uz}M     &C_z+D_{uz}NC_y &D_{zw}+D_{uz}ND_{wy}
\end{array}
\right)
\]
together with
\begin{equation}
\pmatr{K&L\\M&N}\coloneqq \pmatr{U&XB_u\\0&I}\pmatr{A_K  & B_K \\C_K  & D_K}
\pmatr{V^T&0\\C_yY&I}+\pmatr{XAY&0\\0&0}
\label{eq:contmatvartrafo}
\end{equation}
It's indeed not that easy to follow the inner workings of this transformation however after a tedious multiplication exercise
it can be seen that the aforementioned bijective transformation does the job. Thus, boldface variables are now functions of 
$X,Y,K,L,M,N$ matrices and moreover the constraint is once again an LMI. For stability characterization we also need 
$\mathcal{X}\succ 0$ and using the same transformation $\mathcal{Y}^T\mathcal{X}\mathcal{Y}\succ 0$, we obtain
\[
\mathbf{X} \coloneqq \pmatr{X&I\\I&Y}\succ 0.
\]
Therefore, instead of BMIs we have obtained LMI conditions all involving boldface variables entering affinely. This set of varibles 
are typically denoted by $v=\{X,Y,K,L,M,N\}$. Unfortunately, $K(s)$ and $K$ creates a naming clash however we will not resolve this 
to comply with the literature, instead we'll rely on the reader for the distinction and try to state whichever is meant in case of an
ambiguity. 

With all this preparation, we arrive at the nominal synthesis problem : 
\begin{thm}\label{thm:nomsynthLMI} The nominal synthesis problem is solvable if there exists a feasible set of 
variables $v$ such that
\begin{equation}
\mathbf{X}\succ 0,\pmatr{I&0\\\mathbf{A}&\mathbf{B}\\\hline 0&I\\\mathbf{C}&\mathbf{D}}^T
\left(
\begin{array}{cc|c}
	0&I&0\\I&0&0\\\hline 0&0&P_p
\end{array}
\right)
\pmatr{I&0\\\mathbf{A}&\mathbf{B}\\\hline 0&I\\\mathbf{C}&\mathbf{D}} \prec 0
\label{eq:nomsynthLMI}
\end{equation}
hold. Once a feasible solution is found, the controller can be obtained by first by finding arbitrary $U$ and $V$ matrices by 
solving $I-XY= UV^T$ and then back substituting the variables into \eqref{eq:contmatvartrafo}
\end{thm}

In summary, we have obtained a nominally stabilizing and $P_p$-performance level guaranteeing controller. Next, we include 
uncertainty channels to our plant $G$ and add a robustness constraint in our design. Unfortunately, this breaks down our
\enquote{LMIzation} step and there is no known method to obtain LMI solutions from the BMI constraints given below.

\subsection{Adding Uncertainty Channels}
We renew our plant representation by the following relations
\begin{equation}
\pmatr{\dot{x}\\ q\\ z\\ y} = 
\underbrace{\left(
\begin{array}{cccc}
	A    &B_p    &B_w    &B_u\\
	C_q  &D_{pq} &D_{wq} &D_{uq}\\
	C_z  &D_{pz} &D_{wz} &D_{uz}\\
	C_y  &D_{py} &D_{wy} &0
\end{array}
\right)}_{G}
\pmatr{x\\ p\\ w\\ u}
\label{eq:genplant}
\end{equation}
and $p=\Delta q$. The positive integers $n_p,m_q$ are the uncertainty operator row/column numbers respectively. Also this extra
channel is often called the \enquote{uncertainty channel}. Once again we start off with the robustness/performance analysis LMI. 
For that we also need to connect our stabilizing controller. The new closed loop matrices are obtained as
\begin{equation}
\begin{multlined}[][0.9\textwidth]
\left[\begin{array}{c|cc}
\mathcal{A}	&\mathcal{B}_p        &\mathcal{B}_w \\\hline
\mathcal{C}_q	&\mathcal{D}_{pq}   &\mathcal{D}_{wq}\\
\mathcal{C}_q	&\mathcal{D}_{pz}   &\mathcal{D}_{wz}
\end{array}
\right] = 
%\left[\begin{array}{cc|cc}
%A + B_uD_KC_y        & B_uC_K    & B_p + B_uD_KD_{py}        & B_w + B_uD_KD_{wy} \\
%B_KC_y               & A_K       & B_KD_{py}                 & B_KD_wy\\ \hline
%C_q + D_{uq}D_KC_y   & D_{uq}C_K & D_{pq} + D_{uq}D_KD_{py}  & D_{wq} + D_{uq}D_KD_{wy}\\
%C_z + D_{uz}D_KC_y   & D_{uz}C_K & D_{pz} + D_{uz}D_KD_{pz}  & D_{wz} + D_{uz}D_KD_{wy}
%\end{array}\right] \\
\left[
\begin{array}{cc|cc}
A   & 0  & B_p    & B_w    \\
0   & 0  & 0      & 0      \\ \hline
C_q & 0  & D_{pq} & D_{wq} \\
C_z & 0  & D_{pz} & D_{wz}
\end{array}
\right] +\\ 
\left[
\begin{array}{cc}
0  & B_u \\
I  & 0\\ \hline
0  & D_{uq}\\
0  & D_{uz}
\end{array}
\right]
\left[
\begin{array}{cc}
A_K  & B_K \\
C_K  & D_K
\end{array}
\right]
\left[
\begin{array}{cc|cc}
0    & I  & 0      & 0        \\
C_y  & 0  & D_{py} & D_{wy}
\end{array}
\right].
\end{multlined}
\label{eq:uncclinc}
\end{equation}

Suppose a block diagonal collection of uncertainty operators, $\bm{\Delta}$ characterized by a family of constant multipliers $\mathbf{P}$. Moreover,
assume that the $\Delta\star(G\star K)$ is well-posed for all $\Delta\in\bm{\Delta}$.
For notational convenience we also introduce the partitions 
\[
P=\begin{pmatrix}Q&S\\ S^T&R\end{pmatrix},\quad P_p=\begin{pmatrix}Q_p&S_p\\ S^T_p&R_p\end{pmatrix}
\]
Again, as given in \Cref{chap:analysis}, the closed loop system is robustly stable for all $\Delta\in\bm{\Delta}$ and achieves performance characterized 
by the multiplier $P_p$ if there exist a $P\in\mathbf{P}$ and a symmetric matrix $\mathcal{X}$ such that 
\begin{equation}
\pmatr{
I&0&0\\\mathcal{A}&\mathcal{B}_p&\mathcal{B}_w\\
0&I&0\\\mathcal{C}_q&\mathcal{D}_{pq}&\mathcal{D}_{wq}\\
0&0&I\\\mathcal{C}_z&\mathcal{D}_{pz}&\mathcal{D}_{wz}
}^T
\left(
\begin{array}{cc|cc|cc}
	0&\mathcal{X}&&&&\\
	\mathcal{X}&0&&&&\\\hline
	&&Q&S&&\\
	&&S^T&R&&\\\hline
	&&&&Q_p&S_p\\
	&&&&S_p^T&R_p\\
\end{array}
\right)
\pmatr{
I&0&0\\\mathcal{A}&\mathcal{B}_p&\mathcal{B}_w\\
0&I&0\\\mathcal{C}_q&\mathcal{D}_{pq}&\mathcal{D}_{wq}\\
0&0&I\\\mathcal{C}_z&\mathcal{D}_{pz}&\mathcal{D}_{wz}
}\prec 0
\label{eq:analysisLMI}
\end{equation}

%
%\begin{thm} 
%
%\end{thm}

For the controller synthesis case, unfortunately previous transformation does not resolve the additional bilinear terms and in fact it is not even known 
if such a transformation is possible or not. We can either try to solve BMIs directly with nonconvex optimization techniques or we can use another nonconvex 
solution which is known as the multiplier-controller iteration. 

Notice that if we have the stabilizing controller then the outer factors are constant matrices and we have an LMI problem. Conversely, if we have a 
feasible multiplier such that the inequality is satisfied then it's a matter of applying the aforementioned transformation and linearizing lemma to obtain
a controller. Hence, we can perform an iteration by either fixing the multiplier or the controller. 

\subsection{The Multiplier-Controller Iteration for Frequency-Independent Multipliers}

If we start with a uncertain generalized plant representation, we have neither the controller nor the robustness multipliers. Moreover, we don't have a 
method to search for both simultaneously. Thus, we first consider the nominal control design problem and obtain a nominally stabilizing controller as 
given above. It can also be shown that, the closed loop, with the well-posedness assumption and a simple continuity argument, has some, possibly very 
limited, robustness properties against the uncertainty set we would like to consider. In other words, there is no reason for the controller to be robustly 
stabilizing against the full uncertainty region that we originally modeled since we did not enforce it by any constraint. 
%Therefore, we search for the maximum size of a scaled version of original uncertainty set.

\begin{rem}
This is briefly the rationale behind the common assumption of star-shaped uncertainty region, i.e., $[0,1]\bm{\Delta}\in\bm{\Delta}$. However, it is 
important to note that, the scaling needs not to be a simple scalar $r\in[0,1]$ such that the uncertainty is scaled with $r\Delta$. This point is often
implicitly assumed and rarely mentioned. However the notation $r\Delta$ should be taken conceptually. As an example, 
the delay uncertainty cannot be scaled with $re^{-s\tau}$ for any $r\in[0,1]$ since it would scale the unit-circle. In fact what we want to scale 
is the $\tau$ variable as $e^{-sr\tau}$ such that when $r=0$ we have $e^0=1$ i.e. a non-delayed line and for $r=1$ we have $e^{-s\tau}$ i.e. full 
duration of the maximum allowed delay. Similarly, for the saturation and dead-zone nonlinearities are examples of such nontrivial scaling cases. Hence, 
the star-shapedness in this context becomes a parametrization of the size of the uncertainty from the nominal case to the full-sized uncertainty. 
Nevertheless most uncertainty types are amenable to scaling with $r$-multiplication hence there is no need to invent yet another
notation for an already complicated procedure. Therefore, every uncertainty needs to be scaled in a customized way but for notational convenience we will 
still use $r\bm{\Delta}$ to denote the scaled uncertainty size. 
\end{rem}





Summarizing our current situation, we have obtained a nominally stabilizing controller and we have parameterized a custom scaling method for each of our 
uncertainty subblocks. Now we would like to search for the maximum achievable $r$ via analysis. Hence, we can search over the maximum $r$ such that the 
scaled closed loop is robustly stable for all $\Delta\in r\bm{\Delta}$. 



Formally we have the following two theorems that constitutes the initialization and also the two steps of the iteration: We assume that 
the uncertain LTI plant 
\[
G:\mathcal{L}_{2e}^{n_p+n_w+n_u}\to\mathcal{L}_{2e}^{m_q+m_z+m_y},
\] 
is a generalized plant i.e. there exists an LTI controller 
\[
K:\mathcal{L}_{2e}^{m_y}\to\mathcal{L}_{2e}^{n_u},
\]
such that the nominal closed loop plant $G_{nom}\star K\in\mathcal{RH}_\infty^{m_z\times n_w}$ where 
\[
G_{nom}(s) = \pmatr{0_{(m_z+m_y)\times m_q} &I_{m_z+m_y}}G(s)\pmatr{0_{n_p\times(n_w+n_u)}\\I_{n_w+n_u}}. 
\]
Additionally $\mathbf{P}$ denotes the suitable multiplier that all $\Delta\in\bm{\Delta}$ satisfies the 
quadratic constraint.

\begin{figure}%
\centering
\begin{tikzpicture}
\node[draw,minimum size=7mm] (d) {$\Delta$};
\node[draw,below = 1cm of d,minimum size=9mm] (g) {$G$};
\node[draw,below = 9mm of g,minimum size=7mm] (k) {$K$};
\draw[->] (g.150) -| ++(-8mm,8mm)  |- (d);
\draw[->] (d) -| ++(1.2cm,-8mm) |- (g.30);
\draw[<-] (g) --++(1.5cm,0) node[right] {$w$};
\draw[->] (g) --++(-1.5cm,0) node[left] {$z$};
\draw[->] (g.-150) -| ++(-8mm,-8mm)  |- (k);
\draw[->] (k) -| ++(1.2cm,8mm) |- (g.-30);
\end{tikzpicture}
\caption{The uncertain interconnection}%
\label{fig:uncicsynth}%
\end{figure}

\begin{thm}[Analysis Step] The interconnection of an LTI uncertain plant $G(s)$ given by \eqref{eq:uncclinc}
with a nominally stabilizing controller $K$ given by \eqref{eq:contmats}, depicted in \Cref{fig:uncicsynth} 
and admits the realization given in \eqref{eq:uncclinc} is robustly stable in the face of all 
$\Delta\in r\bm{\Delta}$ and achieves the performance level characterized by $P_p$ if there exist a symmetric matrix 
$\mathcal{X}$ and $P\in\mathbf{P}$ such that \eqref{eq:analysisLMI} hold.
\end{thm}

One can simply perform a line search for the largest possible $r\in[0,1]$ such that the conditions are numerically
verified and $P_p$ is optimized. Then the resulting multiplier $P$ is fixed and we switch to the controller design step.

\begin{thm}[Synthesis Step] Assume an uncertain LTI plant $G(s)$ given by \eqref{eq:uncclinc}
is given. There exist an LTI controller $K$ such that the closed loop is stable for all $\Delta\in r\bm{\Delta}$ and achieves 
the performance level characterized by $P_p$ if there exist a set of variables $v=\{X,Y,K,L,M,N\}$ such that the linearized 
version of the constraint (omitted for brevity)
\begin{equation}
\pmatr{
I&0&0\\\mathbf{A}&\mathbf{B}_p&\mathbf{B}_w\\\hline
0&I&0\\\mathbf{C}_q&\mathbf{D}_{pq}&\mathbf{D}_{wq}\\\hline
0&0&I\\\mathbf{C}_z&\mathbf{D}_{pz}&\mathbf{D}_{wz}
}^T
\left(
\begin{array}{cc|cc|cc}
	0&I&&&&\\
	I&0&&&&\\\hline
	&&Q&S&&\\
	&&S^T&R&&\\\hline
	&&&&Q_p&S_p\\
	&&&&S_p^T&R_p\\
\end{array}
\right)
\pmatr{
I&0&0\\\mathbf{A}&\mathbf{B}_p&\mathbf{B}_w\\\hline
0&I&0\\\mathbf{C}_q&\mathbf{D}_{pq}&\mathbf{D}_{wq}\\\hline
0&0&I\\\mathbf{C}_z&\mathbf{D}_{pz}&\mathbf{D}_{wz}
}\prec 0
\label{eq:thmsynthLMI}
\end{equation}
and $\mathbf{X}\succ 0$ hold.
\end{thm}
Again, in this step we optimize over $P_p$ while performing a line search over $r$. Theoretically, since the analysis result
is feasible for some $r_a$, the controller step should at least give a feasible result for $r=r_a$ and possibly larger values
should also return feasible results. However, numerically it's not the case. One might step down a little to actually obtain 
feasible results. In our cases, we allowed the maximum retreat value to be the $0.99r$ of the previous step. We have also 
observed that this might actually improve the conditioning of the LMI solution though by no means guaranteed. 

\begin{rem}

Note that same theorem can be formulated for all $\Delta\in\bm{\Delta}$ and a closed loop plant $\left((G\star K) (s) \right)(r)$
in other words we can also modify the plant information to scale the uncertainty by subsuming the $r$ parameter suitably into the 
plant (for a simple case see \Cref{fig:uncscale}).
To demonstrate the numerical problem, we assume that the uncertainties are of unstructured LTI type and for simplicity assume constant multipliers. 
For parametric uncertainties, it suffices to scale down the respective uncertainty channels for the scaling as shown in \Cref{fig:uncscale}. 

\[
\pmatr{r\Delta\\I}^T\pmatr{Q&S\\S^T&R}\pmatr{r\Delta\\I} = \pmatr{\Delta\\I}^T\pmatr{r^2 Q&rS\\rS^T&R}\pmatr{\Delta\\I}\succeq 0.
\]
Out of our test experience, we have found out that reflecting the scaling to the plant is better for numerical stability. Seemingly the reason for 
this is the numerical noise introduced when $r$ is small in the early iteration steps. Notice how the square of $r$ drives the $Q$ block to zero
if $0<r\ll 1$. In case of a feasible multiplier is found, this multiplier should be stripped off from $r$ since it will be again used in the 
controller step, but due to the numerical inaccuracies wild changes are possible and usually the resulting multiplier information is contaminated. 
We have found out\footnote{Initially suggested by C\'esar Lop\'ez during the experiments.} that we have more freedom in the scaled plant case since
one can decrease the bad effect of small numbers via balanced realizations, cleaning up the state-space matrices etc. Moreover as we show later,
in the frequency dependent multiplier case, $r$ variable becomes garbled in the factorizations, minimal realizations etc. Hence, in this work, it 
is recommended to scale the plant instead of the multipliers.
\end{rem}

\begin{figure}%
\centering%
\begin{tikzpicture}
\node[draw,minimum size=7mm] (d) {$\Delta$};
\node[draw,below = 1cm of d,minimum size=7mm] (g) {$G$};
\draw[->] (g.150) -| ++(-8mm,8mm) node[draw,fill=white] {$r$} |- (d);
\draw[->] (d) -| ++(1.2cm,-8mm) |- (g.30);
\draw[<-] (g.-30) --++(1cm,0) node[right] {$w$};
\draw[->] (g.-150) --++(-1cm,0) node[left] {$z$};
\end{tikzpicture}
\caption{Reflecting the uncertainty scaling to the plant}%
\label{fig:uncscale}%
\end{figure}


Hence, we have a theoretically increasing sequence of analysis and synthesis uncertainty sizes 
\[
r_{si} = 0< r_{a1} \leq r_{s1} \leq \ldots \leq r_{an} \simeq r_{sn}
\]
Similarly, the performance objective gets worse or at least stays constant at after each succesful iteration
revealing an increasing sequence of real scalar such as the robust $\mathcal{L}_2$ gain or a similar 
functional. 
The iteration terminates when the last approximate equality is satisfied with desired accuracy or both limits 
are close enough to $1$ together with agreeing performance level $P_p$. The aforementioned numerical difficulty 
might exhibit a phenomenon such that both $r_{a}$ and $r_{s}$ come very close to $1$ but don't quite reach to $1$.
We have coded an extra condition that if $r$ values come close to $1$ within the prescribed accuracy, the code 
simply assumes $r=1$, such that the oscillations are removed and the $r=1$ is tested at each step.  However, it's 
up to the designer's scrutiny whether such result is acceptable or not. 




















\section{Human as Uncertain Filter}

As we have touched upon in \Cref{chap:litsurvey}, one can see the human action as a force input filtered through the
arm dynamics. Before we start deriving the relevant models, a few remarks seems proper to state here. 

As we have argued up to this point, passivity assumption on the human dynamics seems artificial, conservative and also
the evidence that is brought up for justification for it is questionable. Therefore, we would like to change the human
model or the source termination of the port with a refined impedance model, however, we have no substitute for such change. 

This might seem awkward to state, especially after putting so much emphasis on this issue. 










%
%But, we have also scanned the neuroscience articles and seen that there is no consensus on how the human arm dynamics are altered. 
%There are many studies with extraordinarily bold titles (which might be standard of publishing in the most prestigious journals),
%but, in the end they still seem disagreeing with each other if not contradicting thus the results are unlike the titles are suggestive 
%but not conclusive. \footnote{As an orthogonal comment to the beginners of the field, the scientific standard for claiming to have 
%a solution is much different in the neuroscience than the engineering customs. The main problem is that most of these studies first 
%define a hypothesis and then look for patterns to verify the behavior. However, another similar article uses almost the same 
%experimental setup and reaches to a completely different set of conclusions. Many seem to suffer from the causality often described 
%with \emph{All apples are red, this is red, therefore it is an apple}.}.







%\section{Model Setup}
%We present two possibilities of human modeling. In the first one, the uncertain human arm is taken from literature and 
%in the second we just present the complexity of a possible (Quasi)-LPV model. 
%\subsection{Uncertain LTI Model}
%The synthesis problem that we want attack requires a human model together with the local/remote 
%devices. We will delay the environment discussion until the next section. Our framework can be 
%shown pictorially in \Cref{fig:genframe}. Here we assume that the human is watching a screen that
%shows the remote environment or building up a virtual state of the environment by just interpreting
%the force cues that s/he receives. For simplicity , we concentrate on the former case.
%
%The most important difference with the current literature is that we skip the modeling the environment
%and include its effect with an external perturbation denoted by $f_e$. As an example, a hard contact 
%is modeled as a force perturbation at some point in time, restricting the position of the remote 
%device. However, there are important relations that is being ignored such as the frequency spectrum
%of the force signal $f_e$ is directly related to the position reference supplied by human $x_d$, and 
%the resulting force input. As an advantage, this allows us to include active environments such as 
%underwater current affecting the remote device in a deep-sea exploration mission etc. hence the environment 
%need not to be passive at all. Instead, backdrivability is explicitly considered. The various mesurements
%can be defined depending on what sort of a controller type is required. A typical selection is the position
%and the force signals of remote and local site. 
%
%
%\subsubsection{Assumptions}
%\begin{itemize}
	%\item We assume that the frequency content of the human reference generated by $f(vision)$ has no significant content
	%at high frequencies. For this, we can introduce yet another low-pass filter or modify the existing $W_t$. 
	%\item Similarly we assume that the frequency content of $f_e$ is dominant in the low-frequency band. 
	%\item For now, we don't place transmission delays unilaterally or bilaterally around $K$ block. This would denote the
	%local or remote device distance from the control station. It can be at the local or the remote site or even in between
	%distant to both stations.
%\end{itemize}
%\begin{figure}%
%\centering
%\begin{tikzpicture}[>=stealth]
%\node [draw,circle,inner sep=2pt] (junc) at (-3cm,0) {};
%\node [draw] at (-4.2cm,0cm) (H) {$H_u(\Delta)$};
%\node [draw] (K) at (-1.5,-1) {K};
%\node [draw] (Hp) at (-1.5,0) {$H_p$};
%\node [draw,outer sep=0] (mr1) at (1,0) {$M_{r1}$};
%\node [draw] (mr2) at (3,0) {$M_{r2}$};
%\node [draw] (brain) at (-4,2) {$f(vision)$};
%\draw[<-] (junc.south) node[below right] {$-$}|- (K.west) node[below,midway] {$\tau_l$};
%\draw[<-] (H.west) -- ++(-0.5,0) node[above left] {$x_d$} -| ++(-0.5,2) -- (brain.west);
%\draw[->] (H.east) -- (junc.west);
%\draw[->] (junc.east) -- (Hp.west);
%\fill [pattern = north east lines] (mr1.south west) rectangle ([yshift=-2mm]mr2.south east);
%\draw[decorate,decoration={zigzag,pre length=0.2cm,post length=0.2cm,segment length=6}] (mr1.20) -- (mr2.160);
%\draw[decoration={markings,  
  %mark connection node=dmp,
  %mark=at position 0.5 with 
  %{
    %\node (dmp) [inner sep=0pt,transform shape,rotate=-90,minimum width=5pt,minimum height=1.5pt,draw=none] {};
    %\draw ($(dmp.north east)+(2pt,0)$) -- (dmp.south east) -- (dmp.south west) -- ($(dmp.north west)+(2pt,0)$);
    %\draw ($(dmp.north)+(0,-2.5pt)$) -- ($(dmp.north)+(0,2.5pt)$);
  %}
%}, decorate] (mr1.-15) -- (mr2.-165);
%\draw[->] (K.east) -| ([xshift=-5mm]mr1.west) node[midway,below] {$\tau_r$}-- (mr1.west);
%\draw[<-] (mr2.east) -- ++(5mm,0) node[right] {$f_e$};
%\draw[->] (Hp.east) -- ++(5mm,0) node[right] (xl){$x_l$};
%\draw[dashed,->] (mr2.north) |- ([yshift=1mm]brain.east);
%\draw[dashed,->] (xl.north) |- ([yshift=-1mm]brain.east);
%\draw[<-] (K.south) -- ++(0,-5mm) node[text width=3.5cm,below,align=center] {Various \\ Measurements};
%\end{tikzpicture}
%\caption{The general framework}%
%\label{fig:genframe}%
%\end{figure}
%
%
%
%The inclusion of a human model as such allows us to model the hand tremor of a human with varying abilities. This is certainly 
%important in rehabilitation problems where the individual has difficulty in maintaining a firm hand/arm position. In other words, 
%for the same position reference input, different $H_{arm}$ models can lead to the hand tremor via some underdamped vibrational 
%modes. In our particular example, we refer to an identified model found in the literature. 
%
%
%We start with the identified model \cite{fucavus} : The basic block diagram is given in \Cref{fig:identifiedmodel}. The signal relations can be written as
%\[
%\pmatr{q\\x} = \bmatr{0 &W_{unc}\\H_{arm} & H_{arm}}\pmatr{p\\f}.
%\]
%The numerical data is given as (\cite{fucavus})
%\[
%H_{arm} =     \frac{0.1092 s^2 + 7.035 s + 329.5}{s^3 + 92.77 s^2 + 1921 s + 1.788e004}
%\]
%and 
%\[
%W_{unc}= 10^{th} \ \ order \ \ proper TF\text{ (large text to-be-replaced-later)}.%1.2252 \frac{(s^2 + 3.718s + 10.91) (s^2 + 71.58s + 1324) (s^2 + 16.27s + 242.3) (s^2 + 77.18s + 1948) (s^2 + 5.724s + 2285)}{%
%%(s^2 + 3.07s + 5.277) (s^2 + 138.3s + 4795) (s^2 + 10.95s + 178.5) (s^2 + 30.16s + 1434) (s^2 + 8.164s + 2356)}
%\]
%
%
%\begin{figure}%
%\centering
%\begin{tikzpicture}[>=stealth]
%\node at (-3.5cm,0) (f) {$f$};
%\node [draw,circle,inner sep=2pt] (junc) at (0.2cm,0) {};
%\node [draw] at (-2cm,0.8cm) (W) {$W_{unc}$};
%\node [draw] at (-0.3cm,0.8cm) (delt) {$\Delta$};
%\node [draw] at (1.2cm,0cm) (H) {$H_{arm}$};
%\draw[->] (f) -- (junc.west);
%\draw[->] (f) -- ++ (5mm,0) |- (W.west);
%\draw[->] (W.east) -- (delt.west) node[midway,above] {$q$};
%\draw[->] (delt.east) -| (junc.north) node[midway,above right] {$p$};
%\draw[->] (junc.east) -- (H.west);
%\draw[->] (H.east) -- ++(5mm,0) node[right] {$x$};
%\end{tikzpicture}
%\caption{The experimentally identified human model from \cite{fucavus}}%
%\label{fig:identifiedmodel}%
%\end{figure}
%
%For our purposes we obtain a mapping from position to force via a partial inversion on the second channel. 
%\[
%\pmatr{q\\f} = \bmatr{-W_{unc} &W_{unc}\inv{H_{arm}}\\-1 &\inv{H_{arm}}}\pmatr{p\\x} = \bmatr{W_{unc}\\1}\bmatr{-1 &\inv{H_{arm}}}\pmatr{p\\x}
%\]
%
%
%\begin{figure}%
%\centering
%
%\begin{tikzpicture}[>=stealth]
%\node at (0.6cm,0) (f) {$f$};
%\node [draw,circle,inner sep=2pt] (junc) at (-3cm,0) {};
%\node [draw] at (-0.8cm,0.8cm) (W) {$W_{unc}$};
%\node [draw] at (-2.3cm,0.8cm) (delt) {$\Delta$};
%\node [draw] at (-4.2cm,0cm) (H) {$H_{arm}^{-1}$};
%\draw[<-] (H.west) -- ++(-0.5,0) node[left] {$x$};
%\draw[->] (H.east) -- (junc.west);
%\draw[->] (delt.west) -| node[midway,above left] {$p$} (junc.north) node[above right,inner sep=1pt] {$-$};
%\draw[->] (W.west) -- (delt.east) node[midway,above] {$q$};
%\draw[->] (junc.east) -- (f);
%\draw[->] (f) -- ++(-0.6cm,0) |- (W.east);
%\end{tikzpicture}
%\caption{The mapping is inverted to obtain a model from position to force.}%
%\label{fig:invidentifiedmodel}%
%\end{figure}
%
%
%The LTI transfer function $H_{arm}$ is strictly proper hence its inverse is non-proper. Since we are only interested in the frequency response up to \SI{200}{\hertz} we use a low-pass filter on the position to tame the high frequency behavior. 
%\[
%W_{t} = \frac{4000}{s^2 + 280 s + 40000}
%\]
%
%After the application of this filter we obtain the overall human operator as
%\[
%\pmatr{q\\f} = \bmatr{W_{unc}\\1}\bmatr{-1 &\inv{H_{arm}}W_t}\pmatr{p\\x} = H_u \pmatr{p\\x}
%\]
%The next step is the obtain a local device model from force to position. We have selected the commercial PHANTOM device model which is identified in \cite{cavusfeygintendick}. The model is given by the following transfer function: 
%\[
%H_p = \frac{1}{s^2}\frac{s^2 + 5.716 s + 9.201\cdot 10^{4}}{(3.329\cdot 10^{-6} s^2 + 0.001226 s+1.536)}.
%\]
%After perturbing the integrators, we can incorporate this transfer function into our synthesis setup. 
%
%
%\subsection{An LPV Human Model}
%
%In the same manuscript \cite{fucavus}, there is also a grip force dependent model, given in Table~\ref{tab:fucav}. However it seems 
%that, for our tools the LPV models to-be-obtained from this parametric dependence is too cumbersome in terms of the tractability 
%of the synthesis conditions. The structure of the model is given by
%
%\[
%H = \frac{M_a s^2 +(b_1 +b_2 )s+k_1 +k_2}{b_1 M_a s^3 +(b_1 b_2 +k_1 M_a )s^2 +(b_2 k_1 +b_1 k_2 )s+k_1 k_2}
%\]
%
%Hence, we refrain from attacking the LPV problem with this kind of model structure. Had we had a convenient 
%model scheduled over the grip force then it would be certainly beneficial to consider the output of $H_{arm}$
%as the scheduling parameter leading to a Quasi-LPV model. 
%
%
%\begin{table}%
%\centering
%\begin{tabular}{cccccc}
%X-axis           & Ma (kg)       &k1   (N/m)      &k2   (N/m)      &b1  (Ns/m)          &b2  (Ns/m)\\ \hline
%1N grip          &0.1925            &85.48            & 704.2             & 7.410              &  2.477\\
%2N grip          &0.2037            &76.29            & 785.4             & 7.598              &  2.532\\
%3N grip          &0.2057            &88.91            & 784.3             & 7.592              &  2.525\\
%%\\
%%Y-axis        &Ma (kg)      &k1   (N/m)      &k2   (N/m)      &b1  (N·s/m)      b2  (N·s/m)\\
%%1N grip          0.2775            91.48             649.4              7.217                4.314\\
%%2N grip          0.2984            84.85             779.8              7.632                4.919\\
%%3N grip          0.2954            86.84             775.1              7.719                4.541\\
%%\\
%%Z-axis        Ma (kg)      k1   (N/m)      k2   (N/m)      b1  (N·s/m)      b2  (N·s/m)\\
%%1N grip           4.374             4877              200.4              55.85                32.96
%%2N grip           2.250             2196              181.1              40.83                30.56
%%3N grip           3.107             3289              120.8              50.16                29.70
%
%\end{tabular}
%\caption{Arm model parameters from system identification}
%\label{tab:fucav}
%\end{table}
%
%\section{Discussion}
%
%We have deliberately left out the cognitive delay which models the reaction time of the human for a particular
%change in the environment. But still, it is definitely possible to include it in the model above. Another 
%possibility is to connect an uncertain model of the environment to the $f_e$ terminal if such information 
%is available. As we have shown in our analysis paper, it is certainly a matter of modifying loop equations
%after the inclusion of such models. 
%
%The reaon why we consider such a two-body model for the remote device is to show that the complexity of the 
%device models does not essentialy pose a computational problem, rather increases the problem size. 
%
%There are certainly many other choices but not all of them are control-oriented and hence quite nonlinear
%as far as the models are concerned. We are not at a position to judge whether simplifications of those 
%nonlinear models are valid or not, yet we are still in the search of valid approximations assuming that
%some of the nonlinearities are not essential for the human response characteristics.
%
%\section{Uncertain \texorpdfstring{$M$}{M}-\texorpdfstring{$D$}{D}-\texorpdfstring{$K$}{K} models for environment and the human}
%
%The equations of motion are given by 
%
%\begin{align}
	%m_1\ddot x_1 &= -d_1(\dot{x}_1-\dot{x}_2) - k_1(x_1-x_2) + l + f_h\\
	%m_2\ddot x_2 &= d_1(\dot{x}_1-\dot{x}_2) + k_1(x_1-x_2) - u_l\\	
	%m_h\ddot x_1 & = -k_hx_1 - d_h \dot{x}_1 - l
%\end{align}
%and for the remote side
%\begin{align}
	%m_3\ddot x_3 &= -d_2(\dot{x}_3-\dot{x}_4) - k_2(x_3-x_4) + r + f_e\\
	%m_4\ddot x_4 &= d_2(\dot{x}_3-\dot{x}_4) + k_2(x_3-x_4) - u_r\\	
	%m_e\ddot x_3 & = -k_ex_3 - d_e \dot{x}_3 - r
%\end{align}
%
%The uncertain parameters are $k_h,k_e,d_h,d_e$ and obtaining an uncertainty description is straightforward by introducing 
%the following signal relations: Let 
%
%\[
	%q_1 = x_1, q_2 = \dot{x}_1, q_3 = x_3, q_4 = \dot{x}_3
%\]
%and 
%\[
  %p_1 = k_h q_1,\quad p_2 = d_hq_2,\quad p_3 = k_e q_3,\quad p_4 = d_e q_4,
%\]
%then, rewriting the equations gives:
%\begin{align}
	%(m_h+m_1)\ddot x_1 &= -d_1(\dot{x}_1-\dot{x}_2) - k_1(x_1-x_2) -p_1 -p_2 + f_h\\
	%m_2\ddot x_2 &= d_1(\dot{x}_1-\dot{x}_2) + k_1(x_1-x_2) - u_l	
%\end{align}
%and 
%\begin{align}
	%(m_e+m_3)\ddot x_3 &= -d_2(\dot{x}_3-\dot{x}_4) - k_2(x_3-x_4) -p_3 - p_4+f_e\\
	%m_4\ddot x_4 &= d_2(\dot{x}_3-\dot{x}_4) + k_2(x_3-x_4) - u_r
%\end{align}
%where 
%
%\begin{equation}
%p=\pmatr{
%k_h\\&d_h\\&&k_e\\&&&d_e
%}q.
%\label{eq:synopenloopG}
%\end{equation}
%We select the position mismatch and the force mismatch as the performance channels i.e. 
%\[
%z_1 = x_1 - x_2,\ z_2 = f_e-u_l
%\]
%
%Thus, we obtain the state space representation using the following matrix equation
%\begin{multline*}
%\pmatr{\dot{x}_1\\\dot{x}_2\\\dot{x}_3\\\dot{x}_4 \\\ddot{x}_1\\\ddot{x}_2\\\ddot{x}_3\\\ddot{x}_4 } = 
%E^{-1}\left[\pmatr{
%0  &0  &0  &0  &1  &0  &0 &0\\
%0  &0  &0  &0  &0  &1  &0 &0\\
%0  &0  &0  &0  &0  &0  &1 &0\\
%0  &0  &0  &0  &0  &0  &0 &1\\
%-k_1 &k_1 &0 & 0&-d_1 &d_1 &0 &0\\
%k_1 &-k_1 &0 &0 &d_1 &-d_1 &0 &0\\
%0 &0 &-k_2 &k_2 &0 &0 &-d_2 &d_2\\
%0 &0 &k_2 &-k_2 &0 &0 &d_2 &-d_2
%}\pmatr{x_1\\x_2\\x_3\\x_4 \\\dot{x}_1\\\dot{x}_2\\\dot{x}_3\\\dot{x}_4 }+\right.\\
%\left.
%\pmatr{
%0  &0  &0  &0\\
%0  &0  &0  &0\\
%0  &0  &0  &0\\
%0  &0  &0  &0\\
%-1  &-1  &0  &0\\
%0  &0  &0  &0\\
%0  &0  &-1  &-1\\
%0  &0  &0  &0
%}\pmatr{p_1\\p_2\\p_3\\p_4}+
%\pmatr{
%0 &0\\
%0 &0\\
%0 &0\\
%0 &0\\
%1 &0\\
%0 &0\\
%0 &1\\
%0 &0\\
%}\pmatr{f_h\\f_e}+
%\pmatr{
%0 &0\\
%0 &0\\
%0 &0\\
%0 &0\\
%0 &0\\
%-1 &0\\
%0 &0\\
%0 &-1\\
%}\pmatr{u_l\\u_r}\right]
%\label{eq:OLsynsys}
%\end{multline*}
%
%via defining
%\[
%E = \pmatr{I_{4\times 4} \\ &m_1+m_h \\ &&m_2 \\ &&&m_3+m_e\\ &&&&m_4}.
%\]
%The uncertainty input channels are
%\[
%q = \pmatr{1 &0 &0 &0 &0 &0 &0 &0\\0 &0 &0 &0 &1 &0 &0 &0\\0 &0 &1 &0 &0 &0 &0 &0\\0 &0 &0 &0 &0 &0 &1 &0}x + 0_{4\times 8}\pmatr{p\\f\\u}.
%\]
%The performance channels are similarly given by
%\[
%\pmatr{z_1\\ z_2}=\pmatr{1 &0 &-1 &0 &\ldots &0 &0 &0 &0 \\ 0 &0 &0 &0 &\ldots  &0 &1 &-1 &0}\pmatr{x\\ p\\f\\u}
%\]
%which denotes the position and force mismatch respectively. The remaining step is to define the measurement channels which are the positions and the force measurements from local and remote sites. 
%\[
%y = \pmatr{ 1 &0 &0 &0 &\ldots &0 &0 &0 &0 &0\\ 0 &0 &1 &0 &\ldots &0 &0 &0 &0 &0\\ 0 &0 &0 &0 &\ldots &0 &1 &0 &0 &0\\ 0 &0 &0 &0 &\ldots &0 &0 &1 &0 &0}\pmatr{x\\ p\\f\\u}
%\]
%

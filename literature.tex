\chapter{A Brief Literature Survey}\label{chap:litsurvey}

The teleoperation systems are structurally interesting and equally challenging 
systems. This is especially true from a system theoretical point of view. As an example, 
if we just focus on the local and the remote devices that would be used for manipulation, 
we see that they are, whether linear or nonlinear, motion-control systems with well-studied 
properties. Hence, one can view the open-loop teleoperation system as a system with a block 
diagonal structure. However, unlike the typical motion-control systems, these two disjoint systems must be stabilized 
simultaneously by the same controller (delayed/undelayed local control loops can be seen as a structured central
controller) that is performing sufficiently in order to ``fool" the user such that the user feels 
a force feedback as if s/he is actually operating at the remote medium. Hence, outputs of each system become
exogenous inputs of the other and these are regulated by the to-be-designed controller. Therefore, it's this controller 
that makes a teleoperation system perform adequately or, as in many situations, drive to instability.


For example, in the case of free-air motion (i.e. the remote device is free to roam in the remote site), the human force input 
to the local device and/or the position of the local device should be tracked by the remote device. In the case of a 
hard-contact of remote device with the environment, however, these inputs should be counteracted  if the force vector points into 
the obstacle. Hence, the force signal is simultaneously tracked for mimicking the user motion and  
is defied in case of a resisting force at the remote site. If this is not enough, when the user suddenly decides to release the 
local device, this resistance should die out as soon as possible, preventing a kickback. As an example, when a user leans to a wall located at 
position $x_0\!\!\uparrow$, applying a horizontal force and then stops, it is not expected that the wall continues to 
push the user even after the user has the position $x<x_0$. There are a few other scenarios that would further complicate 
the requirements. In short, the user and the environment properties are time-varying and make it difficult to design a control 
law such that these and many other details are handled properly simultaneously. 


With this short motivation, we can safely claim that looking at the overall system as a motion control 
system is not sufficient in terms of complexity (though necessary). In general, motion tracking specifications 
are a subset of the general performance requirements of the bilateral teleoperation systems. 

%\begin{rem}
%A similar line of reasoning has been given in \cite{buergerhogan1}. But for reasons that are 
%not known to us,  the authors chose to see the classical loop shaping as the sole servo control method. Moreover, modern 
%control theory offers a few options to avoid the obstacles that are given as impossibilities in their paper. Therefore, 
%we emphasize that the difference here is merely about the problem formulation and is not about the 
%limits of performance\footnote{We refer to the elegant book \cite{boydbarratt} for a detailed investigation of the 
%limits of performance.}. 
%\end{rem}


The inception of the bilateral teleoperation technology is often attributed to the work 
of Raymond Goertz in Argonne National Laboratories, \cite{goertz}. (In \cite{basanezsuarez}, 
it's traced back to Nikola Tesla and in \cite{sheridan89}, even some 16th century tools are accepted as precursors 
of the modern teleoperation). The main motivation of Goertz' work (similarly later in Europe 
by Vertut \cite{vertutcoiffet}) was handling and manipulating nuclear material, thus the very 
first teleoperators were purely mechanical to cope with the hostile environment conditions. Though not much 
happened in terms of commercial product realizations, the concept of telemanipulation kept its appeal 
and a large body of research was reported until the 1980s. In that decade, with the help of the 
ever-increasing computational power and the popularity of Virtual Reality (VR), teleoperation 
technology received more attention for a possible use in the space-, underwater-, medical-related 
tasks. Together with the advances in control theory and network theory (e.g. \cite{miyazaki,furuta}), 
a more systematic control methodology is adopted. Especially, stability analysis results that can be 
related to design guidelines (physical parameter bounds, bandwidth limitations etc.) are utilized
and limits of performance were explored. A particular phenomenon, namely the destabilizing effect 
of the delays in the teleoperation, lead the experts of the field to delve more into the systematic 
analysis tools and qualitative aspects of teleoperation. Especially, the use of the concepts; passivity, 
scattering transformations, and wave variables have become the standard methods of analysis and synthesis
(see, e.g., \cite{hannaford89,andersonspong,nieslotine}). We will start to summarize the advances from this 
point as this thesis is precisely built on top these systematic analysis and synthesis results gathered in 
the last two decades. However, the reader is refered to \cite{hokayemspong,burdea}, 
and \cite{sheridan89} for a more detailed overview including other practical aspects of teleoperation analysis and 
the hardware developments with a more historical perspective which we will omit here. 


As we keep on narrowing down our focus to control theoretical parts of this challenging problem, we have to 
note that many parts of the bilateral teleoperation problem can be scrutinized under different frameworks. 
Hence, there is no shortage of different frameworks for which bilateral teleoperation problem is an ideal 
test case. Among the plethora of methods, for example, the variation of human and environment properties give 
naturally rise to a robust or an adaptive control approach, the hard-contact problem can be analyzed under 
switched control systems, jump control systems or constrained linear systems etc. Before we go into the 
details of the proposed methods of this thesis, let us sample a few important and successful approaches 
reported so far together with their shortcomings if any.


\begin{figure}%
%http://chat.stackexchange.com/transcript/message/4857161#4857161
\centering
\begin{tikzpicture}[scale=1,manstyle/.style={line width=4pt,line cap=round,line join=round}]
\node[rectangle,draw,minimum height=3cm,minimum width=3.8cm] at (0.5cm,0)     (a) {};
\node[rectangle,draw,minimum height=3cm,minimum width=3.8cm] at (1.5cm,1cm) (b) {};
\foreach \x in {north east,north west,south east,south west} \draw (a.\x) -- (b.\x);
\node[rectangle,draw,minimum height=3cm,minimum width=3.8cm] at (5.5cm,0)     (a) {};
\node[rectangle,draw,minimum height=3cm,minimum width=3.8cm] at (6.5cm,1cm) (b) {};
\foreach \x in {north east,north west,south east,south west} \draw (a.\x) -- (b.\x);
\node[fill,circle,inner sep=2.5pt,outer sep=1pt] at (-0.2mm,7.1mm) {};
\draw[manstyle] (0,0.5cm) -- ++(0,-1.2cm);
\draw[manstyle] (-1.5pt,0) -- ++(0,0.5cm) (1.2pt,1pt) --(0,5mm)--++(-45:4mm);
\draw[line width=1mm,fast cap-fast cap] (0.5cm,0.2cm) -- ++(0.5cm,0);
\begin{scope}[xshift=1.3cm,-stealth,black!20]
\draw (0,0,0) -- (0,0,1); \draw (0,0,0) -- (0,1,0);\draw (0,0,0) -- (1,0,0);
\end{scope}
\begin{scope}[xshift=4.7cm,-stealth,black!20]
\draw (0,0,0) -- (0,0,1); \draw (0,0,0) -- (0,1,0);\draw (0,0,0) -- (1,0,0);
\end{scope}
\begin{scope}[shift={(1.4cm,-0.2cm)}]
\draw (0,0) -- (0.9cm,0) (0.45,0) circle (1.5mm);
\path [postaction={pattern=north west lines},fill=white] (0cm,0cm) rectangle (0.9cm,-0.16cm);
\draw[line width=1.5mm,round cap-round cap] (0.475cm,0.05cm) -- ++(110:7mm);
\draw[line width=1mm,round cap-round cap] (0.475cm,0.05cm) ++(110:6.3mm) --++ (220:5mm);
\end{scope}
\begin{scope}[shift={(5.4cm,-0.2cm)},cm={-1,0,0,1,(0,0)},transform shape]
\draw (0,0) -- (0.9cm,0) (0.45cm,0) circle (1.5mm);
\path [postaction={pattern=north west lines},fill=white] (0cm,0cm) rectangle (0.9cm,-0.16cm);
\draw[line width=1.5mm,round cap-round cap] (0.475cm,0.05cm) -- ++(110:7mm);
\draw[line width=1mm,round cap-round cap] (0.475cm,0.05cm) ++(110:6.3mm) --++ (220:5mm);
\end{scope}
\draw[line width=1mm,fast cap-fast cap] (5.8cm,0.2cm) -- ++(0.5cm,0);
\begin{scope}[shift={(6.6cm,0.4cm)}]
\draw[thick]
\foreach \i in {1,2,...,10} {%
  [rotate=(\i-1)*36]  (0:2mm)  arc (0:12:2mm) -- (18:2.4mm)  arc (18:30:2.4mm) --  (36:2mm)
};
\node[circle,draw,inner sep=2pt,fill] (merkez) at (0,0) {};
\end{scope}
\begin{scope}[shift={(6.85cm,0cm)},scale=1.2]
\draw[thick]
\foreach \i in {1,2,...,10} {%
  [rotate=(\i-1)*36]  (0:2mm)  arc (0:12:2mm) -- (18:2.4mm)  arc (18:30:2.4mm) --  (36:2mm)
};
\node[circle,draw,inner sep=2pt,fill] (merkez) at (0,0) {};
\end{scope}
\begin{scope}[shift={(6.8cm,0.5cm)},scale=0.8]
\draw[fill] (0.5mm,0.5cm) rectangle (0.49cm,0.6cm);
\draw[fill,postaction={pattern=north west lines,pattern color=white}] (0.2cm,0.5cm) -- ++ (0,-0.4cm) -- ++ (-45:1mm) --++(45:1mm) --++ (0,0.4cm) --cycle;
\end{scope}

\end{tikzpicture}

\caption{General Teleoperation System}%
\label{fig:teleop}%
\end{figure}

\section{Modeling of Bilateral Teleoperation Systems}\label{sec:litchapmodeling}

The dominating modeling paradigm of bilateral teleoperation systems is the two-port network approach. Consider the quote taken from
\cite{hannaford89} published in 1989:
%\begin{mdframed}[style={userdefinedwidth={0.8\textwidth},align=center,outerlinewidth=3pt,innerlinewidth=0pt,outerlinecolor=black!10,roundcorner=3pt}]
\begin{quote}
%In this paper, development of the analytical framework is complemented by modeling of an actual teleoperator system.
The modeling approach is to transform the teleoperation system model into an electrical circuit and simulate it using
SPICE, the electronic circuit simulation program developed at UC Berkeley.
\end{quote}
%\end{mdframed}


As seen from Hannaford's motivation, the computer-based simulation tools are used extensively since then. Arguably
this is one of the main reasons why network based electrical circuit based modeling dominated the teleoperation literature. 
Reinforced with the circuit simulation tools, experts of the field started to construct analogies that go beyond
a mere mechanical-electrical system analogy. Arguably, the most prominent concept borrowed from these analogies is the two-port network 
view of the bilateral teleoperation systems. The reader is referred to \Cref{chap:apdxnetwork} for a short recap of network theory.
Today, the quoted convenience also applies to almost all physical systems i.e. one can simulate arbitrary models via many 
computational packages. Yet, it's a de facto standard to use the circuit modeling while the teleoperation devices are mostly 
mechanical. Hence, it's not clear whether the benefit of such an artificial step didn't vanish. Once the system is represented 
by a mathematical model, as it is demonstrated in the later sections, the mechanical/electrical analogy is, roughly, an equivalence
based on the resulting model and works in the electrical$\to$mechanical direction too. Therefore, to the best of our knowledge, the 
circuit based modeling is merely a convention rather than a requirement.

\subsection{Two-port Modeling of Teleoperation Systems}

In the teleoperation context, if one uses the ``load-source" analogy for the manipulated environment 
and the human, then the system models all the bilateral interaction between the load and the source 
ports (as in \Cref{fig:portrep}). This modeling view is quite powerful since the components are described via their input/output 
(or external) properties i.e. effort variable/flow variable relations (e.g. force/velocity, voltage/current etc.). Also, the non/linearity
properties of the components are not relevant at the outset if we are only interested in energy exchange which is the basis
of the so-called Time-Domain Passivity Methods \cite{hannafordryu} which we will mention later in this chapter. Thus, 
the user, the control system, the environment, the remote and local devices and communication delays are seen as 
$1-$ and $2-$ports exchanging energy in time. Since the external behavior of the ports can be characterized completely by the 
current and the voltage drop across the terminals, it is indeed very convenient to model these components as 
interacting ``black boxes'' (See \Cref{fig:portrep}).


\begin{figure}
\begin{subfigure}[b]{0.5\textwidth}
\centering
\begin{tikzpicture}[>=stealth,baseline=15mm,
every node/.style={draw,minimum size=1cm}]
\node (deltas) at (0,0) {$\Delta_l$};
\node[left=of deltas] (g) {$G$};
\node[left=of g] (deltal) {$\Delta_s$};
\tikzset{every node/.style={draw,circle,inner sep=1pt,fill=white}}
\foreach \y in {s,l}{
    \node at ({$(g)!0.5!(delta\y)$} |- {g.40}) (circ1\y){};
    \node at ({$(g)!0.5!(delta\y)$} |- {g.-40})(circ2\y){};
};
\draw (circ1l) -- (deltal.40) (circ2l) -- (deltal.-40)
(circ1s) -- (deltas.140) (circ2s) -- (deltas.-140);
\draw[->] (circ2s) -- (g.-40) (circ1s) -- (g.40);
\draw[->] (circ2l) -- (g.-140) (circ1l) -- (g.140);
\end{tikzpicture}
\caption{}
\label{fig:portrep}
\end{subfigure}%
\begin{subfigure}[b]{0.5\textwidth}
\centering
\begin{tikzpicture}[>=stealth,scale=0.5, transform shape]
\matrix (G) [draw,matrix of math nodes,inner sep=1mm,row sep=1mm,ampersand replacement=\&]{ G_{11} \& G_{12}\\ G_{21} \& G_{22}\\};
\matrix (delta) at (0,3) [outer sep=0,draw,matrix of math nodes,inner sep=0.5mm,ampersand replacement=\&]{\Delta_s \& \\ \& \Delta_l\\};
\draw[->] (G.160) -| ++ (-0.4,0.8) node[draw,circle,fill=white,inner sep=2,label={[inner sep=0]0:$\scriptstyle -$}] {} |- (delta.160);
\draw[->] (G.200) -| ++ (-0.7,0.6) node[draw,circle,fill=white,inner sep=2,label={[inner sep=0]0:$\scriptstyle -$}] {} |- (delta.200);
\draw[<-] (G.20) -| ++ (0.4,0.5) |- (delta.20);
\draw[<-] (G.-20) -| ++ (0.7,0.5) |- (delta.-20);
\end{tikzpicture}
\caption{}
\label{fig:nom_net}
\end{subfigure}
\caption{Two representations of a 2-port network.}
\end{figure}

Clearly, thanks to this modeling method, we don't even need to 
know exactly what $\Delta_l,\Delta_s$ blocks are, except their class (e.g. linear/nonlinear, time invariant/time varying etc.) to analyze 
the system via $G$.  Thus, the problem of modeling of the human arm or of the uncertain environment
is circumvented. However, due to the same reasoning, passivity property does not distinguish particular systems as long as they are 
passive. For this reason, some of the crucial information is lost about these specific ports. In other words, we discard
any impedance or admittance relations shared by the port variables.

We have mentioned the ports and energy transfer between them as the main modeling paradigm so far. Energy based modeling is 
the natural basis for bond-graphs too. Bond-graphs, much like port representations, are graphical tools to model the dynamical 
systems via energy balancing between subcomponents (See \cite{gawthrop} for an introduction). In other words, the bond-graphs are built on top of the notion 
of bonds representing the instant energy or power exchange between nodes via edges drawn between them. Therefore, bond-graphs
already presents a powerful framework for abstraction of the bilateral interaction between the local and the remote site. For a 
classical use of bond-graphs in impedance control, the reader is referred to the Hogan's trilogy (\cite{hogan:1,hogan:2,hogan:3}) and 
for a more specialized practical case, namely, using hydraulic systems for bilateral teleoperation, to \cite{krishnaswamy}. 



\subsection{Assumptions on the Local and Remote ``Ports''}

Up to this point, we have touched upon how network theory offered a great opportunity for modeling the 
teleoperation systems. Still, to invoke the stability analysis and synthesis results of the network there is a need to further distinguish
$\Delta_s,\Delta_l$ from the universum of $1-$ports. Otherwise there is not much we can conclude from such an 
interconnection, put differently, they can be any arbitrary model with arbitrary behavior set as long as they respect the port 
condition. This is obviously not sufficient to capture the real physical interaction that teleoperation systems embody. 

In teleoperation and haptics literature, it is customary to assume that the load and the source terminations as ``passive''
mathematical operators (see \Cref{chap:apdxnetwork}). By this hypothesis, the main tool that one has is the physical-interaction-based approach. 
Hence the view of the designer is tuned to see the energy interaction between two distant media. This approach treats the human and the 
environment as passive elements with additional effort sources modeling the intentional force input to the system and the 
controller is viewed as the energy regulator preventing excess energy causing the system go unstable. Moreover, as we have
briefly summarized in \Cref{chap:apdxnetwork} and in the analysis section below, one can use the network theory based conditions 
to assess the stability and performance conditions thanks to this hypothesis.

This brings us to the discussion of the justification of the validity of such an assumption as it is generally not given in full generality in the
literature. If one scans through the literature about the passivity of human operators, it is Hogan's paper \cite{hogan89}
that is almost universally cited. The striking detail is, however, that Hogan never claims that the human exhibits a passive
behavior. Instead he clearly shows that under certain conditions, human behavior is indistinguishable than that of a 
passive system: 
\begin{quote}
Thus, despite the fact that the
limb is actively controlled by neuro-muscular feedback, its apparent
stiffness is equivalent to that of a completely passive
system. In the light of Colgate's recent proof [3]\footnote{Reference 
\cite{colgatehogan88} of this thesis.} that an
apparently passive impedance is the necessary and sufficient
condition for a stable actively-controlled system to remain stable
on contact with an arbitrary passive environment, this
experimental result strongly suggests that neural feedback in the
human arm is carefully tuned to preserve stability under the
widest possible set of conditions.
\end{quote}
Moreover, the task that is given to human operators and analyzed afterwards is about making the human 
be as passive as possible. The task is, roughly speaking, holding a handle which is perturbed by random disturbances and 
trying to keep the handle still at a predefined position on the 2D plane. Hence, the task is simply to mimic a passive 
component that is, again roughly, a mass-spring-damper system. Had it been the case that the human would exhibit non-symmetric 
stiffness matrix, it would simply be a failure of the test subject (regardless of the physical limitations). Note that this 
is a plausible assumption for the rehabilitation tasks. The other possibility would then be is that the test subject was unable to keep up 
with the changes, or using the control theory jargon, the bandwidth of the subject was lower than the required agility to 
perform the test adequately.  The well-known phenomenon of such behavior is the ``pilot induced oscillations'' in which the pilot
of an aircraft, while trying to stabilize the aircraft, via overcorrection inputs, destabilizes the system due to many reasons (response time of the pilot 
and thus the phase lag with response time of the aircraft contribute among many other reasons). We refer to the interesting 
report \cite{mcruer} for a more detailed exposition. Moreover, if for some reason, the task at hand is to prevent the system 
to reach a steady state at a certain position and the perturbations are applied accordingly i.e., to create a virtual potential, 
the results obtained from the experiments would most probably different from that of \cite{mussa85}. Thus, it's worth mentioning 
that the passivity of the human is closely linked to the passivity of the environment. 


\begin{rem}
A particular detail should be clarified about the measurements taken in \cite{hogan89}. It is stated that: 
\begin{quote}
While normal human subjects held the handle of the manipulandum at a stable position
in the workspace, small perturbations were applied. Measurements
of the human's restoring force were made after the system
had returned to steady state following the perturbation but before
the onset of voluntary intervention by the subjects.
\end{quote}
Therefore, it is emphasized that only the involuntary response is taken into account during the measurements in order to
capture the natural properties human arm before the human correction intervenes. Note that, this does not imply that the
measurements treat the human arm as a bulky cybernetic device which might only be possible in case of an improbable amputation.
In fact, in this setting, the voluntary input of the human is not included to the analysis, since the human necessarily puts in
energy, at the very least, to move the local device. 
\end{rem}

The question of how, then, a human can actually move anything while remaining passive is one that makes the whole 
story more complicated. The voluntary input of the human is taken as an exogenous input to the system. This is due to the 
fact that, in passivity based analysis, the closed loop stability is tested against square integrable functions of time 
modeling the finite energy inputs by the human which will be covered in \Cref{sec:litchapanalysis} in this chapter.  

A keen eyed reader would spot that this is not inline with Hogan's remark since the human force input has 
alredy been considered in the apparent stiffness measurements. In other words, without the human's active control, the passivity 
argument does not hold as the stiffness matrix might be non-symmetric and hence would have a non-zero curl. Therefore, we have to further
separate the human force into two parts, namely, the active neuromuscular feedback force and the voluntary and cognitive force 
input applied to the system. How this is usually performed is not clear in the literature. To the best of our knowledge, this 
issue is addressed explicitly (but still briefly) only in \cite[Sec. II.B]{kazeroonitsay} and references therein.

This ambiguity becomes much more important since the control oriented focus of this thesis necessitates that we concentrate on
worst cases rather than the experiments performed within the cognitive range of human operators. In other words, we are interested in
the cases where things go wrong due to many other various reasons, sampling distubances, mesurement noises, directionality etc. Therefore
it cannot be a satisfactory argument if stability depends on the user's neuromuscular feedback or simply user's stabilization capabilities. 
In order to use the bilateral teleoperation devices in real-life cases, stability should be addressed regardless of human actions. Hogan's 
findings are not sufficient for supporting the passivity assumption often found in the literature. 


In summary, the passivity of the human and the environment (virtual environment in haptics/virtual reality applications), is 
only plausible in certain occasions which should be verified in order to assume that the corresponding mathematical models 
are passive. Nevertheless, analysis and synthesis methods that invoke this assumption lead to many real-world implementations 
with varying degree of realism. We argue that the success of these methods is due to the conservatism of the analysis/synthesis tools
and does not validate the passivity hypotheses on the respective models.

A compact version of the argument above is given by Yokokohji and Yoshikawa in \cite{yokokohjiyoshikawa}: 
\begin{quote}
Passivity of the system can be a
sufficient condition of stability only when the system interacts
passive environments. In the case of master-slave systems, if
we could assume that the operator and the environment are
passive systems, then the sufficient condition of stability is
that the master-slave system itself must be passive. Strictly
speaking, however, the operator is not passive because he/she
has muscles as the power source. Colgate et al. [21]\footnote{Reference 
\cite{colgatehogan88} of this thesis} mentioned
that even if the system has an active term, the system stability
is guaranteed unless the active term is in some way state dependent.
Obviously, the operator is passive when $\tau_{op}= 0$.
Therefore, we will give the following assumption about $\tau_{op}$:
``The operators input $\tau_{op}$ independent to the state of the
master-slave system. In other words, the operator does not
generate $\tau_{op}$ that will cause the system to be unstable.''
Dudragne et al. [3] gave a similar assumption in order to use
the concept of passivity for stability distinction. The above
assumption seems tricky in a sense, but it is necessary to ensure
the system stability by the passivity.
\end{quote}

\noindent Finally, a supplementary remark is also given by Buerger and Hogan in \cite{buergerhogan1}: 
\begin{quote}
When passivity is used as a stability objective, the only assumption
made about the environment is that it, too, is passive.
This is likely sufficient to guarantee coupled stability with humans
(though, to date, it has not been conclusively proven that
human limbs are passive; see [29]\footnote{Reference \cite{hogan89} in this thesis} for an argument for treating
them as such). However, given the properties of human arms
described above, passivity is unnecessarily restrictive. Our experience
has shown that some controllers that are known to be
nonpassive are adequately stable in clinical rehabilitation tasks [26].
\end{quote}


We should mention here that Hogan's paper together with other identification experiments are extremely important for many
fields and needs no motivation. However, the discussion above only points out that the inference that follows from his results, 
is not inline with the results themselves. 

The idea of modeling the teleoperation as a two-port network seems to have multiple origins and we have no reference to 
point out to a common source. However, in general the popularity of two-ports can be attributed to 
\cite{andersonspong,nieslotine,rajuphd,hannaford89,yokokohjiyoshikawa}. 

\subsection{Uncertain Models of Bilateral Teleoperation for Robustness Tests}

Another possibility of modeling the human arm and its cognitive input is to define an input force signal ``filtered'' by the 
human arm impedance\footnote{The term \emph{impedance} is used in a more general sense than its common usage to denote
LTI transfer functions such that no distinction is made between linear and nonlinear or time-invariant and time-varying 
operators.} 

Various studies pointed out that the identification experiments suggest a mass-spring-damper system pattern is evident
in the frequency response data of the human arm recorded under various task performance similar to the one given in \cite{hogan89}.
The general method is to instruct the human to perform a specific task and then perturb the hardware with certain predesigned
disturbance signals such that the output can be evaluated to obtain a mathematical model. In the literature, the model structure
is often set a priori to be a second order transfer function and the parameters are optimized to minimize the mismatch between
the experimental and predicted response. It is also well-known that the human can change the inherent impedance of the arm during
the task execution (see, e.g., \cite{tsujimorasso}). Therefore, the studies are performed in the ranges where it is safe to assume 
that the human arm characteristics are constant or constant up to negligible changes. 


The modeling is straightforward via uncertain mass-spring-damper system differential equation manupulations. Suppose the human arm
assumes the second order model: 
\[
M(\Delta_1)\ddot{x} + B(\Delta_2)\dot{x} + K(\Delta_3)x = F_h -F_m
\] 
where $F_h,F_m$ denote the human force and the force feedback inputs, respectively. Then choosing a multiplicative or additive uncertainty 
structure and via basic linear fractional transformations, the signal relations are converted to the interconnection shown in 
\Cref{fig:lit:figunc}. The arrows are deliberately left out as it's up to the designer to get different immitance models.
\begin{figure}%
\centering
\begin{tikzpicture}
\matrix[draw,matrix of math nodes] (m) {\Delta_1 &&\\&\Delta_2 &\\&&\Delta_3\\};
\node[below=1cm of m,draw,minimum size=2.0cm] (g) {$G_{nom}$};
\draw (g.155) -| ++(-1cm,1cm) |- (m.west);
\draw (g.25) -| ++(1cm,1cm) |- (m.east);
\draw (g.-155) -- ++(-1cm,0cm) (g.-25) -- ++ (1cm,0cm);
\end{tikzpicture}
\caption{Uncertain model representation by taking out the uncertainty blocks}%
\label{fig:lit:figunc}%
\end{figure}


Many studies have appeared in the literature regarding such modeling and the majority of these assume a 
mechanical model of order from two to five. Note that this is an assumption made a priori and only applies 
to the specific task performed by the human in the experiment from which the frequency response data is 
collected. The commonly utilized models can be found in \cite{kazeroonitsay,tsujigoto,kosuge,colgate1,speich,
fucavus,buergerhogan1,leungfa,husalculoewen,andrifour,laroche}. 
Obtaining these measurements are time-consuming and difficult to parameterize. For this reason, although
the results along this direction are scarce, they are, as in the passivity case, very valuable. 

The disadvantage of such parametrization of the human arm is contrasted with the passivity approach methods
invoking the argument of time-varying nature of the arm parameters. It is often rightfully argued that the 
uncertainty ranges in which the stiffness and damping (and partially inertial) coefficients change, are too 
large to be considered in the structured singular value based robustness tools. Moreover, many auxiliary 
effects such as the visual feedback, cognitive lag of the brain etc. are not considered in the identification 
experiments, but in the passivity approach all of these are lumped into a single port condition. 

The paper \cite{leelee} offers an interesting alternative as it tries to incorporate many of the aforementioned 
effects, however, the results are prospective and yet to be utilized. 



\section{Analysis}\label{sec:litchapanalysis}

The stability analysis is one of the major problems in designing stable yet high-performance teleoperation 
systems. It's often not feasible to manually tune some local controllers and make test subjects use it in order to verify the
design specifications. Moreover, relying only on the experiments can miss an important destabilizing scenario if the field 
experiments do not cover that particular case. Hence, an a priori certificate of stability is much sought after. The stability 
analysis can also give some guidelines about the parameter selection in the hardware design phase and can lead to minimized 
design iterations. Therefore, having a realistic stability test is essential in building these systems. 

Similar to the modeling section, the analysis in the literature extensively relies on network theory based results. In fact 
this is where the network theory stands out as a complete tool for analysis and synthesis of bilateral teleoperation systems
via the celebrated hypothesis that human and the environment models are passive.  

The common terminology for stability is somewhat different than that of the contemporary control theory as \emph{nominal stability} 
is used for the stability properties of isolated two disjoint media and when the interaction is setup between these two media the closed-loop 
stability problem is called \emph{coupled stability}. To the best of our knowledge, this terminology is introduced in 
\cite{colgatehogan88} hence we refer to this paper (or Colgate's thesis \cite{colgatephd}) for a more detailed motivation. 


It's also worth mentioning that the passivity and stability is used often interchangeably and also usually referred to 
\cite{haykin,mitra69,chen91}. Hence, there is a little guesswork required to classify the stability definitions given in the
literature. The important distinguishing point is that the marginal stability is accepted in the definition of stability results since
it arises frequently in lossless (hence passive) models where energy conservation is assumed. However, the analysis results often 
rely on such assumptions do not guarantee asymptotic interconnection stability but only lead to certain passivity 
properties of the interconnection (see \cite[Thm. 6.1]{khalil} and \cite[Sec. V]{lawrence} for the discussion on strict passivity).  


As given in \Cref{sec:litchapmodeling}, passivity property is crucial to many studies in the literature. The direct physical interpretation
of the abstract concepts gives even more appeal to such energy book-keeping methods. Another advantage of passivity methods is that 
the nonlinear counterparts of the results are also available in the literature. The major disadvantage of the passivity methods are that
the procedure is focused almost only on the energy exchange and performance specifications are very difficult to formulate and also 
difficult to integrate into the analysis and synthesis steps. As an example, the signals that are not port variables such as position errors,
and nonlinear effects that are functions of these signals, can't be utilized easily in the performance specifications. 


Another disadvantage is that the power or the energy based analysis, due to the inner-product structure, can not distinguish the individual 
signal properties. Consider the ideal case where the human and the local device is pushing each other and cancelling each other's contribution. 
In this case the energy exchange based on the port variables is zero (negligible) which can not be distinguished form the case of not touching at all
(a small motion on the device). 


\begin{figure}%
\centering
\begin{tikzpicture}[>=stealth]
\node[draw,minimum size=1cm] (h2) {$H_2$};
\node[draw,minimum size=1cm, above=1cm of h2] (h1) {$H_1$};
\node[draw,circle,inner sep=1mm,right= 1cm of h2]   (a1) {};
\node[draw,circle,inner sep=1mm,left= 1cm of h1,label={[inner sep=0]-60:$-$}] (a2) {};
\draw[->] (h1) -| (a1) node[right,midway] {$y_1$};
\draw[->] (h2) -| (a2) node[left,midway] {$y_2$};
\draw[->] (a1) -- (h2) node[above,midway] {$e_2$};
\draw[->] (a2) -- (h1) node[above,midway] {$e_1$};
\draw[<-] (a1) -- +(1cm,0) node[right] {$u_2$};
\draw[<-] (a2) -- +(-1cm,0) node[left] {$u_1$};
\end{tikzpicture}
\caption{Negative feedback interconnection}%
\label{fig:lit:passint}%
\end{figure}




The reader is referred to \Cref{chap:apdxnetwork} for the terminology used from hereafter. Let us recall a few key stability results: 
\begin{thm}[Passivity \cite{hillmoylan77}]
The interconnection shown in \Cref{fig:lit:passint} is passive if both $H_1$ and $H_2$ are passive. Moreover, the interconnection is 
asymptotically stable if any of the following conditions holds: 
\begin{enumerate}
	\item One of $H_1$ and $H_2$ is strictly passive. 
    \item Both $H_1$ and $H_2$  is strictly output or input passive. 
    \item The mapping $H_1\circ(-H_2)$ is zero-state detectable and, either $H_2$ is strictly input passive or $H_1$ is strictly output passive. 
    \item The mapping $H_2\circ H_1$ is zero-state detectable and, either $H_2$ is strictly output passive or $H_1$ is strictly input passive. 
\end{enumerate}
\end{thm}

\begin{coroll}[Passivity (LTI Case)]
Consider again the interconnection shown in \Cref{fig:lit:passint} where $H_1(s),H_2(s)\in\RH_\infty$. The interconnection is said to be 
asymptotically stable if $H_1$ ($H_2$) is positive real and $H_2$($H_1$) is strictly positive real.
\end{coroll}

When the network, human, and the environment models are assumed to be Linear Time Invariant (LTI), the frequency domain methods
allow us the analyze the teleoperation systems for stability and performance. The most common stabiliy analysis tool for such 
models is the Llewellyn's stability criteria (also often called absolute stability theorem or unconditional stability theorem). 
For linear networks, the following definitions seem to be used quite widely (modified from \cite{chen91}): 

\begin{define}[Potential Instability at $\iw_0$] A two-port network is said to be potentially unstable at $\iw_0$ on the real frequency, 
if there exist two passive one-port immitances that, when terminated at the ports, produce a natural frequency at $\iw_0$ overall network.   
\end{define}

\begin{define}[Absolute Stability at $\iw_0$] A two-port network is said to be absolutely stable at $\iw_0$ on the real frequency, 
if it is not potentially unstable at $\iw_0$.   
\end{define}


\subsection{Llewellyn Stability Criteria}\label{sec:llewellyn}
The well known conditions for stability of a two-port network, formulated in \cite{llewellyn,bolinder,rollett}, are recalled in the following theorem. An explicit indication of the frequency dependence is  {often} omitted for notational convenience.
\begin{thm}[Llewellyn's Criteria]\label{thm:llw}
A two-port network $N$,  described by its transfer matrix
\[
N(\iw) = \pmatr{N_{11}(\iw) &N_{12}(\iw)\\N_{21}(\iw) &N_{22}(\iw)}
\]
and interconnected to passive termination immitances as in \Cref{fig:portrep}, is stable if and only if
\begin{equation}R_{11} > 0\text{ or } R_{22} > 0,\label{eq:llew2}\end{equation} and
\begin{equation}4\left(R_{11}R_{22}+X_{12}X_{21}\right)\left(R_{11}R_{22}-R_{12}R_{21}\right)-\left(R_{12}X_{21}-R_{21}X_{12}\right)^2 > 0\label{eq:llew3}\end{equation} 
or
\begin{equation}2R_{11}R_{22}-\abs{N_{12}N_{21}}-\Re{N_{12}N_{21}} > 0\tag{\ref{eq:llew3}$'$}\label{eq:llew3p}\end{equation}
for all $\omega\in\Realext$, where $R_{ij}$ and $X_{ij}$ denote the real and imaginary parts of $N_{ij}$ respectively.
\end{thm}


As shown in \cite{rollett}, the conditions stated in \thmref{thm:llw} are invariant under immitance substitution. 


=====================Still draft================




\vspace{3cm}
==============Sentence drafts after this===============










For our purposes, we consider only the immitance 
matrices that describe $G$ as an input-output mapping (as opposed to transmission or ABCD parameters) as {follows}:
\begin{equation}
\pmatr{q\\y}=\pmatr{G_1 &G_2\\G_3 &G_4}\pmatr{p\\u} \ , \ \pmatr{p\\u}=\pmatr{\Delta_s &0\\0&\Delta_l} \pmatr{q\\y}.
%\label{eq:pasgdel}
\end{equation}
Therefore, the overall interconnection can be {depicted} by the block diagram given in \Cref{fig:portrep}. 
In relation to teleoperation, the {blocks} $\Delta_s$ and $\Delta_l$ refer to the human and the unknown environment. 



\section{Synthesis}


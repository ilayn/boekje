\chapter{Conclusions}
\label{chap:conc}

In the light of all the arguments presented so far, we believe that we have clarified the underlying discrepancy between the actual bilateral 
teleoperation problem and how academic literature handles it. Moreover, we have provided a competitive alternative method that led to a high
degree of realism experimentally. Therefore, we can safely claim that many technical problems reported in the literature are methodological and 
can be solved efficiently via careful modeling and robust control design phases without invoking questionable assumptions a priori. 
However, this claim does not reach out to the real solution of the bilateral teleoperation problem. Put better, we only claim that we have provided
the solution for the simplified version of the problem. The actual question of \emph{What makes this device good?} or even the harder 
\emph{How can we improve the device?} are completely open. Unfortunately, we believe that clarifying this fact is a contribution.

Nevertheless, an important property of the method proposed here is that it provides consistent performance over different motion profiles and 
does not suffer from the artifacts of the method we have utilized. Moreover, we do not alter the hardware specifications by introducing 
virtual dissipation elements hence the results can be used to create devices that can be used in the investigations of the aforementioned 
open questions.  

Another important problem that we have not touched upon is the delayed bilateral teleoperation synthesis problem that somewhat dominated the
literature as if the undelayed case is completely studied. To justify our deliberate choice we distinguish two cases; there is a strong 
possibility backed by various studies that there exists an upper bound on the delay duration a human operator can cognitively compansate for 
during a bilateral teleoperation task. Beyond this delay value the human cannot associate the remote motion with the local device motion. 
Therefore let us denote this upper bound by $T$ whatever it might be other than being very likely to be around \SI{1}{\second}, then we 
have the following obvious dichotomy,

\begin{itemize}
	\item The actual transmission delay is higher than $T$,
    \item The actual transmission delay is lower than or equal to $T$.
\end{itemize}

In the first case, we believe that there is no need to even consider the problem since it is even harder to find the relevant performance 
objectives of how a human operator can be made to immerse into the task with an excessive time delay. In the second case, if there is 
significant time delay, one can use the multipliers given \Cref{chap:analysis} for delay uncertainties and utilize it in the synthesis 
method given in \Cref{chap:synth} directly, with guaranteed improvements over the existing techniques. Additionally, if the communication 
delay is of the time-varying nature, one can always buffer the input/output to regularize the delay and use the known upper bound of the 
buffer period. The reason why this would always give a better performance is simply because the existing time-varying delay robustness analysis 
and synthesis tools are simply too conservative. Utilizing them surely would lead to stable interconnections but at the cost of unacceptable
low performance levels which undermines the motivation the problem. Dealing with a known constant time-delay is, in turn, much easier and 
sharper results can be obtained. Note that practically every packet-switched network video/audio stream protocol use such buffering schemes 
unlike the vast majority of the time-delay teleoperation literature. In fact, this is not even a control-theoretical issue and should be left 
to digital communication experts for the optimal methods which go well-beyond control design knowledge. Moreover, the problem is far more 
sophisticated than the choice between TCP or UDP protocols. 


Following this argument, we have the following conclusions elaborated in this thesis;
\begin{enumerate}
	\item The bilateral teleoperation is fundamentally an interdisciplinary problem. Current literature underestimates the broadness of the 
    scope of this technology and claims to solve a stability problem that is not inline with the actual bilateral teleroperation. The majority
    of the proposed problem formulations are of \emph{What if we had sampling, two users, time-varying delay?} nature. Though, these scenarios 
    are certainly worth considering, the proposed solutions only handle the stability issues. Our first conclusion emphasizes this;
    
    {\bfseries The bilateral teleoperation problem is not a typical control problem in which stabilization is the crucial point and 
    achievable performance is an extra bonus. Without the required performance levels, a stable bilateral teleoperation system is useless.It 
    might even decrease the human task performance.}
    
    \item As we have shown in \Cref{chap:application}, the 2-port network modeling framework is not general enough to capture the problem in
    its entirety. Over the last two decades, certain derivations are established as facts for the perfect transparent device however one 
    can still obtain better designs with alternative methods that do not obey the predicted performance conditions which are stated in 
    numerous sources. If the following definition of transparency is adopted
    \begin{displayquote}[\cite{hirchebookchap}][.]
    Transparency is defined, meaning that the human operator should ideally feel as if directly acting in the remote environment
    (is not able to feel the technical systems/communication network at all).
    \end{displayquote}
    which seems to be the case in the literature, then there is a discrepancy between what is being sought after and the corresponding 
    formulation. 
    
    {\bfseries Transparency objective that relates the performance to the operator feel and comfort with the definiton above does not 
    necessarily imply that an ideal teleoperation system should have a hybrid system representation $\begin{psmallmatrix}0&I\\I&0\end{psmallmatrix}$. 
    This formulation completely ignores the human perception and moreover it is impossible to achieve. Additionally, as a control objective 
    it relies on naive control concepts such as exact dynamics cancellation and plant inversion. In a time-varying system these arguments 
    are invalid.
    }
    \item A vast majority of the network-theory based stability conditions can be rederived by the IQC framework in a lossless fashion. 
    Due to this equivalence there is no added value of using scattering transformations or wave variables over the proposed framework. 
    
    {\bfseries Insisting on the network theoretical treatment of the subject is a matter of preference. IQC framework already covers the 
    classical methods and offers significantly larger set of possibilities to be utilized in stability analysis and controller synthesis. Here 
    the emphasis is on the anachronistic focus of the literature. 
    }
    \item As we have shown via a simple implementation, high-performance controllers can be designed using a sufficiently accurate model of
    the system and careful simplifications by the robust control design methodology. 
    
    {\bfseries The D-K iteration with dynamic multipliers leads to significantly less conservative results compared to static multiplier 
    based designs which includes wave-variable based-and passivity based methods. The disadvantege of model based design is the weight 
    selection and that is a significant obstacle in judging the true optimality of the design. Even in the cases where the problem solution 
    is guaranteed to be optimal, the design itself due to the performance weights selection can be non-optimal. However, this difficulty 
    is not in par with the conservative methods. In other words, this difficulty can be overcome with educated guesses in a trial-and-error 
    phase which permits at least some systematic procedure up to an extent. A conservative method does not permit such by-pass steps.
    }
    \item It is shown that depending on the uncertainty modeling complexity, the control design problem can be made robust to different 
    uncertain operators of different kind including delays, parameters, particular nonlinearities etc. However, as reassuring and positive 
    it might seem, more robust solutions lead necessarily to lower performance levels. Therefore, it is the highest priority to get a 
    model with reduced uncertainty as much as possible. Even if there does not exist a suitable method to handle the resulting obstacles, 
    this will nevertheless make the problem visible and unambigious. 
    
    {\bfseries Due to the absence of a rigorous objective, we might pursue for the improvements over the method presented here. The 
    immediate improvement that can be relevant is the application of Linear Parameter Varying controller synthesis via scheduling over the 
    forces sensed in remote and local environments. The synthesis framework is already established however once again, the performance 
    objective is missing therefore we hit the same bottleneck. 
    }
    \item The teleoperation literature (and partially the control theory literature in general) have the tendency to motivate engineering 
    problems inside a seemingly rigorous framework. However, often there is an implicit transition from the actual problem to the watered 
    down oversimplified version of the same problem that disconnects the solution from the physical motivation. Alternatively, unrealistic 
    reasons are used to justify certain assumptions. A typical example is the claim that force sensors are often expensive to implement. That 
    statement is only true relatively. If we can succeed providing the operator a realistic touch sensation, the sensors would compensate the
    investment in a very short amount time\footnote{A surgical robotic system price is in the order of million USD with very high maintenance 
    costs}. We certainly refrain from declaring what is worth of focus or not however motivating a mathematical problem with questionable
    engineering scenarios is false and unfortunately very common. 
    
    {\bfseries There are more important open problems than the rather specialized delayed teleoperation problem or sans-sensor teleoperation.
    The delay robustness problem is studied in the last two decades extensively outside the teleoperation context. As a result of this 
    effort we already have a variety of methods IQCs, Lyapunov-Krasovskii functionals etc. The importance of the delay instability is due 
    to the unrealistic ambitious objective of physics equalization. Similarly force sensorless teleoperation or multi-user teleoperation etc.
    are problems with invalid motivation from a technological point-of-view since the bilateral teleoperation system is an human oriented
    technology not a network controlled system. The problem is finding the tool for designing the controller not modifying the problem 
    for the control design tool.
    }
    
\end{enumerate}



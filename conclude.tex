\chapter{Conclusions}
\label{chap:conc}

In the light of all the arguments presented so far, we believe that we have clarified the underlying discrepancy between the actual bilateral 
teleoperation problem and how academic literature handles it. Moreover, we have provided a competitive alternative method that led to a high
degree of realism experimentally. Therefore, we can safely claim that the internal technical problems specific to the academic circles can 
be solved efficiently via careful modeling and robust control design phases without invoking questionable assumptions a priori. Unfortunately, 
this claim does not reach over to the real solution of the bilateral teleoperation problem. Put better, we only claim that we have provided
the solution for the watered-down academic version of the problem. The actual question of \emph{Does this feel good?} or even the harder 
\emph{How can I make it feel better?} is completely open. 

Nevertheless an important property of the method proposed here is that it provides consistent performance over different motion profiles and 
does not suffer from the artifacts of the method we have utilized. Moreover, we do not alter the hardware specifications by introducing 
virtual dissipation elements hence the results can be used to create devices that can be used in the investigations of the aforementioned 
open questions.  

An important problem that we have not touched upon in this thesis is the delayed bilateral teleoperation problem that somewhat dominated the
literature as if the undelayed case is completely studied. To justify our deliberate choice we distinguish two cases; 







the bilateral teleoperation problem reported in the literature is, 
in its current state, an academic practice that cannot be utilized 

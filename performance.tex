\chapter{Performance Objectives}
\label{chap:perf}


Bilateral teleoperation problem is probably one of the most difficult problems in the control field due to its 
subjective nature involving the human comfort and liking. However, to put it bluntly, the experts are not helping 
either. In other words, most of the \enquote{good performance} motivations come from the first principles. Some 
literature argue that since most of the tools we, human operators, utilize are (almost-)lossless, say, a 
screwdriver or even a simple stick, it's natural to seek for a passive bilateral teleoperation system that 
ideally behaves like a rigid transmission mechanism. In general, there is no established consensus on what makes 
a teleoperation system good. Quite the contrary, this question is openly and unambiguously avoided in some 
well-known articles but some alternatives are proposed and rigorously pursued to the end. Therefore, their conclusions 
are the implications of their initial hypotheses. However, the results that follow these publications do not take 
this crucial detail into account and proceed as if the performance objectives are indeed the ultimate goals. 
We have to claim that the motivation of most of the studies given in the literature erroneously put emphasis 
on performance objectives that are at best questionable. 

On the other hand, there is a different school that focuses only on stability of the teleoperation system in the 
face of human, environment, communication line, quantization and many more uncertainty/perturbation sources. 
Especially some nonlinear control studies do not even bother to define performance criteria. This view simply 
regards the human and the environment as perturbations to be protected against for our precious robotic systems 
and neglects the \enquote{\emph{reason d'etre}} of the very problem that is under consideration. In our opinion, 
operator perception is the indispensable performance objective and can not be overlooked. More importantly, it is
beyond the scope and expertise of control theory (though with certain overlap) to find the relevant 
objectives. Other experts of the related fields need to contribute from a technological point of view in 
contrast with a pure physiological point of view and, in fact, guide the control theorists and practicioners towards 
the relevant issues. 

We have the strong opinion that the contemporary bilateral teleoperation control results, including this thesis, 
can not and thus should not claim a comprehensive understanding of a good and useful bilateral teleoperation 
system. Because we just don't know, yet.

\section{Types of Performance}

In terms of the quality of the force-feedback, there are a few leading choices of methodological performance 
definitions. The widely accepted so-called \enquote{transparency} stems from the ideal case of lossless, undistorted
exact replica of the remote side physics at the local site. Hence the ultimate goal is selected to be 
faithfully representing the remote site motion and allowing the user intervene just as good as s/he is operating 
directly at the remote site. 

\subsection{Transparency}
Using a vision analogy, the less distorted a system transmits the remote motion to the local site, the better
the system is. Hence the term \emph{transparency}. This is defined in \cite{lawrence,yokokohjiyoshikawa} independently. 
The notion of transparency is handled via three ideal response definitions in \cite{yokokohjiyoshikawa}. In terms 
of the signals involved a perfect transparent 2-port network admits the hybrid matrix 
\[
\pmatr{v_{human}\\f_{human}}=\pmatr{0&I\\-I&0}\pmatr{f_{env}\\v_{env}}
\]
If we wish to translate this into a control theoretical performance objective, we have mainly two options. First 
is minimizing the differences between the measured quantities such as forces, velocities, positions etc. Second is 
to make our system approach to an ideal transparent system as much as possible. Thus, we can use a performance index
\[
\min_K \pmatr{f_{human}-f_{remote}\\f_{env}-f_{local}\\x_{local}-x_{remote}\\\vdots} 
\]
for the signals in some suitable space that we wish to consider. Note that, there is already $n$-channel controller
structure evident from these error signals.  Alternatively, using a suitable system norm and denoting the controlled 
$2$-port teleoperation system $N(K)$, the problem becomes
\[
\min_K \abs{N(K) - \pmatr{0&I\\-I&0}}.
\]
This is obviously intuitive and agrees with the underlying physics. Thus, one might argue that the global minimizer
of these optimization problems would lead to the best teleoperation system. However, there are two additional implicit 
assumptions made here. On one hand, it is assumed that there is a partial ordering, in other words, if $K_1$ has the cost 
$c_1$ and $K_2$ has the cost $c_2$ with $c_1<c_2$ then this implies that $K_1$ is better than $K_2$ which is not 
necessarily true or better, it holds only for some particular $K$'s that are close enough to the ideal case and in general no
such ordering can be expected by this performance index. It might happen that the cost function is not even continuous 
let alone being smooth e.g. the transparency is only achieved by the global minimizer $K^*$ and not with any other $K$. 
On the other hand, we don't have a metric for how much we need to get close to the ideal matrix. Let us first quote three 
very important questions posed by Lawrence in his well-known paper;
\begin{displayquote}[{\cite{lawrence}}][.]
In practice, perfectly transparent teleoperation will not be possible. So it makes sense to ask the following questions:
\begin{itemize}
	\item What degree of transparency is necessary to accomplish a given set of teleoperation tasks? 
	\item What degree of transparency is possible?
	\item What are suitable teleoperator architectures and control laws for achieving necessary or optimal transparency?
\end{itemize}
We focus on the second two questions in this paper. Instead of evaluating the performance of a specific teleoperation architecture,
as in [2], we seek to understand the fundamental limits of performance and design trade-offs of bilateral teleoperation in
general, without the constraints of a preconceived architecture
\end{displayquote}
In \cite{lawrence}, Lawrence then invokes the passivity assumption and then passivity theorem to arrive at structural 
properties of the controller $K$. Evidently, this allows for the back-substitution of the controller entries and solution
for the ideal case. Then the resulting controller is denoted with \enquote{\emph{Transparency Optimized Controller}}. In 
control theoretical terminology, this amounts to a cross-coupling control action where the bilateral dynamical differences
are canceled out and then SISO control channels are tuned to maximum performance bound to the stability constraints. It 
should be clear that this is the transparency from a passivity-based point of view and we refrain from iterating 
what is given in the previous chapter. 

Even if we accept it to be the distinguishing performance criterion, we have to emphasize that we have not touched the most 
important question, that is the first of the three, rather we hope to achieve the required transparency levels just enough 
to fool the user. After two decades, this point is in our opinion simply discarded and many studies in the literature somewhat 
treats the conclusions of Lawrence in a different context than what has been given by Lawrence. As is for the case for the 
Hogan's paper on passivity, Lawrence never claims that this is a definite performance measure. Instead he clearly shows the 
implications that follow from such assumptions. 


Moreover, there are interesting studies inline with our claims about irrelevance of the remote media recreation in bilateral
teleoperation problem. For example, \cite{kilchenman,wildenbeest,boessenkool} and a few other studies report that there is a saturation
effect on how much realism that can be projected to the user. In other words, there is an inherent bandwidth limitation for 
the realism increase such that beyond a certain band of frequency, the transparency does not increase significantly, possibly
unless backed up by tactile feedback. Even further, in the case of shared control applications, it might happen that transparency 
is not needed at all. 



\subsection{\texorpdfstring{$Z$}{Z}-width}

In \cite{colgate4}, the performance of a haptic device is related to the dynamic range of impedances (hence the name $Z$) that
the device can display to the user. In this context we have two extremes; on one hand we have purely the local device impedance 
for the free-air motion and on the other hand we have the maximally stiff local device for the rigid and immobile obstacle collision. 
Let $Z_f$, $Z_c$ denote these two distinct cases then the more pronounced the difference between these impedances, the more 
capable the teleoperation system can reflect various impedances inbetween. Thus, we implicitly assume that the rigid contact 
case and the free-air case are the extreme points of the uncertainty set and testing for these two cases are sufficient to 
conclude that any impedance on the path from $Z_f$ to $Z_c$ is a valid impedance that can be displayed by the device. This
in turn implies that there is an ordering in the uncertainty set from \enquote{big} to \enquote{small} etc. and moreover 
the destabilizing uncertainty is at the boundary of the set such that these two extreme cases can vouch for stability over 
the whole possible environments. We are not fully convinced that this should be the case for all possible environment 
scenarios. A particular subset of second-order mass-spring-damper models of environments can be shown to be compatible 
with this claim if passivity theorem is used. However, when combined with other uncertain blocks in the loop we don't see
how the argument follows.

Similar to what Lawrence has given, the authors also include a clear statement of purpose: 

\begin{displayquote}[{\cite{colgate4}}][.]
This paper will not address the psychophysics of what
makes a virtual wall \enquote{feel good} except to say that one
important factor seems to be dynamic range. An excellent
article on this topic has recently been written by
Rosenberg and Adelstein [11]\footnote{Reference \cite{rosenberg} of this thesis.}. 
We will present instead
some of our findings, both theoretical and experimental,
concerning achievable dynamic range. In short, we will
address the question of how to build a haptic interface
capable of exhibiting a wide range of mechanical
impedances while preserving a robust stability property
\end{displayquote}


Under these assumptions, via defining a functional to measure the distance between $Z_f$ and $Z_c$, we can assess the performance
of different bilateral teleoperation devices. In \cite{goranthesis}, $Z$-width is defined as 
\begin{equation}
Z_{\text{width}} = \int_{\omega_0}^{\omega_1}{\abs{\log|Z_{\vphantom{f}c}(\iw)|-\log |Z_f(\iw)|}}d\omega
\label{eq:zwidth}
\end{equation}
or alternatively, a simulation/experiment-based method can be utilized as in \cite{weir}. 


Note that \eqref{eq:zwidth} does not appear in the original paper \cite{colgate4} but proposed in \cite{goranthesis,passenberg} 
though we can see neither the reasoning behind this expression nor how it constitutes a comparative quantity. In both 
\cite{colgate4,goranthesis} no additional information is provided except some general rules of thumb about device 
damping and other related issues. 

It should be noted that the differences at each frequency are lumped into one scalar number and moreover, the impedance 
gain curves can cross each other (see \cite{goranthesis}) and might lead to an overly optimistic result. Similarly, 
resonance peaks and zeros of the involved impedances can be smeared out if we solely rely on this functional. 

Since $Z_f$ and $Z_c$ are functions of the environment impedance, these curves can be obtained for one particular environment
at a time. This also holds for the derivation of \cite{lawrence}. In \cite{goranthesis}, the difference  is evaluated for more 
than one environment and then averaged out i.e. let $Z_{act}(Z_e)$ be the impedance displayed to the user in order to render 
$Z_e$ on the local site. Then, for a particular controller, average $Z$-error to each candidate $Z_e$ is given by
\begin{equation}
Z_{avgerr} = \frac{1}{n}\sum_{i=1}^n{\left[
    \frac{1}{\omega_{1i}-\omega_{0i}}\int_{\omega_0}^{\omega_1}{%
                                     \abs{\log|(Z_{act}(Z_{ei}))(\iw)|-\log |Z_{ei}(\iw)|}}d\omega.
                                     \right]}
\label{eq:zdiff}
\end{equation}
This cost function is denoted by \enquote{\emph{Transparency Error}} or \enquote{\emph{Fidelity}}.We refer to \cite{weir} 
for a more detailed discussion. 


\subsection{Fidelity}
In \cite{cavusoglu}, a variant of a transparency error is proposed to assess the performance. In this context, 
the emphasis is on the variation of the environment impedance and the resulting effect on the displayed impedance. 
Also the motivation is focused on the surgical procedures via bilateral teleoperation. If, for 
example, the remote device slides over some tissue that involves a tumor or any other irregularity that would be felt
if the same motion would have been performed directly by the surgeon, the better the nuances transmitted, the higher
the fidelity. This performance objective is in a sense enforces the high frequency content of the information (closer
to tactile bandwidth). It has been noted that the JND of \SIrange{14}{25}{\percent} for the relative compliance 
of distinct surfaces goes under \SI{1}{\percent} for fast compliance variation detection via scanning a surface
\cite{dhruvtendick}. Similar to the definitions given for transparency, the change of the displayed impedance 
$Z_{disp}(Z_e)$ with respect to the change in the environment $Z_e$ can obtained via a straightforward calculation.


Given the scalar complex LTI uncertainty block $\Delta$ and the LTI plant $G\in\mathcal{RH}_\infty^{2\times 2}$, 
such that 
\[
\pmatr{q\\z} = G\pmatr{p\\w},
\]
the LFT interconnection of $\Delta-G$ is given by, 
\[
P = G_{22}+G_{21}\Delta\inv{(I-G_{11}\Delta)}G_{12}
\]
Here $P$ denotes the impedance seen by the operator, $G$ denotes the teleoperation system and $\Delta$ being the 
environment impedance. Now, under the well-posedness assumption, define the derivative operation with respect to change in $\Delta$
\[
\frac{d}{d\Delta} P = \frac{G_{21}G_{12}}{(I-G_{11}\Delta)^2}.
\]
Then, though not pursued in \cite{cavusoglu}, this can, in turn, be rewritten as an LFT again;
\[
\pmatr{q_1\\q_2\\z} = \pmatr{2G_{11} &-G_{11} &1\\G_{11} &0 &0\\ 2G_{11}G_{12}G_{21} &-G_{11}G_{12}G_{21} &G_{12}G_{21}}\pmatr{p_1\\p_2\\w}
\]
and 
\[
\pmatr{p_1\\p_2} = \pmatr{\Delta &0\\0 &\Delta}\pmatr{q_1\\q_2}.
\]

Note that the matrix case follows a similar but more involved computation. This led to the authors defining 
a transparency-like performance objective using a rather subtle choice of system $2$-norms. Let $G$ be a 
stable LTI system with transfer matrix $G(s)$. Then 
\[
\norm{G}_2^2 \coloneqq \infint{\trace(G^*(\iw)G(\iw))}d\omega
\] 
The measure of fidelity is defined as the norm
\[
\norm{\left.W_s\frac{dP}{d\Delta}\right|_{\Delta_{enom}}}_2
\]
where $W_s$ is a typically low-pass type weighting function to emphasize the frequency band of interest. Therefore
the synthesis problem is to find the optimizer, controller $K$ to the problem
\[
\sup_{\substack{\text{Stability}\\\text{Other Constraints}}}\inf_{\Delta_{ei}\in\bm{\Delta_e}}
\norm{\left.W_s\frac{dP}{d\Delta}\right|_{\Delta_{ei}}}_2
\]
where $\Delta_{ei}$ are the worst case environments that are of interest.


There are a few interpretations of this norm in the literature, mainly, the deterministic \enquote{area under the Bode
Plot} interpretation i.e. energy of the impulse response for scalar case, and the stochastic \enquote{steady-state 
white-noise-input response}. We are under the impression that the authors argue in the line of the former interpretation
with a similar reasoning given in the $Z$-width discussion via an area computation. 


Designing a robust controller while minimizing the $\mathcal{H}_2$ norm of an uncertain system in the face of a predefined 
uncertainty set i.e. \enquote{Robust $\mathcal{H}_2$ Synthesis} problem has already recevied a lot of attention and 
the results can be found in the literature, e.g., \cite{dullerud}. Hence, the problem definition in \cite{cavusoglu} is 
in fact tractable which might be inline with the intuition reported about the convexity of the performance objective. However, 
it's not clear to us why we choose the system $2$-norm for the performance cost. Additionally, the infimum needs to be 
computed in the face of a set at infinitely many points hence an appropriate relaxation is required. This point is also not 
given though a gridding approach might have been utilized in the numerical optimization procedure described in the paper.

It is also not clear for which case we should utilize this performance objective. The initial difficulty is that all the 
involved operators are LTI hence there is no time variation involved. The test reads as; we select an arbitrary element 
in the predefined uncertainty set, say $\Delta_e$, and evaluate the derivative at $\Delta_e$. Hence, in some $\epsilon$-
neighborhood of $\Delta_e\in\bm{\Delta_e}$ we can see the change in $P$. Thus, if the environment is slightly off from our 
nominal guess, this tells us how much fidelity measure would change. But the environment is still assumed to be LTI.


Note that, this does not imply that time-variations are taken into account. Suppose a particular admissible trajectory
$\hat{\Delta}_e(t)$ in time is given such that $\hat{\Delta}_e(t_1)=\Delta_e$ i.e. its time-frozen LTI copy coincides 
with the particular nominal environment model $\Delta_e$ and at some time instant $t_2$, it  coincides with another 
LTI model $\mathring\Delta_e$ that is within some $\epsilon$-neighborhood of $\Delta_e$. Even if we achieve very good 
fidelity properties evaluated at each $\Delta_e$ and a sequence of LTI model elements each being in the small neighborhood 
of the other, this does not guarantee that we would have good fidelity for the trajectory $\hat{\Delta}_e$. Actually, 
it might be more desirable to have low fidelity since drastic changes in the performance with respect to LTI uncertainties
might confuse the user.


\section{Closing Remarks and Discussion}
There are a few other performance criteria reported in the literature. Consider the definition of the impedance seen by 
the operator $P$ above. In \cite{katsura}, this term is divided into two individual terms, denoted by 
\enquote{reproducibility} and \enquote{operationality}. The idea is similar to a sensitivity/complementary sensitivity 
function definitions. 

In \cite{yokokohjiyoshikawa}, also an ideal response is also partitioned into two parts and denoted by \enquote{index of 
maneuverability} and, in essence, is similar to what is given above, hence omitted. 

We refer to the survey papers \cite{hokayemspong,passenberg} for a general treatment and \cite{klomp,dennis} and references 
therein for a more detailed overview about many variations in the literature.


In summary, there is no general perfomance criteria that can lead to a dedicated control design procedure. The aforementioned 
performance objectives always start from the direct manipulation case and assume a distance between the interacting parties.
Then the implications of such hypotheses are pursued and some results are obtained. It might very well happen that all or none 
of those conclusions are correct. Put better, these studies always try to remedy the distortion caused by splitting two 
interacting bodies. Thus, the goal becomes too ambitious at the outset. Similar to the delay problem there is not much
we can do about the distortion within the laws of physics. In fact, even the slightest delay can destabilize the system. 
Thus, we can speculate that by doing so, we create a stability problem that we should not have had in the first place. Moreover, 
as we have mentioned in the introduction, the problem is exclusively about human perception and not related to the reconstruction 
of the remote scene. As long as we can \enquote{fool} the user for the sake of efficiency and operational comfort, we are done. 

Because we fail to provide an alternative performance criterion, we are forced to use a particular analogy from the audio
technology, to express more clearly what we intend to emphasize. 

When it comes to the faithful reconstruction of the recorded audio, contemporary high-end sound systems offer great fidelity, hence 
the name hi-fi systems\footnote{The term was coined way before the systems become truly hi-fi compared to today's systems}. There has 
been such a great success that now listeners and component manufacturers are striving for a full audio immersion i.e. listening to 
a recording that feels like actually sitting in the concert hall or venue. However, very similar to the transparency discussion 
in bilateral teleoperation, there is a fundamental obstacle in rendering a live performance sound with recorded version of it. 
Following quote is from \cite{atkinson}:

\begin{displayquote}
\enquote{What on earth can be the readily identifiable difference}, I wrote in 1995, \enquote{between the sound of a loudspeaker 
producing the live sound of an electric guitar and that same loudspeaker reproducing the recorded sound of an electric guitar?} 
I went on to conjecture that the act of recording inevitably diminishes the dynamic range of the real thing. The in-band phase 
shift from the inevitable cascade of high-pass filters that the signal encounters on its passage from recording microphone to 
playback loudspeakers smears the transients that, live, the listener perceives in all their spiky glory. And as a high-pass filter 
is never encountered with live acoustic music, that's where the essential difference must lie, I concluded, quoting Kalman Rubinson 
(who had not yet joined the magazine's team of reviewers) that \enquote{Something in Nature abhors a capacitor.}

But two more recent experiences suggest that there must be more to the difference than the presence of unnatural high-pass filters. 
(...)
\end{displayquote}

The author goes on to list the involved hardware, the signal chain, and other relevant details about the mic set-up at both events.
In the first event, a performer plays a piece through the listed hardware and is simultaneously recorded by the author. Then the recorded 
version of the performance is played back to the same audience from the same hardware. In the second event, a nontrivial analog/
digital hybrid device supposedly replicates a grand piano via sophisticated mechanisms used to generate the sound. In both 
events, it has been noted that though the reproduction quality is quite impressive for the audience, a certain liveness was missing.

\begin{displayquote}
So these days, I'm starting to feel that it is something that is never captured by recordings at all that ultimately defines the 
difference between live and recorded sound. (...)[the described systems] succeeded in 
every sonic parameter but one: the intensity of the original sound. Intensity, defined as the sound power per unit area of the 
radiating surface, is the reason why, even if you could equalize a note played on a flute to have the same spectrum as the note 
played on a piano at the same sound pressure level, it will still sound different.

Ultimately, therefore, it is perhaps best to just accept that live music and recorded music are two different phenomena. (...)
Eisenberg's thesis\footnote{See \cite{eisenberg})} is that any attempt to capture the sound of an original event is doomed to 
failure, and that stripping a concert from its cultural context by recording only the audio bestows a sterility on the result 
from which it cannot escape. The recording engineer may be able to pin the butterfly to the disc, but it sure doesn't fly any 
more.(...)

In Eisenberg's words, \enquote{In the great majority of cases, there is no original musical event that a record records or reproduces. 
Instead, each playing of a given record is an instance of something timeless. The original musical event never occurred; it exists, if 
it exists anywhere, outside history.}
\end{displayquote} 



Obviously, these are all subjective opinions rather than rigorous scientific propositions though the first anecdote can be considered as
a user experience study. However, we have to remind that the audio technology is tremendously advanced if compared to haptics and 
teleoperation. In fact, the comparison is not fair in the sense that bilateral teleoperation is not a true technology yet but rather 
in its infancy. Still, after decades of improvements, the sound systems are not capable of producing a live sound, real enough to make 
the listener immerse into, though comes impressively close. Nevertheless, the performance objectives that we have enumerated a few above 
claims to compete at the level of hi-fi systems which is simply too ambitious.

Coming back to our discussion, in the light of our analogy, we think that the bilateral teleoperation literature is focused on finding
the system that can deliver the \enquote{live sound} rather than a high quality \enquote{playback}. This holistic search is certainly
relevant to the field but it cannot serve as the justification as a driver of technological advances reported in the literature. Task-%
dependence is already emphasized in many studies as an item of importance and a too general performance criterion would be very unlikely
to serve as a general guideline. Though, we acknowledge the motivation behind the holistic approach and a truly transparent device 
might be the ideal, we also believe that the timing and the feasibility of this approach needs to be modified. The immediate engineering 
problems such as the communication delays and other \enquote{today} problems are the strong indicators of the fact that such a goal cannot 
be tackled prematurely. We can't overemphasize the key issue; even the undelayed case remains unresolved let alone (time-varying 
or constant) delayed case. 


Again from our analogy, it took decades for the hi-fi systems to reach to the current level to claim that
a search for the live sound is justified. The sound reconstruction task is divided into components such as amplifiers, pre-amplifiers, 
direct digital-to-analog converters etc. for the signal conditioning and similarly the sound regeneration is also divided into active-passive 
loudspeakers with having dedicated single or multiple tweeters, sub-woofers etc. Only then the community is convinced that the hardware 
is not the problem\footnote{We also
have to state that there is an additional compulsive habit of overemphasizing the component quality such as the transmission cables etc. 
Hence, we can observe a trend among hi-fi enthusiasts of picking up artifacts that are impossible to be audible or simply don't exist.}.
Similarly, TVs and other futuristic vision technologies are following the same trend for the ultimate vision quality. However, if compared 
with these, bilateral teleoperation definitions are nothing but academic stability problem exercises. Moreover, as we show in the next
chapter, these problems are, whether linear or nonlinear, no different than the mainstream control problems in disguise. Therefore, we
can not yet argue about a dedicated stability and a performance problem for bilateral teleoperation. This is the reason why we have 
chosen a significantly advanced methodology again from mainstream control theory and applied to the bilateral teleoperation problem. 
This does not imply that we have offered an alternative, in fact, quite the contrary, our goal is to make it obvious that the studies 
so far can be subsumed into the mainstream control theory research and there is no benefit to keep the seemingly specialized terminology 
of the field. However, we have shown that if the problem is in fact a special case of a general control problem there is no need to use
the outdated methodologies while important advances are reported in the literature in the past two decades over the small-gain/passivity
based results. 



A search over the number of studies in the literature published in the 2000's to date with \enquote{bilateral teleoperation} and 
\enquote{delay} as keywords gives hundreds of results. Yet we have no clear understanding of why these devices are unstable. We 
can pinpoint different effects depending on whether we look at it from an energy exchange/passivity point of view or from 
sensitivity function-based analysis. But this does not help us to prioritize certain design aspects of a high-performance system
and unfortunately we have to assume a few questionable hypotheses along the way. It's very difficult to follow the train-of-thought
often given in the literature as we first define the ultimate performance of a teleoperation system then we openly accept the fact that
this is not achievable, however, we, in turn, don't modify our performance criteria and then completely neglect the issue. Finally we 
convert the problem into a stability of some interconnected devices with a great uncertainty associated with them and then, in the majority
of the cases inject damping to the hardware which deteriorates not only the relative performance but usability of the device as a whole.
Specific to the problem at hand, having a stable but poor-performing bilateral teleoperation has less functionality than that of a 
unilateral teleoperation as the added-value of robotic manipulators are wasted with the inclusion of damping. 


After using the audio analogy extensively, let us finish the alternative suggestions with the same analogy. As we have briefly mentioned
the hi-fi loudspeakers involve different dedicated components for different frequency bands such as tweeters for high-frequency band,
sub/woofers for the low frequency bands and occasionally mid-range speakers etc. with individual drivers. Even though different components are 
utilized, the resulting harmony of these components lead to a very satisfactory listening experince if compared to generic single driver
louspeakers. Now, obviously somatosensory system already involves of such sensors which is reported to be responsible for different 
frequency bands (see \Cref{chap:intro}). Hence, this already gives us a direct cue for separating the motion into different categories 
in terms of the frequency/amplitude content. We don't have a working methodology yet however we would like to include our reasoning here
for comparison. 



There has been quite a number of studies published on combining the tactile feedback with kinesthetic feedback such as \cite{kammermeier,
kimcolgate,kyung,pacchierotti,meli} among many others and references therein. In many of these reports it has been clearly shown that 
combining these modalities lead to substantial increase in human perception about the unstructured environment. Not only the motion but 
also temperature can be transmitted via the tactile thermal actuators. Hence, different overloading of the vibrational patterns can also 
be obtained. We have to underline that the studies mentioned above does not necessarily promote our reasoning for the control design but 
rather superimposing tactile and kinesthetic perception simultaneously. However, there is no apparent obstacle to use the same hardware 
for cooperative kinesthetic profiling. 



Therefore, from a control design perspective this gives rise to a completely different type of performance objective that has no relation 
with the immediate transparency requirements. In other words, the performance of the device is now comprised of the individual excitation 
of the required human receptors. And this is inline with what we have touched in the introduction of this thesis. To the critical reader
this might look like we are shifting the difficulties of the control design to the hardware design since the required devices are indeed
nontrivial. We would argue that it is not the case. The required hardware already exists but used in a different context. We have performed 
preliminary signal processing studies and there is no significant result that can be reported here. Nevertheless, we can speculate about 
the links a solution along these lines can bring in to finalize this discussion. 


The immediate possibilities are to separate either the amplitude or the frequency content of the force signal for playback. In the frequency
case; instead of a \enquote{\emph{physics}} matching goal, we instead encode the force signal to be similar to an audio signal. Then, the 
bilateral teleoperation system goal is to playback the measured force pattern with different components simultaneously. This has been pursued for 
pre-recorded signals in \cite{kuchenbecker} within the concept of \enquote{Event-based} haptics. The authors do not use measured 
signals but material contact signals from a contact-bank or look-up table but an increase of perception quality is noted. If we can achieve this separation, 
we have the option to separate the bandwidth of different components and hence making the performance specifications much more 
relaxed if compared to one device-the whole frequency spectrum strategy. Moreover, the high frequency contact information can be shifted 
to the \enquote{tweeter} of the device and this makes the low frequency force playback much easier. The low-bandwidth transmission is already 
well-studied and within reach of the current technology. Hence, a standalone high frequency action can be added on top of the perception 
which would otherwise lead to a deteriorated performance if pursued with the same device. A hard contact can still be displayed with a 
rather compliant \enquote{sub/woofer} and an agile stiff tweeter. Moreover, a stiff wall can possibly be rendered up to the perception 
threshold via it's Fourier components in other words, a hard contact perception might be achieved with a combination of mid-high frequency
vibrations contingent upon the task requirements. The force signal can be transmitted and decomposed into frequency band at the local site
or directly decomposed and sent over different channels with different line. This might even encapsulate the Model-mediated teleoperation-like
fast/slow bus discrimination without the need of recreating a proxy virtual environment update (See e.g. \cite{mitraniemeyer}). Moreover, this 
solution already embodies an inherent robustness to packet losses and similar artifacts as we have no obligation to match the physics any more. 
We conjecture that this would not jeopardize the stability but will make the sensation only deteriorate as we would expect from a noisy telephone 
conversation. 


It can be argued whether this would lead to a faithful representation of the remote location and we can directly see that the answer is no. 
However, we are trying to remove precisely that requirement and put a device-dependent varying degree of realism instead of working for 
stability and neglecting performance. 


Without any further evidence, there is not much we can extrapolate hence we will leave this discussion to a future work. In what follows
we will instead use a typical force error/position error minimization based control design and show that at least we can achieve good 
robustness properties for a large class of uncertainties due to human/environment dynamics with relatively high performance. 



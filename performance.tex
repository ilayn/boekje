\chapter{Performance Objectives}
\label{chap:perf}



Bilateral teleoperation problem is probably one of the most difficult problems in the control field due to its 
subjective nature involving the human comfort and liking. However, to put it bluntly, the experts are not helping 
either. In other words, most of the \enquote{good performance} motivations come from the first principles. Some 
literature argue that since most of the tools we, human operators, utilize are (almost-)lossless, say, a 
screwdriver or even a simple stick, it's natural to seek for a passive bilateral teleoperation system that 
ideally behaves like a rigid transmission mechanism. 

\section{Transparency}
Using a vision analogy, the less distorted a system transmits the remote motion to the local site, the better
the system is. Hence the term \emph{transparency}. This is defined in \cite{lawrence,yokokohjiyoshikawa} independently. 
The notion of transparency is handled via three ideal response definitions in \cite{yokokohjiyoshikawa}. In terms 
of the signals involved a perfect transparent 2-port network admits the hybrid matrix 
\[
\pmatr{v_{human}\\f_{human}}=\pmatr{0&I\\-I&0}\pmatr{f_{env}\\v_{env}}
\]
If we wish to translate this into a control theoretical performance objective, we have two main 
options. First is minimizing the differences between the measured quantities such as forces, velocities, positions
etc. Second is to make our system behave like an ideal transparent system. Thus, we have
\[
\min_K \pmatr{f_{human}-f_{remote}\\f_{env}-f_{local}\\x_{local}-x_{remote}\\\vdots} 
\]
for all signals in some suitable space that we wish to consider. Note that, there is already $n$-channel controller
structure evident from these error signals.  Alternatively, using a suitable system norm and denoting the controlled 
$2$-port teleoperation system $N(K)$, the problem is 
\[
\min_K \abs{N(K) - \pmatr{0&I\\-I&0}}
\]
This is obviously intuitive and agrees with the underlying physics. Thus, one might argue that the global minimizer
of these optimization problems would lead to the best teleoperation system. However, there are two additional implicit 
assumptions made here. On one hand, it is assumed that there is a partial ordering, in other words, if $K_1$ has the cost 
$c_1$ and $K_2$ has the cost $c_2$ with $c_1<c_2$ then this implies that $K_1$ is better than $K_2$ which is not 
necessarily true or better, it holds only for some particular $K$'s that are close enough to the ideal case. It might 
happen that the cost function is not even continuous let alone being smooth e.g. the transparency is only achieved by 
the global minimizer $K^*$ and not with any other $K$. On the other hand, we don't have a metric for how much we need 
to get close to the ideal matrix. Let us first quote three very important questions posed from the seminal paper of Lawrence;
\begin{displayquote}[{\cite{lawrence}}][.]
In practice, perfectly transparent teleoperation will not be possible. So it makes sense to ask the following questions:
\begin{itemize}
	\item What degree of transparency is necessary to accomplish a given set of teleoperation tasks? 
	\item What degree of transparency is possible?
	\item What are suitable teleoperator architectures and control laws for achieving necessary or optimal transparency?
\end{itemize}
We focus on the second two questions in this paper. Instead of evaluating the performance of a specific teleoperation architecture,
as in [2], we seek to understand the fundamental limits of performance and design trade-offs of bilateral teleoperation in
general, without the constraints of a preconceived architecture
\end{displayquote}
In \cite{lawrence}, Lawrence then invokes the passivity assumption and then passivity theorem to arrive at structural 
properties of the controller $K$. Evidently, this allows for the back-substitution of the controller entries and solution
for the ideal case. Then the resulting controller is denoted with \enquote{\emph{Transparency Optimized Controller}}. In 
control theoretical terminology, this amounts to a cross-coupling control action where the bilateral dynamical differences
are canceled out and then single control channels are tuned to maximum performance bound to the stability constraints. It 
should be clear that this is the transparency definition from a passivity-based point of view and we refrain from iterating 
what is given in the previous chapter. 

Even it is so, we have to emphasize that we have not touched the most important question, that is the first of the three, 
rather we hope to achieve the required transparency levels just enough to fool the user. After two decades, this point is 
in our opinion simply discarded and many studies in the literature somewhat treats the conclusions of Lawrence in a different 
context than what has been given by Lawrence. As is for the case for the Hogan's paper on passivity, Lawrence never claims 
that this is a definite performance measure. Instead he clearly shows the implications that follow from such assumptions. 


Moreover, there are interesting studies inline with our claims about irrelevance of the remote media recreation in bilateral
teleoperation problem. For example, \cite{kilchenman,wildenbeest} and a few other studies report that there is a saturation
effect on how much realism that can be projected to the user. In other words, there is an inherent bandwidth limitation for 
the realism increase such that beyond a certain band of frequency, the transparency does not increase significantly. Moreover, 
in the case of shared control applications, it might happen that transparency is not needed at all. 



\section{\texorpdfstring{$Z$}{Z}-width}

In \cite{colgate4}, the performance of a haptic device is related to the dynamic range of impedances (hence the name $Z$) that
the device can display to the user. In this context we have two extremes; on one hand we have purely the local device impedance 
for the free-air motion and on the other hand we have the maximally stiff local device for the rigid and immobile obstacle collision. 
Let $Z_f$, $Z_c$ denote these two distinct cases then the more pronounced the difference between these impedances, the more 
capable the teleoperation system can reflect various impedances inbetween. Thus, we implicitly assume that the rigid contact 
case and the free-air case are the extreme points of the uncertainty set and testing for these two cases are sufficient to 
conclude that any impedance on the path from $Z_f$ to $Z_c$ is a valid impedance that can be displayed by the device. This
in turn implies that there is an ordering in the uncertainty set from \enquote{big} to \enquote{small} etc. and moreover 
the destabilizing uncertainty is at the boundary of the set such that these two extreme cases can vouch for stability over 
the whole possible environments.

Similar to what Lawrence has given, the authors also include a statement of purpose: 

\begin{displayquote}[{\cite{colgate4}}][.]
This paper will not address the psychophysics of what
makes a virtual wall \enquote{feel good} except to say that one
important factor seems to be dynamic range. An excellent
article on this topic has recently been written by
Rosenberg and Adelstein [11]\footnote{Reference \cite{rosenberg} of this thesis.}. 
We will present instead
some of our findings, both theoretical and experimental,
concerning achievable dynamic range. In short, we will
address the question of how to build a haptic interface
capable of exhibiting a wide range of mechanical
impedances while preserving a robust stability property
\end{displayquote}


Under these assumptions, via defining a functional to measure the distance between $Z_f$ and $Z_c$, we can assess the performance
of different bilateral teleoperation devices. In \cite{goranthesis}, $Z$-width is defined as 
\begin{equation}
Z_{\text{width}} = \int_{\omega_0}^{\omega_1}{\abs{\log|Z_{\vphantom{f}c}(\iw)|-\log |Z_f(\iw)|}}d\omega
\label{eq:zwidth}
\end{equation}

In both \cite{colgate4,goranthesis} no additional information is provided except general rules of thumb about device damping and other
related issues. In general, the inherent local device damping reduces the gain of $Z_f$ or the maximum stiffness displayed by the 
device increases the gain of $Z_c$ and hence the $Z_{\text{width}}$ increases.

It should be noted that all frequency ranges are rather lumped into one scalar number and moreover, the impedance gain curves can
cross each other (see \cite{goranthesis}) and might lead to an overly optimistic result. Similarly, resonance peaks and zeros of the
involved operators can be smeared out if we solely rely on the functional. 

Since $Z_f$ and $Z_c$ are functions of the environment impedance, one can only test 




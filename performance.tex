\chapter{Performance Objectives}
\label{chap:perf}


Bilateral teleoperation problem is probably one of the most difficult problems in the control field due to its 
subjective nature involving the human comfort and liking. However, to put it bluntly, the experts are not helping 
either. In other words, most of the \enquote{good performance} motivations come from the first principles. Some 
literature argue that since most of the tools we, human operators, utilize are (almost-)lossless, say, a 
screwdriver or even a simple stick, it's natural to seek for a passive bilateral teleoperation system that 
ideally behaves like a rigid transmission mechanism. In general, there is no established consensus on what makes 
a teleoperation system good. Quite the contrary, this question is openly and unambiguously avoided in some 
well-known articles but some alternatives are proposed. Therefore their conclusions are merely the implications of 
their initial hypothesis. However, the results that follow these publications do not take this crucial detail into account
and proceed as if the performance objectives are indeed the ultimate goals. We have to claim that the motivation 
of most of the studies given in the literature erroneously put emphasis on performance objectives that are at 
best questionable. 

On the other hand, there is a different school that focus only on stability of the teleoperation system in the 
face of human, environment, communication line, quantization and many more uncertainty/perturbation sources. 
Especially some nonlinear control studies do not even bother to define performance criteria. This view simply 
regards the human and the environment as perturbations to be protected against for our precious robotic systems 
and neglects the \enquote{\emph{reason d'etre}} of the very problem that is under consideration. In our opinion, 
operator perception is the indispensable performance objective and can not be overlooked. More importantly, it is
beyond the scope and expertise of control theory (though with certain overlap) to find the relevant 
objectives. Other experts of the related fields need to contribute from a technological point of view in 
contrast with a pure physiological point of view and, in fact, guide the control theorists and practicioners towards 
the relevant issues. 

We have the strong opinion that the contemporary bilateral teleoperation control results, including this thesis, 
can not and thus should not claim a comprehensive understanding of a good and useful bilateral teleoperation 
system. Because we just don't know, yet.

\section{Types of Performance}

In terms of the quality of the force-feedback, there are a few leading choices of methodological performance 
definitions. The widely accepted so-called \enquote{transparency} stems from the ideal case of lossless, undistorted
exact replica of the remote side physics at the local site. Hence our the ultimate goal is selected to be 
faithfully representing the remote site motion and allowing the user intervene just as good as s/he is operating 
directly at the remote site. 

\subsection{Transparency}
Using a vision analogy, the less distorted a system transmits the remote motion to the local site, the better
the system is. Hence the term \emph{transparency}. This is defined in \cite{lawrence,yokokohjiyoshikawa} independently. 
The notion of transparency is handled via three ideal response definitions in \cite{yokokohjiyoshikawa}. In terms 
of the signals involved a perfect transparent 2-port network admits the hybrid matrix 
\[
\pmatr{v_{human}\\f_{human}}=\pmatr{0&I\\-I&0}\pmatr{f_{env}\\v_{env}}
\]
If we wish to translate this into a control theoretical performance objective, we have two main 
options. First is minimizing the differences between the measured quantities such as forces, velocities, positions
etc. Second is to make our system behave like an ideal transparent system. Thus, we have
\[
\min_K \pmatr{f_{human}-f_{remote}\\f_{env}-f_{local}\\x_{local}-x_{remote}\\\vdots} 
\]
for all signals in some suitable space that we wish to consider. Note that, there is already $n$-channel controller
structure evident from these error signals.  Alternatively, using a suitable system norm and denoting the controlled 
$2$-port teleoperation system $N(K)$, the problem is 
\[
\min_K \abs{N(K) - \pmatr{0&I\\-I&0}}
\]
This is obviously intuitive and agrees with the underlying physics. Thus, one might argue that the global minimizer
of these optimization problems would lead to the best teleoperation system. However, there are two additional implicit 
assumptions made here. On one hand, it is assumed that there is a partial ordering, in other words, if $K_1$ has the cost 
$c_1$ and $K_2$ has the cost $c_2$ with $c_1<c_2$ then this implies that $K_1$ is better than $K_2$ which is not 
necessarily true or better, it holds only for some particular $K$'s that are close enough to the ideal case. It might 
happen that the cost function is not even continuous let alone being smooth e.g. the transparency is only achieved by 
the global minimizer $K^*$ and not with any other $K$. On the other hand, we don't have a metric for how much we need 
to get close to the ideal matrix. Let us first quote three very important questions posed from the seminal paper of Lawrence;
\begin{displayquote}[{\cite{lawrence}}][.]
In practice, perfectly transparent teleoperation will not be possible. So it makes sense to ask the following questions:
\begin{itemize}
	\item What degree of transparency is necessary to accomplish a given set of teleoperation tasks? 
	\item What degree of transparency is possible?
	\item What are suitable teleoperator architectures and control laws for achieving necessary or optimal transparency?
\end{itemize}
We focus on the second two questions in this paper. Instead of evaluating the performance of a specific teleoperation architecture,
as in [2], we seek to understand the fundamental limits of performance and design trade-offs of bilateral teleoperation in
general, without the constraints of a preconceived architecture
\end{displayquote}
In \cite{lawrence}, Lawrence then invokes the passivity assumption and then passivity theorem to arrive at structural 
properties of the controller $K$. Evidently, this allows for the back-substitution of the controller entries and solution
for the ideal case. Then the resulting controller is denoted with \enquote{\emph{Transparency Optimized Controller}}. In 
control theoretical terminology, this amounts to a cross-coupling control action where the bilateral dynamical differences
are canceled out and then single control channels are tuned to maximum performance bound to the stability constraints. It 
should be clear that this is the transparency definition from a passivity-based point of view and we refrain from iterating 
what is given in the previous chapter. 

Even it is so, we have to emphasize that we have not touched the most important question, that is the first of the three, 
rather we hope to achieve the required transparency levels just enough to fool the user. After two decades, this point is 
in our opinion simply discarded and many studies in the literature somewhat treats the conclusions of Lawrence in a different 
context than what has been given by Lawrence. As is for the case for the Hogan's paper on passivity, Lawrence never claims 
that this is a definite performance measure. Instead he clearly shows the implications that follow from such assumptions. 


Moreover, there are interesting studies inline with our claims about irrelevance of the remote media recreation in bilateral
teleoperation problem. For example, \cite{kilchenman,wildenbeest} and a few other studies report that there is a saturation
effect on how much realism that can be projected to the user. In other words, there is an inherent bandwidth limitation for 
the realism increase such that beyond a certain band of frequency, the transparency does not increase significantly. Moreover, 
in the case of shared control applications, it might happen that transparency is not needed at all. 



\subsection{\texorpdfstring{$Z$}{Z}-width}

In \cite{colgate4}, the performance of a haptic device is related to the dynamic range of impedances (hence the name $Z$) that
the device can display to the user. In this context we have two extremes; on one hand we have purely the local device impedance 
for the free-air motion and on the other hand we have the maximally stiff local device for the rigid and immobile obstacle collision. 
Let $Z_f$, $Z_c$ denote these two distinct cases then the more pronounced the difference between these impedances, the more 
capable the teleoperation system can reflect various impedances inbetween. Thus, we implicitly assume that the rigid contact 
case and the free-air case are the extreme points of the uncertainty set and testing for these two cases are sufficient to 
conclude that any impedance on the path from $Z_f$ to $Z_c$ is a valid impedance that can be displayed by the device. This
in turn implies that there is an ordering in the uncertainty set from \enquote{big} to \enquote{small} etc. and moreover 
the destabilizing uncertainty is at the boundary of the set such that these two extreme cases can vouch for stability over 
the whole possible environments.

Similar to what Lawrence has given, the authors also include a clear statement of purpose: 

\begin{displayquote}[{\cite{colgate4}}][.]
This paper will not address the psychophysics of what
makes a virtual wall \enquote{feel good} except to say that one
important factor seems to be dynamic range. An excellent
article on this topic has recently been written by
Rosenberg and Adelstein [11]\footnote{Reference \cite{rosenberg} of this thesis.}. 
We will present instead
some of our findings, both theoretical and experimental,
concerning achievable dynamic range. In short, we will
address the question of how to build a haptic interface
capable of exhibiting a wide range of mechanical
impedances while preserving a robust stability property
\end{displayquote}


Under these assumptions, via defining a functional to measure the distance between $Z_f$ and $Z_c$, we can assess the performance
of different bilateral teleoperation devices. In \cite{goranthesis}, $Z$-width is defined as 
\begin{equation}
Z_{\text{width}} = \int_{\omega_0}^{\omega_1}{\abs{\log|Z_{\vphantom{f}c}(\iw)|-\log |Z_f(\iw)|}}d\omega
\label{eq:zwidth}
\end{equation}
or alternatively, a simulation/experiment-based method can be utilized as in \cite{weir}. 


Note that this expression does not appear in \cite{colgate4} but in \cite{goranthesis,passenberg}. In both \cite{colgate4,
goranthesis} no additional information is provided except some general rules of thumb about device damping and other
related issues. In general, the inherent local device damping reduces the gain of $Z_f$ or the maximum stiffness displayed by the 
device increases the gain of $Z_c$ and hence the $Z_{\text{width}}$ increases.

It should be noted that the differences at each frequency are lumped into one scalar number and moreover, the impedance gain curves can
cross each other (see \cite{goranthesis}) and might lead to an overly optimistic result. Similarly, resonance peaks and zeros of the
involved impedances can be smeared out if we solely rely on this functional. 

Since $Z_f$ and $Z_c$ are functions of the environment impedance, these curves can be obtained for one particular environment
at a time. This also holds for the derivation of \cite{lawrence}. In \cite{goranthesis}, the difference  is evaluated for more 
than one environment and then averaged out i.e. let $Z_{act}(Z_e)$ be the impedance displayed to the user in order to render 
$Z_e$ on the local site. Then, for a particular controller, average $Z$-error to each candidate $Z_e$ is given by
\begin{equation}
Z_{avgerr} = \frac{1}{n}\sum_{i=1}^n{\left[
    \frac{1}{\omega_{1i}-\omega_{0i}}\int_{\omega_0}^{\omega_1}{%
                                     \abs{\log|(Z_{act}(Z_{ei}))(\iw)|-\log |Z_{ei}(\iw)|}}d\omega.
                                     \right]}
\label{eq:zdiff}
\end{equation}
This cost function is denoted by \enquote{\emph{Transparency Error}} or \enquote{\emph{Fidelity}}


\subsection{Fidelity}
In \cite{cavusoglu}, a variant of a transparency error is proposed to assess the performance. In this context, 
the emphasis is on the variation of the environment impedance and the resulting effect on the displayed impedance. 
Also the motivation is focused on the surgical procedures via bilateral teleoperation. If, for 
example, the remote device slides over some tissue that involves a tumor or any other irregularity that would be felt
if the same motion would have been performed directly by the surgeon, the better the nuances transmitted, the higher
the fidelity. This performance objective is in a sense enforces the high frequency content of the information (closer
to tactile bandwidth). It has been noted that the JND of \SIrange{14}{25}{\percent} for the relative compliance 
of distinct surfaces goes under \SI{1}{\percent} for fast compliance variation detection via scanning a surface
\cite{dhruvtendick}. Similar to the definitions given for transparency, the change of the displayed impedance 
$Z_{disp}(Z_e)$ with respect to the change in the environment $Z_e$ can obtained via a straightforward calculation.


Given the scalar complex LTI uncertainty block $\Delta$ and the LTI plant $G\in\mathcal{RH}_\infty^{2\times 2}$, 
such that 
\[
\pmatr{q\\z} = G\pmatr{p\\w},
\]
the LFT interconnection of $\Delta-G$ is given by, 
\[
P = G_{22}+G_{21}\Delta\inv{(I-G_{11}\Delta)}G_{12}
\]
Here $P$ denotes the impedance seen by the operator, $G$ denotes the teleoperation system and $\Delta$ being the 
environment impedance. Now define the derivative operation with respect to change in $\Delta$
\[
\frac{d}{d\Delta} P = \frac{G_{21}G_{12}}{(I-G_{11}\Delta)^2}.
\]
Then, though not pursued in \cite{cavusoglu}, this can, in turn, be rewritten as an LFT again;
\[
\pmatr{q_1\\q_2\\z} = \pmatr{2G_{11} &-G_{11} &1\\G_{11} &0 &0\\ 2G_{11}G_{12}G_{21} &-G_{11}G_{12}G_{21} &G_{12}G_{21}}\pmatr{p_1\\p_2\\w}
\]
and 
\[
\pmatr{w_1\\w_2} = \pmatr{\Delta &0\\0 &\Delta}\pmatr{z_1\\z_2}.
\]


This led to the authors defining a transparency-like performance objective using a rather subtle choice of system 
$2$-norms. Let $G$ be a stable LTI system transfer function $G(s)$. Then 
\[
\norm{G}_2^2 \coloneqq \infint{\trace(G^*(\iw)G(\iw))}d\omega
\] 
The measure of fidelity is defined as the norm
\[
\norm{\left.W_s\frac{dP}{d\Delta}\right|_{\Delta_{enom}}}_2
\]
where $W_s$ is a typically low-pass type weighting function to emphasize the frequency band of interest. Therefore
the synthesis problem is to find the optimizer, controller $K$ to the problem
\[
\sup_{\substack{\text{Stability}\\\text{Other Constraints}}}\inf_{\Delta_{ei}\in\bm{\Delta_e}}
\norm{\left.W_s\frac{dP}{d\Delta}\right|_{\Delta_{ei}}}_2
\]
where $\Delta_{ei}$ are the worst case environments that are of interest. How to detect such 



There are a few interpretations of this norm in the literature, mainly, the deterministic \enquote{area under the Bode
Plot} interpretation i.e. energy of the impulse response for scalar case, and the stochastic \enquote{steady-state 
white-noise-input response}. We are under the impression that the authors argue in the line of the former interpretation
with a similar reasoning given in the $Z$-width discussion via an area computation. 


Designing a robust controller while minimizing the $\mathcal{H}_2$ norm of an uncertain system in the face of a predefined 
uncertainty set i.e. \enquote{Robust $\mathcal{H}_2$ Synthesis} problem has already recevied a lot of attention and 
the results can be found in the literature, e.g., \cite{dullerud}. Hence, the problem definition in \cite{cavusoglu} is 
in fact tractable which might be inline with the intuition reported about the convexity of the performance objective. However, 
it's not clear to us why we choose the system $2$-norm for the performance cost. Additionally, the infimum needs to be 
computed in the face of a set at infinitely many points hence an appropriate relaxation is required. This point is also not 
given though a gridding approach might have been utilized in the numerical optimization procedure described in the paper.

It is also not clear for which case we should utilize this performance objective. The initial difficulty is that all the 
involved operators are LTI hence there is no time variation involved. The test reads as; we select an arbitrary element 
in the predefined uncertainty set, say $\Delta_e$, and evaluate the derivative at $\Delta_e$. Hence, in some $\epsilon$-
neighborhood of $\Delta_e\in\bm{\Delta_e}$ we can see the change in $P$. Thus, if the environment slightly off from our 
nominal guess of the environment, this tells us how much fidelity measure would change. But the environment is still 
assumed to be LTI.


Note that, this does not imply that time-variations are taken into account. Suppose a particular admissible trajectory
$\hat{\Delta}_e(t)$ in time is given such that $\hat{\Delta}_e(t_1)=\Delta_e$ i.e. its time-frozen LTI copy coincides 
with the particular nominal environment model $\Delta_e$ and at some time instant $t_2$, it  coincides with another 
LTI model $\mathring\Delta_e$ that is within some $\epsilon$-neighborhood of $\Delta_e$. Even if we achieve very good 
fidelity properties evaluated at each $\Delta_e$ and a sequence of LTI model elements each being in the small neighborhood 
of the other, this does not guarantee that we would have good fidelity for the trajectory $\hat{\Delta}_e$.


\section{Closing Remarks and Discussion}
There are a few other performance criteria reported in the literature. Consider the definition of the impedance seen by 
the operator $P$ above. In \cite{katsura}, this term is divided into two individual terms, dnoted by 
\enquote{reproducibility} and \enquote{operationality}. The idea is similar to a sensitivity/complementary sensitivity 
function definitions. 

In \cite{yokokohjiyoshikawa}, also an ideal responses is also partitioned into two parts and denoted by \enquote{index of 
maneuverability} and, in essence, is similar to what is given above, hence omitted. 

We refer to the survey papers \cite{hokayemspong,passenberg} for a general treatment and \cite{klomp,dennis} and references 
therein for a more detailed overview about many variations in the literature.


In summary, there is no general perfomance criteria that can lead to a dedicated control design procedure. The aforementioned 
performance objectives always start from the direct manipulation case and assume a distance between the interacting parties.
Then the implications of such hypotheses are pursued and some results are obtained. It might very well happen that all or none 
of those conclusions are correct. Put better, these studies always try to remedy the distortion caused by splitting two 
interacting bodies. Thus, the problem becomes too ambitious at the outset. Similar to the delay problem there is not much
we can do about the distortion within the laws of physics. Moreover, as we have mentioned in the introduction, the problem is 
exclusively about human perception and not related to the reconstruction of the remote scene. As long as we can \enquote{fool}
the user for the sake of efficiency and operational comfort, we are done. 


Based on our control design attempts, it seems that all the performance objectives above and also many motivational sections 
in the papers describe a particular projection of a more general objective in terms of human perception. In \Cref{chap:application}, 
we give the necessary details but by emphasizing one performance objective to the other one can obtain different controllers
with varying perception for the user. 





%
%
%
%
%
%%\subsection{Matrix case: $\mathcal{F}(\Delta\star G) - \mathcal{F}(\Delta_0\star G) - L(H,\Delta_0)$}
%\[
%L(\Delta_0,H) = G_{21}\inv{(I-\Delta_0G_{11})}H\inv{(I-G_{11}\Delta_0)}G_{12}
%\]
%which, in turn, can be rewritten as
%\begin{align*}\pmatr{z_4\\z_5\\z_6\\z_L}&= \pmatr{G_{11}&0&0&G_{12}\\0&G_{11}&G_{11}&0\\G_{11}&0&0&G_{12}\\0&G_{21}&G_{21}&0}
%\pmatr{w_4\\w_5\\w_6\\w_L} \\
%\pmatr{w_4\\w_5\\w_6} &= \pmatr{\Delta_0&&\\&\Delta_0&\\&&H}\pmatr{z_4\\z_5\\z_6}
%\end{align*}
%We have also, 
%\[
%\mathcal{F}(\Delta\star G)  = \mathcal{F}((\Delta_0+H)\star G)
%\]
%with the LFT representation,
%\begin{align*}
%\pmatr{z_1\\z_2\\z_{\Delta}}&= \pmatr{G_{11}&G_{11}&G_{12}\\G_{11}&G_{11}&G_{12}\\G_{21}&G_{21}&G_{22}}
%\pmatr{w_1\\w_2\\w_{\Delta}} \\
%\pmatr{w_1\\w_2} &= \pmatr{\Delta_0&\\&H}\pmatr{z_1\\z_2}
%\end{align*}
%and obviously, 
%\begin{align*}
%\pmatr{z_3\\z_{\Delta_0}}&= \pmatr{G_{11}&G_{12}\\G_{21}&G_{22}}
%\pmatr{w_3\\w_{\Delta_0}} \\
%w_3 &= \Delta_0 z_3\end{align*}
%
%We have, 
%\[
%w_{\Delta_0} = w_\Delta = w_L
%\]
%Then, the derivative can be written with $e = z_\Delta - z_{\Delta_0} - z_L$
%\[
%\pmatr{z_1\\z_2\\z_3\\z_4\\z_5\\z_6\\e} = 
%\pmatr{G_{11}&G_{11}&0&0&0&0&G_{12}\\G_{11}&G_{11}&0&0&0&0&G_{12}\\&&G_{11}&&&&G_{12}\\&&&G_{11}&0&0&G_{12}\\&&&0&G_{11}&G_{11}&0\\}
%\pmatr{w_1\\w_2\\w_3\\w_4\\w_5\\w_6\\w}
%\]






\chapter{Introduction}
\label{chap:intro}

The success of the technological advances often can be associated with an unprecedented convenience that they bring in. At the heart of this convenience lies the ability to relax the limitations of the human body to a certain extent. From this point of view, it is not a surprise that the three most prominent technological wonders of the last century, namely the \emph{Television}, the \emph{Telephone} and the \emph{radio} (which was originally called ``radiotelegraphy"), bears the same Greek prefix \emph{tele-} which corresponds to ``\emph{at a distance}" in our context. This shows that there is something of extreme importance about our drive to extend our capabilities beyond the constraints that our bodies impose.

% We can even think of slightly utopian cases such as teleportation which is something we are not sure if it is possible at all. 

It is quite remarkable, in retrospect, that these ``gadgets" did not perish but rather kept on evolving since initially they were far from perfect. Quite the contrary, they were hardly operational. Even the commercialized version of the early TVs had a narrow bandwidth and minimum image quality. Similarly radio and telephone was barely transmitting sensible information as far as the signal-to-noise ratio is concerned. Nevertheless, they have provided the ways of communication which were unimaginable before their time. Therefore the added value dominated the shortcomings and even though they were quite imperfect, we kept using them. The important lesson to be learned was that a technology should not be judged by its imperfections but rather should be weighed by its contribution in this context or the convenience that is brought in by using it.  

The success is also related to the fact that these technologies mainly relied on the human brain itself at their early stages. For example, the human brain did most of the noise filtering and data recovery by just guessing the missing pieces and identifying patterns from the signal brought by the respective medium. Today, with our smart mobile phones and 3D LED TVs, we can assume that the computational load on the human brain is drastically reduced. In other words, we are still identifying patterns and utilizing the relevant parts of our brain to make sense out of a TV broadcast\footnote{Pun intended.}. However, we don't need to use a higher level of concentration to reconstruct the words that we hear or to identify the image on the display due to the high quality output.

It seems that we are on the same track with the technological developments involving our touch sense. Considering the importance of our touch sense in any given situation, the added value of extending of our perception in this modality needs no motivation. Take the most familiar example: the vibrating mobile phone in the silent mode in our pocket. This is a very important example since every individual learns what that vibration might mean, either an SMS or a call, depending on the vibrational pattern. This means that the touch sense can be used to convey messages and more importantly we can process those messages for inference. 


This type of information said to be received via the haptic channel (or the collaborative use of tactile and proprioceptive modalities). Note that, we use the term ``touch sense'' pretty vaguely as a shortcut and we leave it to the experts of the field to define the sophisthicated mechanisms (pertaining to the somatosensory system) that we utilize when we manipulate objects, say with our bare hands. 


Since our skin and muscles form one of most sophisticated and complex sensory systems, the somatosensory system, the brain can easily interpret the slightest changes and this extra signal processing power gives us a chance to hack into this system by providing artificial inputs. Still, it is rather conspicuous that this is impossible to achieve with today's technology. The essential complication is twofold; the high sensitivity of the very same sensory system makes it difficult to fake or mimic a natural phenomenon by artificial means and on the other hand we don't have a well-defined mapping from the to-be-created sensation to the required excitation signals. Moreover, even if we have such mappings available, the related hardware must execute the computed haptic signal profiles perfectly which is generally not the case. 

Then, we could simply ask \emph{Why bother?} 


\section{The Objectives}

We first give an opiniated view about the objectives of the technology and later on, define our microscopic focus of this thesis in this vast generality. This would hopefully give some perspective to what follows in the analysis and synthesis sections.  

\subsection{Haptic Teleoperation}
The touch related applications are diverse. Not only in terms of sensation they are related to but also how they encode the signals and 

\subsection{Objective of This Thesis}





\section{Random Text Population}
\kant[3-4]

\subsection{O Ye Man!}

\kant[3-5]
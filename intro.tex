\chapter{Introduction}
\label{chap:intro}

The success of the technological advances often can be associated with an unprecedented convenience that they bring in. At the heart of this convenience lies the ability to relax the limitations of the human body to a certain extent. From this point of view, it is of no surprise that the two prominent technological wonders of the last century, namely \emph{Television} and \emph{Telephone}, bears the same Greek prefix \emph{tele-} which corresponds to ``\emph{at a distance}" in our context. Another crucial technology, seemingly absent of the prefix, was the radio which was originally called radiotelegraphy. This allows us to speculate that there is something of extreme importance about our drive to extend our sensing capabilities beyond the constraints that our bodies impose. 


It is quite remarkable, in retrospect, that these ``gadgets" did not perish but rather kept on evolving since initially they were far from perfect. Quite the contrary, they were hardly operational. The TV was black and white and had terrible image quality. Similarly radio and telephone was barely transmitting sensible information as far as the signal-to-noise ratio is concerned. Nevertheless, they have provided fantastics ways of communication which were unimaginable before their time. Therefore the added value dominated the technical problems and even though they were quite imperfect, we kept using them. The lesson to be learned from this is that one should never judge a technology by its imperfections but rather weigh the added value or the convenience that is brought in by using it. 

Until recently, most of the required computational load e.g. signal processing, image recognition, noise estimation etc. relied on the human brain itself. For example, the human brain did most of the disturbance filtering and data recovery by just guessing missing pieces and identifying patterns from the signal brought via the respective medium.  



While teleportation is out of sight for now, we are at the brink of a new breakthrough extending our touch\footnote{By ``touch", we mean a large, imprecisely defined subset of the somatosensory system} sense at a distance. Although its history goes almost as far as the vision systems go, the touch technology remained mostly obscure until 1980's. 




Since our skin and muscles form one of most sophisticated and complex sensory systems, the somatosensory system, the brain can easily interpret the slightest changes and this extra signal processing power gives us a chance to hack into this system by providing artificial inputs. 



\section{Random Text Population}
\kant[3-14]
\section{adsf}

\kant[3-5]



\subsection{O Ye Man!}

\kant[3-5]
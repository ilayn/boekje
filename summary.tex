\chapter{Summary}

The human body is equipped with one of the most complex network structure involving a huge number of components for the sense of touch. 
The human brain, then, processes this massive amount of information to infer the nature of the interaction with its surroundings. At a 
higher level, this can be understood better via the engineering concept \enquote{sensor fusion} which studies the ways of utilizing qualitatively 
different information coming from distinct sources. Similar to 3D vision and audio system technologies, it is also possible to create an 
artificial sense of touch using dedicated devices and predefined motion patterns. Motivated by these inference capabilities of the brain, 
the haptic feedback technology aims to provide touch cues for the human user while operating a remote device via using a local device 
where there is no actual physical interaction between the user and the remote medium. Hence, there has been an active line of research 
since the 1960s to realize such devices and control algorithms that provide the functionality of providing artificial touch sense for the 
human to immerse into a virtual reality. Still, the technology is far from being established and in fact even the potential benefit of 
this technology is a subject of debate. This is partially due to the fact that the important technical problems mainly remain unsolved. 

Current state-of-art methods attack the haptic feedback problem as the problem of directly copying and matching the interaction at the 
local and the remote devices in terms of the involved forces and positions. Since the physics of each media can be significantly 
different, there is an inherent stability problem i.e., the control algorithm in fact tries to achieve a physically impossible 
configuration of device positions and applied forces simultaneously. The common remedy to remove this instability is the injection of 
damping such that both the local and remote devices can only exhibit severely limited motion profiles in terms of the maximum force and 
velocities.

This thesis investigates the fundamental reasons why such stability problems arise. The approach for the investigation is to start from 
the initial assumptions and hypotheses and work out the implications to see whether the inferences indeed match the literature. We argue 
that the stability problems that we are often faced with, are not fundamentally essential. But in fact they are due to the overambitious 
control objectives and questionable assumptions about the components involved in the interaction described above. Moreover, the 
performance objectives that are assumed to be relevant turn out to be either impossible to obtain or irrelevant to the problem at hand. 

In the second part of the thesis, using the so-called Integral Quadratic Constraints framework, developed by many contributors but 
refined and popularized by Megretski and Rantzer in the end of 1990s, we show that many of the seemingly different stability methods in 
the frequency domain are in fact equivalent. This also allows us to show that often the results in the literature are incorrectly 
formulated and equivalent statements are compared with each other. 

After this unification, we apply the general multiplier methods of IQCs to demonstrate the convenience of the stability analysis for 
different types of uncertainties involved for which no classical test is available. Then, using the analysis framework we turn to the 
controller design problem and derive an iterative control design method which can be seen as a generalized version of  the well-known D-K 
iteration from the Structured Singular Value theory. It is shown that we can bypass the complications of the state-of-art control design 
techniques by clearly defining a performance objective that is posed as a minimization problem. The validity of the method is shown by 
the performing teleoperation over two 1-degree of freedom devices. 

Although the results show superiority in terms of performance, we still insist on arguing that the haptic feedback problem is not well 
understood by the control engineering community. Thus, we strongly encourage the experts of the relevant fields such as the neurosciences, 
cognitive psychophysiology and others that we are not yet aware of to join and guide the engineers in creating the performance 
specifications for a successful and beneficial haptic technology.
%
%\chapter{Samenvatting}
%
%The human body is equipped with one of the most complex network structure involving a huge number of components for the sense of touch. 
%The human brain, then, processes this massive amount of information to infer the nature of the interaction with its surroundings. At a 
%higher level, this can be understood better via the engineering concept \enquote{sensor fusion} which studies the ways of utilizing qualitatively 
%different information coming from distinct sources. Similar to 3D vision and audio system technologies, it is also possible to create an 
%artificial sense of touch using dedicated devices and predefined motion patterns. Motivated by these inference capabilities of the brain, 
%the haptic feedback technology aims to provide touch cues for the human user while operating a remote device via using a local device 
%where there is no actual physical interaction between the user and the remote medium. Hence, there has been an active line of research 
%since the 1960s to realize such devices and control algorithms that provide the functionality of providing artificial touch sense for the 
%human to immerse into a virtual reality. Still, the technology is far from being established and in fact even the potential benefit of 
%this technology is a subject of debate. This is partially due to the fact that the important technical problems mainly remain unsolved. 
%
%Current state-of-art methods attack the haptic feedback problem as the problem of directly copying and matching the interaction at the 
%local and the remote devices in terms of the involved forces and positions. Since the physics of each media can be significantly 
%different, there is an inherent stability problem i.e., the control algorithm in fact tries to achieve a physically impossible 
%configuration of device positions and applied forces simultaneously. The common remedy to remove this instability is the injection of 
%damping such that both the local and remote devices can only exhibit severely limited motion profiles in terms of the maximum force and 
%velocities.
%
%This thesis investigates the fundamental reasons why such stability problems arise. The approach for the investigation is to start from 
%the initial assumptions and hypotheses and work out the implications to see whether the inferences indeed match the literature. We argue 
%that the stability problems that we are often faced with, are not fundamentally essential. But in fact they are due to the overambitious 
%control objectives and questionable assumptions about the components involved in the interaction described above. Moreover, the 
%performance objectives that are assumed to be relevant turn out to be either impossible to obtain or irrelevant to the problem at hand. 
%
%In the second part of the thesis, using the so-called Integral Quadratic Constraints framework, developed by many contributors but 
%refined and popularized by Megretski and Rantzer in the end of 1990s, we show that many of the seemingly different stability methods in 
%the frequency domain are in fact equivalent. This also allows us to show that often the results in the literature are incorrectly 
%formulated and equivalent statements are compared with each other. 
%
%After this unification, we apply the general multiplier methods of IQCs to demonstrate the convenience of the stability analysis for 
%different types of uncertainties involved for which no classical test is available. Then, using the analysis framework we turn to the 
%controller design problem and derive an iterative control design method which can be seen as a generalized version of  the well-known D-K 
%iteration from the Structured Singular Value theory. It is shown that we can bypass the complications of the state-of-art control design 
%techniques by clearly defining a performance objective that is posed as a minimization problem. The validity of the method is shown by 
%the performing teleoperation over two 1-degree of freedom devices. 
%
%Although the results show superiority in terms of performance, we still insist on arguing that the haptic feedback problem is not well 
%understood by the control engineering community. Thus, we strongly encourage the experts of the relevant fields such as the neuro-
%sciences, cognitive psychophysiology and others that we are not yet aware of to join and guide the engineers in creating the performance 
%specifications for a successful and beneficial haptic technology.

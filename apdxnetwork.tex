\chapter{Network Theory Primer}
\label{chap:apdxnetwork}


We present the often encountered concepts in the bilateral teleoperation studies in a rather abstract fashion. Our aim is to
prepare the basis for the main arguments in the body of this thesis. Hence, the experts of the field can safely skip this appendix. 
The aim is to run down the basic stability related concepts, hence many circuit theory preliminaries are omitted.

\section{Terminology}
When dealing with electrical circuits, often the engineering practice is to neglect any changes to the signal traveling through 
a wire from location $A$ to $B$ if the distortion from various sources
%\begin{itemize}
	%\item the length of the path between $A$ and $B$
    %\item the conduction properties of the medium
    %\item the energy disspation along the path
    %\item the wave propagation speed compared to the speed of light
%\end{itemize}
is negligible. This allows us to use the abstraction in block diagrams and calculations that point
$A$ and $B$ have the same measurable characteristics through time. This is typically shown with a 
straight line connecting $A$ and $B$. The interpretation is that $A$ and $B$ share the same variables of 
interest on that line. Therefore, $A$ and $B$ are said to be interconnected terminals connected by a wire. 
Clearly creating terminals are as simple as cutting the connection arbitrarily on a path. Often, we define 
hypothetical terminals as if there were distinct points on the circuit and we have connected them artificially. 

Hence, following Jan C. Willems' formulation for a systematic definition of terminals and ports in \cite{willemsCSM}; 
an electrical circuit is a device, a black box, with wires so-called terminals
through which the circuit can interact with its surroundings. In electrical circuits, the interaction takes place
via the electrical voltage drop across the terminals and the electrical current that flows through the black box.
Therefore each terminal admits two real variables attached to it, the potential and the current. Conventionally, 
the current is denoted with a positive number when it flows into the circuit. Thus, an interconnection is connecting
a wire from terminal $A$ of the black box $(1)$ to $B$ of $(2)$ enforcing the following to hold
\[
V_A = V_B \quad \text{and} \quad I_A=-I_B
\]
Then, we can look at the resulting interconnected system (3) as (1) and (2) combined and exhibiting the same
phenomenon at their terminals $A$ and $B$. 

The collection of physically attainable phenomena are abstracted with the notion of the behavior set (\cite{behavbook}):
Consider a circuit with N terminals. Let $\mathcal{B}\subseteq (\Real^N\times\Real^N)^\Real$ denote the behavior set 
that is defined as the set of all admissible potential and current trajectories compatible with the network architecture 
at each terminal. Here $(\Real^N\times\Real^N)^\Real$ denotes 
the set of all maps $f: \Real\to\Real^N\times\Real^N$ e.g. each terminal voltage and current evolution through time 
and $\mathcal{B}$ is the restriction to the maps that are compatible with the network structure. Roughly speaking, behavior
set excludes all the trajectories that are physically impossible to attain by the black box. Thus, when it is said that
a particular trajectory is in the behaviour set i.e., 
\[
(V_1,\ldots,V_N,I_1,\ldots,I_N)\coloneqq (V,I) \in\mathcal{B}
\]
it implies that there exists an initial condition $(V(0),I(0))$ such that $(V,I)$ is a an admissible trajectory through time
(with a particular external excitation sequence if present).


Assuming a conservative magnetic field, a circuit obeys the well-known Kirchhoff voltage and current laws compactly described 
as 
\begin{align}
(V,I)\in\mathcal{B} &\implies (V+\alpha\mathbb{1},I)\in\mathcal{B} \quad \forall \alpha\in\Real^\Real \tag{KVL}\\
(V,I)\in\mathcal{B} &\implies \sum_{k=1}^N{I_k}=\mathbb{1}^TI=0 \tag{KCL}
\end{align}
where $\mathbb{1}$ denotes a vector with all entries are equal to $1$ whose size is clear from the context.

Let, $P \subseteq \{1,\ldots,N\}$ denote an $m-$tuple selection of indices out of $N$ terminals of a circuit. Then, terminals
$P_i$ for all $i\in P$ are said to form a port if 
\[
(V,I)\in\mathcal{B} \implies p^TI=0 \tag{Port KCL}
\]
where $p$ is a vector with $k-$th entry being $1$ if $k\in P$ and $0$ otherwise. This is nothing but a reformulation of the well-known 
port condition from circuit theory. Thus, we can also define a port as the set of terminals that satisfy port KCL. Given a port with 
$n-$terminals with $V,I$ denoting the through and across variables, the instantaneous power is given by 
\[
P = \sum_{k=1}^n{V_k(t)I_k(t)}
\]
and the energy transfered in the time interval is given by the total power delivered
to/from that port in the time interval $[t_1,t_2]$: 
\[
E = \int^{t_{2}}_{t_{1}}\sum_{k=1}^n{V_k(t)I_k(t)}dt
\]
These formulas hold only if the terminals form a port and a port can have arbitrary number of terminals 
e.g., op-amps, transistors, $Y-\Delta$ resistance networks are examples for three terminal ports. 


%
%Using block diagrams, bond, or signal-flow graphs etc. this basic assumption is always inherent. In the heart of 
%feedback control theory also lies this assumption as shown in \Cref{fig:apdx:fct}. The arrows implicitly shows that
%the information at the source is equivalent to the one at the destination. Obviously, it is trivially true for 
%digitally encoded information sent over a network in terms of the encoded information. However our focus is on the 
%technology that allows that information to reach to its destination. 
%
%Hence we can view the closed loop system shown in \Cref{fig:apdx:fct} as a black box with two wires sticking out 
%from this box. 
%
%
%\begin{figure}%
%\centering
%\begin{tikzpicture}[
%auto, 
%node distance=2cm,
%>=latex',
%block/.style={draw, fill=blue!20, rectangle, minimum height=2em, minimum width=3em},
%sum/.style={draw, fill=blue!20, circle, node distance=1cm}
%]
    %\coordinate (input);
    %\node [sum, right of=input] (sum) {};
    %\node [block, right of=sum] (controller) {K};
    %\draw[<-] node[block, right of=controller, append after command={(system.north)--++(0,2em)}, node distance=3cm] (system) {P};
    %\draw [->] (controller) -- node[name=u] {$u$} (system);
    %\coordinate[right of=system] (output) {};
    %\node [block, below of=u] (measurements) {S};
%
    %% Once the nodes are placed, connecting them is easy. 
    %\draw [draw,->] (input) -- node {$r$} (sum);
    %\draw [->] (sum) -- node {$e$} (controller);
    %\draw [->] (system) -- node [name=y] {$y$}(output);
    %\draw [->] (y) |- (measurements);
    %\draw [->] (measurements) -| node[pos=0.99] {$-$} node [near end] {$y_m$} (sum);
%\end{tikzpicture}
%\caption{Basic feedback interconnection}%
%\label{fig:apdx:fct}%
%\end{figure}



%
%
%In modeling language and block diagrams, the terminals denote the information carriage
%of certain properties of the system. Hence, the reader is warned about the ambiguity of this hidden 
%switch from a purely electrical view of the networks to a more general and symbolic nature of these
%logical terminals. For example a spring can be modeled as a black box with two terminals, for which each terminal 
%having two variables attached to it, namely the force being exerted ``to'' or ``by'' and the position of 
%that terminal. Hence, when interconnected with other terminals, these variables are shared by both connected terminals.
%This is the basic reformulation of the port condition from network theory, namely a pair of terminals form a port 
%if the magnitude of current flowing in from one is equal to the magnitude of the current flowing out from the other.
%
%As almost always stability is related to one way or the other to a finite-energy notion, the energy transfer through ports
%is the main focus of network theory based modeling paradigm.
%
%This becomes crucial when modeling multi-local multi-remote teleoperation devices
%as one needs to keep track of the energy distribution between different users or different remote devices in order to 
%keep the network modeling paradigm consistent with the physical system (see e.g. \cite{panzirsch}).
%
%A $n-$port with effort $F_i$ and flow $V_i$ variables (positive if into the system) associated 
%with the $i$\textsuperscript{th} port is passive if 
%
%\[
%\int^t_0 F^T(\tau)V(\tau)d\tau \geq 0 \quad \forall t\geq 0.
%\]
%\begin{rem}There is no consensus in the literature on whether the above definition or the following is assumed for passivity
%\[
%\int^\infty_0 F^T(\tau)V(\tau)d\tau \geq 0
%\]
%However, it's an important difference as the former does not allow for temporary activity while the latter does. One can also extend
%this definition to incremental passivity such that instead of the integrated versions, the power condition is evaluated at each time 
%instance $t>0$. The former condition often appears in the nonlinear context and the latter is generally used in the linear frequency 
%domain counterparts. In the IQC framework, this distinction is made via \emph{Hard IQCs} vs. \emph{Soft IQCs} however the details 
%are omitted. Thus, it makes no difference to give 
%\end{rem}

Using a mechanical-electrical analogy, the mechanical teleoperation devices are converted to a network of ports. In network theory 
applications to bilateral teleoperation, the ``system" refers to the network model that is hypothetically disconnected 
(thus admitting virtual terminals) from its ``surroundings" such as the ``load" and the ``source" of a circuit. This system is 
allowed to interact with its surroundings via ``ports". In our context, load refers to the environment that is to be explored and 
the source is the human exploring the environment from a distance via the teleoperation system.

\begin{rem}
We have to note that, in this formulation, a single mass can not be interpreted as a port: The mass does not satisfy the 
port KCL unless it is thought to be applying an opposite force to a fixed inertial frame at a distance. This is the underlying
reason for the electrical analog of a mass (in the force-current context) is required to be a \emph{grounded} capacitor 
(see \cite{smith} for the development of the exact mechanical analogue of a capacitor, \emph{the inerter} which is succesfully implemented by 
McLaren Mercedes and Renault F1 teams and being used succesfully since 2005\footnote{According to the official statements.}). We will not 
go into the details and and simply refer to \cite{willemsCSM} for a thorough analysis of the pitfalls and quite nonintuitive power/energy 
results. Though, the inappropriateness of such view is brought to attention in numerous studies by Jan C. Willems and his 
colleagues, we are obliged to use the input/output formulation as we have no results regarding the behavioral approach to bilateral 
teleoperation systems (yet). However, we remind the reader that the potential of a behavioral modeling of teleoperation systems
is just unavoidable. 
\end{rem}


Since every two-terminal port can be characterized by two variables ( ``through'' and ``across'' quantities), it is possible to 
characterize the interconnected $n-$port networks as if one quantity is due to the other. This is done by imposing an artificial 
causality scheme, two of these time-dependent trajectories can be selected as free variables and the remaining ones become dependent variables 
(\cite{behavbook}). This is the simplest kind of input/output view of physical systems via treating one port variable as the 
\emph{cause} and the other one as the \emph{effect} to this cause e.g. the current is due to the voltage drop across the terminals or 
vice versa. 

Depending on the choice of the free variables, the system can be expressed in terms of impedance, admittance 
and hybrid parameters for two-port networks and their combinations for general $n$-port network interconnections. 
In the cases where the two-port is LTI, with a slight abuse of notation, we will use the term ``immitance" matrix to refer to 
any of these representations. Suppose that a two-port immitance matrix is partitioned as
\[
\pmatr{q\\y}=\pmatr{G_1 &G_2\\G_3 &G_4}\pmatr{p\\u}
\]
where $q,y,p,u$ represent the flow (current, velocity etc.) and the effort (potential difference, force etc.) signals. Then, 
obtaining one representation from another is possible by a combination of the following elementary ``permutation" and 
``{partial} inversion" operations:
\begin{align*}
\pmatr{q\\y}&=\pmatr{G_2 &G_1\\G_4 &G_3}\pmatr{u\\p}, \ \ \ \tag{Permutation}\\
\pmatr{p\\y}&=\pmatr{\inv{G_1} &-\inv{G_1}G_2\\G_3\inv{G_1} &G_4-G_3\inv{G_1}G_2}\pmatr{q\\u}. \ \ \ \tag{Partial Inversion}
\end{align*}
In the latter operation it is assumed that the inverse exists. The existence of inverses are limiting the realizability
of networks as impedance or admittance matrices. Moreover, in \cite{andersonHmat}, it has been shown that a hybrid matrix
realization is always possible regardless. This is yet another artifact of input/output formulation. 



For our purposes, we consider only the immitance 
matrices that describe $G$ as an input-output mapping (as opposed to transmission or ABCD parameters) as {follows}:
\begin{equation}
\pmatr{q\\y}=\pmatr{G_1 &G_2\\G_3 &G_4}\pmatr{p\\u} \ , \ \pmatr{p\\u}=\pmatr{\Delta_s &0\\0&\Delta_l} \pmatr{q\\y}.
%\label{eq:pasgdel}
\end{equation}
Therefore, the overall interconnection can be {depicted} by the block diagram given in \Cref{fig:portrep}. 
In relation to teleoperation, the {blocks} $\Delta_s$ and $\Delta_l$ refer to the human and the unknown environment. 


\section{Passivity Theorem}
In this section passivity related important concepts are defined in a compact fashion. Nevertheless, the material given here 
is well-known and can be found in many classical sources such as \cite{hillmoylan77,desvid,vdschaftbook}. We chose to 
follow closely \cite{desvid} for the presentation style. 

Let $V$ be a linear space equipped with a scalar product, $\mathcal{T}$ be the index set of time and $F$ be a class of functions 
$x:\mathcal{T}\to V$.
\begin{define} A linear truncation operator $P_T$ is defined on $F$ as 
\begin{equation}
P_T(x)(t) = \begin{cases} x(t) &\text{for } t\leq T\\ 0 &\text{for } t>T\end{cases}
\label{eq:apdx:trunc}
\end{equation}
\end{define}

In most of the cases, $\mathcal{T}$ is the closed positive real line and $V=\Real^n$, hence the scalar product becomes the inner product 
defined on $\Real^n$. The subscript ${\cdot}_T$ will be used as a shorthand notation for $P_T(\cdot)$. We also define $\mathcal{H}$ 
and its extension $\mathcal{H}_e$ as 
\[
\mathcal{H} := \left\{ x\in F\mid \norm{x}^2=\langle x_T,x_T \rangle<\infty \right\}
\]
and
\[
\mathcal{H}_e := \left\{ x\in F\mid \forall T\in\mathcal{T}, \norm{x_t}^2=\langle x_T,x_T \rangle<\infty \right\}
\]



\begin{define}Let $H:\mathcal{H}_e\to\mathcal{H}_e$. $H$ is said to be passive if and only if there exists a constant 
$\beta$ such that
\[
\langle Hx,x \rangle_T \geq \beta\quad \forall x\in\mathcal{H}_e,\forall t\in\mathcal{T}
\]
If moreover, there exists positive real number $\delta$ such that
\[
\langle Hx,x \rangle_T \geq \delta\norm{x_T}^2+ \beta\quad \forall x\in\mathcal{H}_e,\forall t\in\mathcal{T}
\]
$H$ is said to be strictly passive. 
\end{define}
The scalar $\beta$ is used to model the initial offset for nonlinear systems and 
will be taken as zero since we would be focusing on linear systems exclusively. 

\begin{figure}%
\centering
\begin{tikzpicture}[>=stealth]
\node[draw,minimum size=1cm] (h2) {$H_2$};
\node[draw,minimum size=1cm, above=1cm of h2] (h1) {$H_1$};
\node[draw,circle,inner sep=1mm,right= 1cm of h2]   (a1) {};
\node[draw,circle,inner sep=1mm,left= 1cm of h1,label={[inner sep=0]-60:$-$}] (a2) {};
\draw[->] (h1) -| (a1) node[right,midway] {$y_1$};
\draw[->] (h2) -| (a2) node[left,midway] {$y_2$};
\draw[->] (a1) -- (h2) node[above,midway] {$e_2$};
\draw[->] (a2) -- (h1) node[above,midway] {$e_1$};
\draw[<-] (a1) -- +(1cm,0) node[right] {$u_2$};
\draw[<-] (a2) -- +(-1cm,0) node[left] {$u_1$};
\end{tikzpicture}
\caption{Negative feedback interconnection}%
\label{fig:apdx:passint}%
\end{figure}

\begin{define}For an LTI operator $H$, passivity is equivalent to the corresponding transfer function $H(s)$ being positive real:
\begin{align*}
{\Re{H(\iw)}\geq 0} &{\Longleftrightarrow
\hat{u}^*(\iw) \Re{H(\iw)}\hat{u}(\iw) \geq 0}\\
&{\Longleftrightarrow \int_{-\infty}^\infty{\hat{u}^*(\iw)\Re{H(\iw)} \hat{u}(\iw)d\omega} \geq 0}\\
&{\Longleftrightarrow \int_{-\infty}^\infty{{[H(\iw)\hat{u}(\iw)]^*}\hat{u}(\iw)d\omega} \geq 0}
\end{align*}
for all $\omega\in\Realext$ and for all $u\in\mathcal{L}_{2}^n$. Furthermore, if the condition
\[
\int_0^\infty{H(u)(\tau)^Tu(\tau)d\tau} \geq \delta\norm{u}_2^2 + \epsilon\norm{H(u)}_2^2
\]
is satisfied with $\delta > 0$, $\epsilon = 0$ (or $\delta = 0$, $\epsilon > 0$) then the operator is said to be Strictly Input 
(Output) Passive with level $\delta$ (or level $\epsilon$) respectively. 
\end{define}



Suppose we are given with two dynamical systems $H_1,H_2: \mathcal{H}_e\to\mathcal{H}_e$ for which the system structure is given in time-domain with 
\begin{align}
\dot{x} &= f(x,e)\\
y &= h(x,e)
\end{align}
for both systems where $f: \Real^n\times\Real^m\mapsto \Real^n$ is locally Lipschitz and $h: \Real^n\times\Real^m\mapsto \Real^p$
is a continous function satisfying $f(0,0)=0$, $h(0,0)=0$. 

\begin{define}[Well-posedness] Consider the system interconnection given in \Cref{fig:apdx:passint}. The interconnection is said to 
be well-posed if there exist unique solutions $e_1,e_2$ to the equations
\begin{align}\label{eq:apdx:passint1}
e_1 &= u_1-h_2(x_2,e_2)\\\label{eq:apdx:passint2}
e_2 &= u_2+h_1(x_1,e_1)
\end{align}
for all admissible $(x_1,x_2,u_1,u_2)$. In the LTI case, assume that the respective transfer functions $H_1,H_2\in\RH_\infty$. Then
the interconnection is well posed if $\inv{(I-H_1H_2)}$ is a proper transfer matrix.
\end{define}

Note that, in our context $u_1,u_2$ model the voluntary part of the human/en\-vi\-ron\-ment force input as mentioned in \Cref{chap:litsurvey}. 



\begin{thm}Consider the well-posed feedback interconnection shown in \Cref{fig:apdx:passint} and 
described by \Cref{eq:apdx:passint1,eq:apdx:passint2}. Assume that there exist constants $\gamma_1,\beta_1,
\delta_1,\beta'_1,\epsilon_2,\beta'_2$ such that the following conditions hold
\begin{align}
\norm{H_1x}_T &\leq \gamma_1\norm{x_T}+\beta_1\\
\langle x,H_1x\rangle_T &\geq \delta_1\norm{x_T}^2+\beta'_1\\
\langle H_2x,x\rangle_T &\geq \epsilon_2\norm{H_2x_T}^2+\beta'_2
\end{align}
for all $x\in\mathcal{H}_e$ and for all $T\in\mathcal{T}$. If 
\begin{equation}
\delta_1+\epsilon_2 >0
\label{eq:apdx:actpas}
\end{equation}
then, $u_1,u_2 \in \mathcal{H}$ imply that $e_1,e_2,y_1,y_2\in\mathcal{H}$ 
\end{thm}

A well-known absolute stability analysis result for the LTI network is due to Llewellyn \cite{llewellyn}. 
An explicit indication of the frequency dependence is omitted for notational convenience.
\begin{thm}[Llewellyn's Criteria]\label{thm:apdx:llw}
A two-port network $N$,  described by its transfer matrix
\[
N(\iw) = \pmatr{N_{11}(\iw) &N_{12}(\iw)\\N_{21}(\iw) &N_{22}(\iw)}
\]
and interconnected to passive LTI termination immitances as in \Cref{fig:portrep}, is stable if and only if
\begin{equation}R_{11} > 0\text{ or } R_{22} > 0,\label{eq:lit:llew2}\end{equation} and
\begin{equation}4\left(R_{11}R_{22}+X_{12}X_{21}\right)\left(R_{11}R_{22}-R_{12}R_{21}\right)-\left(R_{12}X_{21}-R_{21}X_{12}\right)^2 > 0\label{eq:lit:llew3}\end{equation} 
or
\begin{equation}2R_{11}R_{22}-\abs{N_{12}N_{21}}-\Re{N_{12}N_{21}} > 0\tag{\ref{eq:lit:llew3}$'$}\label{eq:lit:llew3p}\end{equation}
for all $\omega\in\Realext$, where $R_{ij}$ and $X_{ij}$ denote the real and imaginary parts of $N_{ij}$ respectively.
\end{thm}

Similarly, as is for Llewellyn's stability conditions, it is straightforward to derive 
unconditional stability {tests if} the network is represented by scattering parameters. 
In what follows, we denote transformed passive LTI uncertainties with $\tilde{\Delta}_{s}, 
\tilde{\Delta}_l$ which are unity gain bounded. The corresponding interconnection is 
supposed to be given by the loop equations
$q = S p$, $p=\tilde{\Delta} q$ i.e.
\begin{equation}
\underbracket[0.2pt]{\pmatr{q_1\\q_2}}_{q}=
\underbracket[0.2pt]{\pmatr{S_{11} &S_{12}\\S_{21} &S_{22}}}_{S}
\underbracket[0.2pt]{\pmatr{p_1\\p_2}}_{p} \ , \ \pmatr{p_1\\p_2}=
\underbracket[0.2pt]{\pmatr{\tilde{\Delta}_s&0\\0&\tilde{\Delta}_l}}_{\tilde{\Delta}} \pmatr{q_1\\q_2}.
\label{eq:rolic}
\end{equation}
Rollett's conditions (\cite{stern,rollett,kurokawa}) for stability are then
formulated as follows: 

\begin{thm} Consider the same network given in Llewellyn's criteria which is represented in scattering parameters.
The interconnection is stable if and only if the inequality 
\begin{equation}
K = \frac{1+\abs{\nabla}^2-\abs{S_{11}}^2-\abs{S_{22}}^2}{2\abs{S_{12}S_{21}}} > 1
%\label{eq:rollett}
\end{equation}
holds for all frequencies together with an auxiliary condition in terms of $\nabla = S_{11}S_{22}-
S_{12}S_{21}$. This extra condition can be stated in at least five different ways, such as 
\[ 
1-\abs{S_{11}}^2 > \abs{S_{12}S_{21}}\ \text{ or } \ (1-\abs{S_{22}}^2)> \abs{S_{12}S_{21}}. 
\] 
(See \cite{edsin} for further details). 
\end{thm}

In \cite{edsin}, Edwards and Sinsky reduced Rollet's conditions to a single quantity denoted as $\mu$ to be checked 
for being greater than unity for all $\omega\in\Realext$. 
 \begin{thm} Verifying Rollet's conditions is equivalent to verifying the single condition
\[
\mu\coloneqq \frac{1-\abs{S_{11}}^2}{\abs{S_{22}-S_{11}^*\Delta}+\abs{S_{12}S_{21}}} > 1\qquad \forall \omega\in\Realext.
\]
\end{thm}

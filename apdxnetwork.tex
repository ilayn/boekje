\chapter{Basic Network Theory}\label{chap:apdxnetwork}


Let us give a short introduction about the related terminology at this point for 
the sake of completeness. We would like to follow Jan C. Willems' formulation for a systematic definition of ports appeared 
in \cite{willemsCSM}. 

\section{A Recap of Network Theory Preliminaries}

In network theory and also in its application {to bilateral teleoperation}, the ``system" refers 
to the network model that is hypothetically disconnected (thus admitting virtual terminals) from 
its ``surroundings" such as the ``load" and the ``source" of a circuit. This system is allowed to 
interact with its surroundings via ``ports" or, more generally, circuit terminal pairings that 
satisfy some technical assumptions. 


\tikz \draw[->,ultra thick] (0,0) -- ++(4cm,0) node[right,fill=yellow] {(More info on behaviors below!)};


In modeling language and block diagrams, the terminals denote the information carriage
of certain properties of the system. Hence, the reader is warned about the ambiguity of this hidden 
switch from a purely electrical view of the networks to a more general and symbolic nature of these
logical terminals. For example a spring can be modeled as a black box with two terminals, for which each terminal 
having two variables attached to it, namely the force being exerted ``to'' or ``by'' and the position of 
that terminal. Hence, when interconnected with other terminals, these variables are shared by both connected terminals.
This is the basic reformulation of the port condition from network theory, namely a pair of terminals form a port 
if the magnitude of current flowing in from one is equal to the magnitude of the current flowing out from the other.

Consider a network with N terminals. Let $\mathcal{B}\subseteq (\Real^N\times\Real^N)^\Real$ denote the behavior set 
that is defined as the set of all admissible potential and current (counted positive when into the circuit) 
trajectories compatible with the network architecture at each terminal. Here $(\Real^N\times\Real^N)^\Real$ denotes 
the set of all maps $f: \Real\to\Real^N\times\Real^N$ e.g. each terminal voltage and current evolution through time 
and $\mathcal{B}$ is the restriction to the maps that are compatible with the network structure.

A subset of terminals such that potential/current behavior of the circuit implies that the sum of the currents into the 
circuit along the terminals of the subset is zero and such that the behavior is invariant under the addition of a constant 
to the potentials of the terminals of the subset, is a port. As an example, the terminals of an electrical circuit that stores no electrical 
charge should satisfy the Kirchoff laws: 
\[
(V,I) \in \mathcal{B}\,\text{and}\, \alpha\in\Real \implies (V+\alpha e,I) \in \mathcal{B}\,\text{and}\, e^T I = 0
\]
where $e=\bmatr{1&1&\cdots&1}^T$, all ones vector of length $N$. Thus, in the mechanical analogy, the behavior of the 
port should only depend on the net force acting on the port (not the absolute values) and also the directed sum of 
velocities should add up to zero. Note that a port can have arbitrary number of terminals such as a op-amps, transistors, 
$Y-\Delta$ resistance networks etc. as long as the terminals satisfy the port condition given above. 

Given a port with $n-$terminals with $V,I$ denoting the effort and the flow variables, the instantaneous power is given by 
\[
P = \sum_{k=1}^n{V_k(t)I_k(t)}
\]
and the energy transfered in the time interval is given by the total power delivered
to/from that port in the time interval $[t_1,t_2]$: 
\[
E = \int^{t_{2}}_{t_{1}}\sum_{k=1}^n{V_k(t)I_k(t)}dt
\]

As almost always stability is related to one way or the other to a finite-energy notion, the energy transfer through ports
is the main focus of network theory based modeling paradigm.

This becomes crucial when modeling multi-local multi-remote teleoperation devices
as one needs to keep track of the energy distribution between different users or different remote devices in order to 
keep the network modeling paradigm consistent with the physical system (see e.g. \cite{panzirsch}).

Without going into the philosophical roots, by far and large the most common ports among the practically relevant are the 
$2-$terminal ports that are strongly related to the second order systems (one can invoke the analogies between mechanical/%
electrical/thermal etc. systems quite early however serves no purpose for our discussion).  Since every $2-terminal$ port
can be characterized by two variables (flow and the effort variable) it is customary to characterize the interconnected
n-port networks by differential equation systems. This is done by imposing an artificial causality scheme, 
two of these time=dependent trajectories can be selected as free variables and the remaining ones become dependent variables 
(\cite{behavbook}). This is the simplest input/output modeling of physical systems via treating one port variable as the 
\emph{cause} and the other one as the \emph{effect} to this cause e.g. the current is due to the voltage drop across the terminals or 
vice versa. 




Depending on the choice of the free variables, the system can be expressed in terms of impedance, admittance 
and hybrid parameters for 2-port networks and their combinations for general $n$-port network interconnections. 
With a slight abuse of notation, we will use the term ``immitance" matrix to refer to any of these representations. 
Suppose that a 2-port immitance matrix is partitioned as
\[
\pmatr{q\\y}=\pmatr{G_1 &G_2\\G_3 &G_4}\pmatr{p\\u}
\]
where $q,y,p,u$ represent the flow (current, velocity etc.) and the effort(potential difference, force etc.) signals. Then, 
obtaining one representation from another is possible by a combination of the following elementary ``permutation" and 
``{partial} inversion" operations:
\begin{align*}
\pmatr{q\\y}&=\pmatr{G_2 &G_1\\G_4 &G_3}\pmatr{u\\p}, \ \ \ \tag{Permutation}\\
\pmatr{p\\y}&=\pmatr{\inv{G_1} &-\inv{G_1}G_2\\G_3\inv{G_1} &G_4-G_3\inv{G_1}G_2}\pmatr{q\\u}. \ \ \ \tag{Partial Inversion}
\end{align*}
In the latter operation it is assumed that the inverse exists. The existence of inverses are limiting the realizability
of networks as impedance or admittance matrices. Moreover, in \cite{andersonHmat}, it has been shown that a hybrid matrix
realization is always possible regardless of the zero diagonals. However, since we are interested in asymptotical stability
properties of networks, a simple open-circuit argument reveals that the inverses must exist.

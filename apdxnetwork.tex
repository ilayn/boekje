\chapter{Network Theory Primer}\label{chap:apdxnetwork}

We would like to follow Jan C. Willems' formulation for a systematic definition of terminals and ports which 
appeared in \cite{willemsCSM}. 


In network theory and also in its application {to bilateral teleoperation}, the ``system" refers 
to the network model that is hypothetically disconnected (thus admitting virtual terminals) from 
its ``surroundings" such as the ``load" and the ``source" of a circuit. This system is allowed to 
interact with its surroundings via ``ports" or, more generally, circuit terminal pairings that 
satisfy some technical assumptions. 


\tikz \draw[->,ultra thick] (0,0) -- ++(4cm,0) node[right,fill=yellow] {(More info on behaviors below!)};


In modeling language and block diagrams, the terminals denote the information carriage
of certain properties of the system. Hence, the reader is warned about the ambiguity of this hidden 
switch from a purely electrical view of the networks to a more general and symbolic nature of these
logical terminals. For example a spring can be modeled as a black box with two terminals, for which each terminal 
having two variables attached to it, namely the force being exerted ``to'' or ``by'' and the position of 
that terminal. Hence, when interconnected with other terminals, these variables are shared by both connected terminals.
This is the basic reformulation of the port condition from network theory, namely a pair of terminals form a port 
if the magnitude of current flowing in from one is equal to the magnitude of the current flowing out from the other.

Consider a network with N terminals. Let $\mathcal{B}\subseteq (\Real^N\times\Real^N)^\Real$ denote the behavior set 
that is defined as the set of all admissible potential and current (counted positive when into the circuit) 
trajectories compatible with the network architecture at each terminal. Here $(\Real^N\times\Real^N)^\Real$ denotes 
the set of all maps $f: \Real\to\Real^N\times\Real^N$ e.g. each terminal voltage and current evolution through time 
and $\mathcal{B}$ is the restriction to the maps that are compatible with the network structure.

A subset of terminals such that potential/current behavior of the circuit implies that the sum of the currents into the 
circuit along the terminals of the subset is zero and such that the behavior is invariant under the addition of a constant 
to the potentials of the terminals of the subset, is a port. As an example, the terminals of an electrical circuit that stores no electrical 
charge should satisfy the Kirchoff laws: 
\[
(V,I) \in \mathcal{B}\,\text{and}\, \alpha\in\Real \implies (V+\alpha e,I) \in \mathcal{B}\,\text{and}\, e^T I = 0
\]
where $e=\bmatr{1&1&\cdots&1}^T$, all ones vector of length $N$. Thus, in the mechanical analogy, the behavior of the 
port should only depend on the net force acting on the port (not the absolute values) and also the directed sum of 
velocities should add up to zero. Note that a port can have arbitrary number of terminals such as a op-amps, transistors, 
$Y-\Delta$ resistance networks etc. as long as the terminals satisfy the port condition given above. 

Given a port with $n-$terminals with $V,I$ denoting the effort and the flow variables, the instantaneous power is given by 
\[
P = \sum_{k=1}^n{V_k(t)I_k(t)}
\]
and the energy transfered in the time interval is given by the total power delivered
to/from that port in the time interval $[t_1,t_2]$: 
\[
E = \int^{t_{2}}_{t_{1}}\sum_{k=1}^n{V_k(t)I_k(t)}dt
\]

As almost always stability is related to one way or the other to a finite-energy notion, the energy transfer through ports
is the main focus of network theory based modeling paradigm.

This becomes crucial when modeling multi-local multi-remote teleoperation devices
as one needs to keep track of the energy distribution between different users or different remote devices in order to 
keep the network modeling paradigm consistent with the physical system (see e.g. \cite{panzirsch}).

A $n-$port with effort $F_i$ and flow $V_i$ variables (positive if into the system) associated 
with the $i$\textsuperscript{th} port is passive if 

\[
\int^t_0 F^T(\tau)V(\tau)d\tau \geq 0 \quad \forall t\geq 0.
\]
\begin{rem}There is no consensus in the literature on whether the above definition or the following is assumed for passivity
\[
\int^\infty_0 F^T(\tau)V(\tau)d\tau \geq 0
\]
However, it's an important difference as the former does not allow for temporary activity while the latter does. One can also extend
this definition to incremental passivity such that instead of the integrated versions, the power condition is evaluated at each time 
instance $t>0$. The former condition often appears in the nonlinear context and the latter is generally used in the linear frequency 
domain counterparts. In the IQC framework, this distinction is made via \emph{Hard IQCs} vs. \emph{Soft IQCs} however the details 
are omitted.
\end{rem}
Without delving into the philosophical reasons, by far and large the most common ports among the practically relevant are the 
$2-$terminal ports that are strongly related to the second order systems (one can invoke the analogies between mechanical/%
electrical/thermal etc. systems quite early however serves no purpose for our discussion).  Since every $2-terminal$ port
can be characterized by two variables (flow and the effort variable) it is customary to characterize the interconnected
n-port networks by differential equation systems. This is done by imposing an artificial causality scheme, 
two of these time-dependent trajectories can be selected as free variables and the remaining ones become dependent variables 
(\cite{behavbook}). This is the simplest input/output modeling of physical systems via treating one port variable as the 
\emph{cause} and the other one as the \emph{effect} to this cause e.g. the current is due to the voltage drop across the terminals or 
vice versa. 

\section{Passivity Theorem}
In this section passivity related important concepts are defined. We have neither the intention nor the possibility to give a full 
treatment of the subject. Nevertheless, the material given here is well-known and can be found in many classical sources such as
\cite{hillmoylan77,desvid,vdschaftbook}. We chose to follow closely \cite{desvid} for the presentation style. 

Let $V$ be a linear space equipped with a scalar product, $\mathcal{T}$ be the index set of time and $F$ be a class of functions 
$x:\mathcal{T}\to V$.
\begin{define} A linear truncation operator $P_T$ is defined on $F$ as 
\begin{equation}
P_T(x)(t) = \begin{cases} x(t) &\text{for } t\leq T\\ 0 &\text{for } t>T\end{cases}
\label{eq:apdx:trunc}
\end{equation}
\end{define}

In most of the cases, $\mathcal{T}$ is the closed positive real line and $V=\Real^n$, hence the scalar product becomes the inner product 
defined on $\Real^n$. The subscript ${\cdot}_T$ will be used as a shorthand notation for $P_T(\cdot)$. We also define $\mathcal{H}$ 
and its extension $\mathcal{H}_e$ as 
\[
\mathcal{H} := \left\{ x\in F\mid \norm{x}^2=\langle x_T,x_T \rangle<\infty \right\}
\]
and
\[
\mathcal{H}_e := \left\{ x\in F\mid \forall T\in\mathcal{T}, \norm{x_t}^2=\langle x_T,x_T \rangle<\infty \right\}
\]



\begin{define}Let $H:\mathcal{H}_e\to\mathcal{H}_e$. $H$ is said to be passive if and only if there exists a constant 
$\beta$ such that
\[
\langle Hx,x \rangle_T \geq \beta\quad \forall x\in\mathcal{H}_e,\forall t\in\mathcal{T}
\]
If moreover, there exists positive real number $\delta$ such that
\[
\langle Hx,x \rangle_T \geq \delta\norm{x_T}^2+ \beta\quad \forall x\in\mathcal{H}_e,\forall t\in\mathcal{T}
\]
$H$ is said to be strictly passive. The scalar $\beta$ is to model the offset for nonlinear systems and 
will be taken as zero since we would be focusing on linear systems exclusively. 
\end{define}
Suppose we are given with two dynamical systems $H_1,H_2: \mathcal{H}_e\to\mathcal{H}_e$ for which the system structure is given in time-domain with 
\begin{align}
\dot{x} &= f(x,e)\\
y &= h(x,e)
\end{align}
for both systems where $f: \Real^n\times\Real^m\mapsto \Real^n$ is locally Lipschitz and $h: \Real^n\times\Real^m\mapsto \Real^p$
is a continous function satisfying $f(0,0)=0$, $h(0,0)=0$. 

\begin{define}[Well-posedness] Consider the system interconnection given in \Cref{fig:apdx:passint}. The interconnection is said to 
be well-posed if there exist unique solutions $e_1,e_2$ to the equations
\begin{align}\label{eq:apdx:passint1}
e_1 &= u_1-h_2(x_2,e_2)\\\label{eq:apdx:passint2}
e_2 &= u_1+h_1(x_1,e_1)
\end{align}
for all admissible $(x_1,x_2,u_1,u_2)$. In the LTI case, assume that the respective transfer functions $H_1,H_2\in\RH_\infty$. Then
the interconnection is well posed if $\inv{(I-H_1H_2)}(s)$ is a proper transfer matrix.
\end{define}

Note that, in our context $u_1,u_2$ model the voluntary part of the human force input as mentioned in \Cref{chap:litsurvey}. 


\begin{figure}%
\centering
\begin{tikzpicture}[>=stealth]
\node[draw,minimum size=1cm] (h2) {$H_2$};
\node[draw,minimum size=1cm, above=1cm of h2] (h1) {$H_1$};
\node[draw,circle,inner sep=1mm,right= 1cm of h2]   (a1) {};
\node[draw,circle,inner sep=1mm,left= 1cm of h1,label={[inner sep=0]-60:$-$}] (a2) {};
\draw[->] (h1) -| (a1) node[right,midway] {$y_1$};
\draw[->] (h2) -| (a2) node[left,midway] {$y_2$};
\draw[->] (a1) -- (h2) node[above,midway] {$e_2$};
\draw[->] (a2) -- (h1) node[above,midway] {$e_1$};
\draw[<-] (a1) -- +(1cm,0) node[right] {$u_2$};
\draw[<-] (a2) -- +(-1cm,0) node[left] {$u_1$};
\end{tikzpicture}
\caption{Negative feedback interconnection}%
\label{fig:apdx:passint}%
\end{figure}


\begin{thm}Consider the well-posed feedback interconnection shown in \Cref{fig:apdx:passint} and 
described by \Cref{eq:apdx:passint1,eq:apdx:passint2}. Assume that there exist constants $\gamma_1,\beta_1,
\delta_1,\beta'_1,\epsilon_2,\beta'_2$ such that the following conditions hold
\begin{align}
\norm{H_1x}_T &\leq \gamma_1\norm{x_T}+\beta_1\\
\langle x,H_1x\rangle_T &\geq \delta_1\norm{x_T}^2+\beta'_1\\
\langle H_2x,x\rangle_T &\geq \epsilon_2\norm{x_T}^2+\beta'_2\\
\end{align}
for all $x\in\mathcal{H}_e$ and for all $T\in\mathcal{T}$. If 
\begin{equation}
\delta_1+\epsilon_2 >0
\label{eq:apdx:actpas}
\end{equation}
then, $u_1,u_2 \in \mathcal{H}$ imply that $e_1,e_2,y_1,y_2\in\mathcal{H}$ 
\end{thm}





== === =======================================


Depending on the choice of the free variables, the system can be expressed in terms of impedance, admittance 
and hybrid parameters for 2-port networks and their combinations for general $n$-port network interconnections. 
With a slight abuse of notation, we will use the term ``immitance" matrix to refer to any of these representations. 
Suppose that a 2-port immitance matrix is partitioned as
\[
\pmatr{q\\y}=\pmatr{G_1 &G_2\\G_3 &G_4}\pmatr{p\\u}
\]
where $q,y,p,u$ represent the flow (current, velocity etc.) and the effort(potential difference, force etc.) signals. Then, 
obtaining one representation from another is possible by a combination of the following elementary ``permutation" and 
``{partial} inversion" operations:
\begin{align*}
\pmatr{q\\y}&=\pmatr{G_2 &G_1\\G_4 &G_3}\pmatr{u\\p}, \ \ \ \tag{Permutation}\\
\pmatr{p\\y}&=\pmatr{\inv{G_1} &-\inv{G_1}G_2\\G_3\inv{G_1} &G_4-G_3\inv{G_1}G_2}\pmatr{q\\u}. \ \ \ \tag{Partial Inversion}
\end{align*}
In the latter operation it is assumed that the inverse exists. The existence of inverses are limiting the realizability
of networks as impedance or admittance matrices. Moreover, in \cite{andersonHmat}, it has been shown that a hybrid matrix
realization is always possible regardless of the zero diagonals. However, since we are interested in asymptotical stability
properties of networks, a simple open-circuit argument reveals that the inverses must exist.
